\documentclass[ngerman,twoside,headsepline]{scrartcl}

\usepackage{scrpage2}

\pagestyle{scrheadings}
\ofoot{\pagemark}
\lehead{Uniformisierung kompakter Riemannscher Flächen}
\rohead{\headmark}
\automark[section]{section}


\usepackage[backend=biber, style=alphabetic]{biblatex}
\bibliography{biblio}
\DefineBibliographyStrings{ngerman}{
  bibliography={Literatur}
  }

\usepackage{amssymb}
\usepackage[]{babel}
\usepackage[]{amsmath}
\usepackage{xparse}
\usepackage[colorlinks=true,linkcolor=blue,pdfborder={0 0 0}]{hyperref}
\usepackage{microtype}
%\usepackage{luacode}
\usepackage{tikz}
%\usepackage{listings}
%\usepackage{siunitx}
\usepackage{makeidx}
\usepackage{amsthm}
\usepackage{mathtools}
% \usepackage{unicode-math}
\usepackage{todonotes}


\usepackage{fontspec}
\setmainfont{Linux Libertine}
\setsansfont{Linux Biolinum}
%\fontspec[ItalicFont={Linux Libertine Italic}, BoldSlantedFont={Linux Libertine}]{Linux Libertine}

%Abkürzungen für Standardzahlmengen
\let\C\relax
\NewDocumentCommand\R{}{\mathbb{R}}
\NewDocumentCommand\Q{}{\mathbb{Q}}
\NewDocumentCommand\N{}{\mathbb{N}}
\NewDocumentCommand\C{}{\mathbb{C}}
\NewDocumentCommand\Z{}{\mathbb{Z}}
\NewDocumentCommand\A{}{\mathcal{A}}
\NewDocumentCommand\K{}{\mathbb{K}}
\NewDocumentCommand\p{}{\mathbb{P}}
\NewDocumentCommand\h{}{\mathbb{H}}
\NewDocumentCommand\F{}{\mathcal{F}}
\NewDocumentCommand\D{}{\mathcal{D}}
\NewDocumentCommand\lie{}{\mathcal{L}}
\NewDocumentCommand\jo{}{\mathfrak{J}}
\NewDocumentCommand\hol{}{\mathcal{O}}
\NewDocumentCommand\mer{}{\mathcal{M}}
\NewDocumentCommand\diff{}{\mathcal{E}}
\NewDocumentCommand\runge{}{\mathfrak{h}}
\NewDocumentCommand\fu{}{\mathfrak{U}}
\NewDocumentCommand\dist{}{\mathcal{D}}
\NewDocumentCommand\Supp{}{\operatorname{Supp}}
\NewDocumentCommand\im{}{\operatorname{im}}
\NewDocumentCommand\sm{}{\operatorname{sm}}
\NewDocumentCommand\Reg{}{\operatorname{Reg}}
\NewDocumentCommand\be{}{\mathfrak{B}}
\NewDocumentCommand\pe{}{\mathfrak{P}}
\NewDocumentCommand\res{}{\operatorname{res}}
\let\S\relax
\NewDocumentCommand\S{}{\mathcal{S}}
\let\P\relax
\NewDocumentCommand\P{}{\mathbb{P}}
\NewDocumentCommand\Fix{}{\operatorname{Fix}}
\NewDocumentCommand\Mat{}{\operatorname{Mat}}
\NewDocumentCommand\SL{}{\operatorname{SL}}
\NewDocumentCommand\GL{}{\operatorname{GL}}
\NewDocumentCommand\PSL{}{\operatorname{PSL}}



%Pfeile und Stuff
\NewDocumentCommand\Ra{}{\Rightarrow}
\NewDocumentCommand\La{}{\Leftarrow}
\NewDocumentCommand\LRa{}{\Leftrightarrow}
\NewDocumentCommand\ra{}{\rightarrow}
\NewDocumentCommand\la{}{\leftarrow}

\NewDocumentCommand\tang{ O{p} O{M}}{T_{#1}#2}
\NewDocumentCommand\cotang{ O{p} O{M}}{T^\ast_{#1}#2}
\NewDocumentCommand\del{ O{i} O{x} O{} }{\frac{\partial {#3}}{\partial {#2}^{#1}}}
\NewDocumentCommand\delat{ O{p} O{i} O{x} O{} }{\left . \del[#2][#3][#4] \right |_{#1}}
\NewDocumentCommand\christ{O{i} O{j} O{k} }{ \Gamma_{#1 #2}^{#3} }

\NewDocumentCommand\quot{m m}{\left .\raisebox{.2em}{$#1$}\middle/\raisebox{-.2em}{$#2$}\right .}

%richtiges epsilon
\let\epsilon\relax
\NewDocumentCommand\epsilon{}{\varepsilon}
\let\phi\relax
\NewDocumentCommand\phi{}{\varphi}

\let\d\relax
\NewDocumentCommand\d{ O{} }{\operatorname{d}\hspace{-0.1em}#1}
\NewDocumentCommand\arsinh{}{\operatorname{arsinh}}
\NewDocumentCommand\id{}{\operatorname{id}}
\NewDocumentCommand\supp{}{\operatorname{supp}}
\NewDocumentCommand\rank{}{\operatorname{rank}}
\NewDocumentCommand\tr{}{\operatorname{tr}}
\NewDocumentCommand\diam{}{\operatorname{diam}}
\NewDocumentCommand\ric{}{\operatorname{ric}}
\NewDocumentCommand\scal{}{\operatorname{scal}}
\NewDocumentCommand\g{m m}{\langle #1, #2 \rangle}
\NewDocumentCommand\ord{}{\operatorname{ord}}
\let\Re\relax
\NewDocumentCommand\Re{}{\operatorname{Re}}
\let\Im\relax
\NewDocumentCommand\Im{}{\operatorname{Im}}
\NewDocumentCommand\Div{}{\operatorname{Div}}
\NewDocumentCommand\Aut{}{\operatorname{Aut}}
\NewDocumentCommand\Deck{}{\operatorname{Deck}}


\NewDocumentCommand\init{m}{\emph{#1}\index{#1}}

%siunitx
%\sisetup{separate-uncertainty,exponent-product=\cdot}

%tikz
\usetikzlibrary{shadows}

%amsthm
\theoremstyle{plain}
\newtheorem{thm}{Satz}[section]
\newtheorem{lemma}[thm]{Lemma}
\newtheorem{prop}[thm]{Proposition}
\newtheorem{cor}[thm]{Korollar}
\theoremstyle{definition}
\newtheorem{defin}[thm]{Definition}
\newtheorem{bsp}[thm]{Beispiele}
\theoremstyle{remark}
\newtheorem{rem}[thm]{Bemerkung}

\tikzset{node distance=3cm, auto}

\makeindex

%%% Local Variables: 
%%% mode: latex
%%% TeX-master: "Bachelor"
%%% End: 


\title{Uniformisierung kompakter Riemannscher Flächen}
\subtitle{Bachelor-Seminar}
\author{Tim Adler}

\begin{document}

\maketitle

Als erste müssen Begriffe geklärt werden. Was ist eine Riemannsche
Fläche? Warum sind sie interessant?

In Funktionentheorie betrachtet man die komplexe Ebene $\C$ und
holomorphe bzw. meromorphe Funktionen auf ihr. Das Konzept soll
verallgemeinert werden. Wir wollen gekrümmte Flächen betrachten und
Funktionentheorie auf ihnen machen.

\begin{defin}[Riemannsche Fläche]
  \label{def:rf}
  Eine \emph{Riemannsche Fläche} $X$ ist eine 2-dimensionale
  zusammenhängende, glatte Mannigfaltigkeit, deren
  Kartenwechselabbildungen aufgefasst als Abbildungen von $\C$ nach
  $\C$ holomorph sind.
\end{defin}

Zeichne das Bild eines Torus mit Kartenabbildung.

Wir wollen, dass unsere Räume $X$ lokal wie die Ebene $\C$ aussehen,
global können sie jedoch sehr verschieden sein, was sich z.B. auf die
Existenz holomorpher und meromorpher Funktionen auswirkt.

Neue Struktur wurde definier $\Ra$ Klassifikation? 

\emph{Ziel der Arbeit:} Klassifiziere zumindest alle \emph{kompakten}
Riemannschen Flächen. Dazu muss noch kurz gesagt werden, was die
strukturerhaltenden Abbildungen sind:

\begin{defin}
  Seien $X,Y$ Riemannsche Flächen und $f: X \ra Y$ eine Abbildung. $f$
  heißt \init{holomorph}, falls für jede Karte $(U,z)$ von $X$ und
  jede Karte $(V, w)$ von $Y$ mit $f(U) \subset W$ die Abbildung
  \[
  w \circ f \circ z^{-1} : z(U) \subset \C \ra w(V) \subset \C
  \]
  holomorph im Sinne der Funktionentheorie 1 ist. $f$ heißt
  \init{biholomorph}, falls $f$ bijektiv und sowohl $f$ als auch $f^{-1}$
  holomorph sind. Zwei Riemannsche Flächen $X$ und $Y$ heißen \init{konform
  äquivalent}, falls es eine biholomorphe Abbildung zwischen ihnen gibt.
\end{defin}

Wir wollen also alle Riemannschen Flächen bis auf konforme Äquivalenz
klassifizieren. Dieses Resultat lässt sich relativ einfach
formulieren, allerdings brauchen wir noch einen Einschub über
kompakte, orientierbare topologische Flächen. Diese sind letztendlich
Riemannsche Flächen, wenn man die komplexe Struktur vergisst.

Für diese ist das Geschlecht definiert:

\begin{defin}
  Das Geschlecht einer kompakten, orientierbaren Fläche ist die Anzahl
  ihrer Löcher.
\end{defin}

Wie man sich leicht vorstellen kann, sind zwei topolgische Flächen
genau dann äquivalent, d.h. homöomorph, wenn sie das gleiche
Geschlecht haben. Denke an Tasse und Torus.

Deshalb ist es kaum verwunderlich, dass das Geschlecht auch bei unserem
Uniformisierungssatz eine wichtige Rolle spielt. 

Ein weiteres wichtiges Hilfsmittel sind Gruppenwirkungen auf
Riemannschen Flächen. Wir können jeder Fläche $X$ die Gruppe
\[
\Aut(X) = \{ f: X \ra X \mid f, f^{-1} \text{ holomorph}\}
\]
zuordnen. Sie wirkt klarerweise auf $X$ und es zeigt sich, dass
Quotientenräume aus
diskretoperierenden Untergruppen ohne Fixpunkte wieder Riemannsche
Flächen erzeugen!

\begin{thm}
  Sei $X$ eine kompakte Riemannsche Fläche mit Geschlecht $g$. Dann gilt:
  \begin{itemize}
  \item $g = 0 \quad \Ra \quad X \cong \P^1 = \C \cup \{\infty\}$
  \item $g = 1 \quad \Ra \quad X \cong \quot{\C}{\Gamma}$ mit $\Gamma
    = \Z\gamma_1 + \Z \gamma_2$, $\gamma_i$ linear unabhängig über $\R$
  \item $g \geq 2 \quad \Ra \quad X \cong \quot{\h}{G}$ und $G \leq
    \Aut(\h)$ geeignet.
  \end{itemize}
\end{thm}

Was ist die Herangehensweise an diesen Satz? Auffälligkeit: $\C, \h$
und $\P^1$ sind einfach zusammenhängend (haben also keine Löcher).

Können wir jeder kompakten Riemannschen Fläche eine andere nicht
notwendigerweise kompkate Riemannsche Fläche
ohne Löcher zuordnen? Antwort: Ja! und die Theorie heißt
Überlagerungstheorie.

\begin{thm}
  Sei $X$ eine kompakte Riemannsche Fläche, dann existiert eine
  einfachzusammenhängende Riemannsche Fläche $\tilde X$ und eine
  diskret- und fixpunktfreioperierende Untergruppe $G \leq \Aut(\tilde
  X)$, so dass
  \[
  X \cong \quot{\tilde X}{G}
  \]
  gilt. Außerdem ist $G \cong \pi_1(X)$.
\end{thm}

Bild für Fundamentalgruppe malen.

Dieser Satz ist schon alles andere als trivial, aber wir müssen ihn
für diesen Vortrag hinnehmen.

Beweisidee: Wir charakterisieren einfachzusammenhängende Riemannsche
Flächen und deren Automorphismengruppen. Dann haben wir alle kompakten
verstanden.

In der Funktionentheorie 1 wird der (kleine) Riemannsche
Abbildungssatz bewiesen er besagt, dass jedes echte einfach zsh. Gebiet in $\C$
konform äquivalent zum Einheitskreis und damit zu $\h$ ist. Wir
verallgemeinern dieses Resultat auf Riemannsche Flächen. Dort lautet
es:

\begin{thm}
  Sei $X$ eine Riemannsche Fläche $Y \subset X$ ein
  einfachzusammenhängendes Gebiet. Dann ist $Y$ konform äquivalent zu
  $\P^1, \C$ oder $\h$.
\end{thm}

Der Großteil der Arbeit beschäftigt sich mit diesem
Resultat. Letztendlich gibt es viele Parallelen zum Beweis wie in der
Funktionentheorie 1, aber es müssen eben erstmal viele der Sätze der
Funktionentheorie auf Riemannsche Flächen ausgeweitet werden.


\begin{tikzpicture}
  \node (uni) {Uni};
  \node (ül) [above of=uni]{Überlagerung};
  \node (rmt) [above left of=ül]{RMT};
  \node (auto) [right of=rmt]{Auto};
  \node (weier) [above of=rmt]{Weierstraß};
  \node (runge) [left of =weier]{Runge};
  \node (diric) [right of=weier]{Dirchlet};
  \node (roch) [above right of=uni]{RiRo};
  \draw[->] (weier) to (rmt);
  \draw[->] (runge) to (rmt);
  \draw[->] (diric) to (rmt);
  \draw[->] (rmt) to (ül);
  \draw[->] (auto) to (ül);
  \draw[->] (ül) to (uni);
  \draw[->] (roch) to (uni);
\end{tikzpicture}

Für große Teile dies RMT ist wichtig, dass Funktionentheorie auf
\emph{nicht-}kompakten Riemannschen Flächen quasi genau so
funktioniert wie in $\C$. Dazu müssen wir aber erstmal genügend viele
holomorphe Funktionen konstruieren können. Dazu nutzen wir die enge
Verwandtschaft zwischen harmonischen und holomorphen Funktionen aus.

Den Rest der Zeit wollen wir uns nun mit den Automorphismengruppen von
$\C, \h$ und $\P^1$ auseinandersetzen. Wir erinnern uns an die
Funktionentheorie 1:

\begin{thm}
  \begin{itemize}
  \item $\Aut(\C) = \{z \mapsto az +b \mid a,b \in \C, a \neq 0 \}$
  \item $\Aut(\P^1) = \PSL(2,\C)$
  \item $\Aut(\h) = \PSL(2, \R)$
  \end{itemize}
\end{thm}

$\Aut(\C)$ berechnet sich über den Satz von
Casorati-Weierstraß. $\Aut(\P^1)$ lässt sich auf $\Aut(\C)$
zurückführen. $\Aut(\h)$ ist etwas involvierter und wird am
einfachsten mit $\Aut(B)$ geführt.

Wir versuchen die Automorphismengruppen besser zu verstehen und
beginnen mit $\P^1$.

\begin{prop}
  Jedes Element $M \in \Aut(\P^1)$ besitzt einen Fixpunkt.
\end{prop}

\begin{proof}
  Sei
  \[
  M =
  \begin{pmatrix}
    a & b\\
    c & d
  \end{pmatrix}.
  \]
  Dann muss ein Fixpunkt $M\langle z \rangle = z$ die Gleichung
  \[
  cz^2 + (d-a)z - b = 0
  \]
  erfüllen. 
  Falls $c \neq 0$ gilt, so erhalten wir als Fixpunkte
  \[
  z_{1,2} = \frac{a}{2c} \pm \sqrt{\frac{a^2}{4 c^2} - \frac{d-b}{c}}.
  \]
  Falls $c = 0$ gilt, so muss $a \neq d$ gelten, ansonsten wäre
  $\det M = 0$ und wir erhalten als Fixpunkt
  \[
  z = \frac{b}{d-a}.
  \]
\end{proof}

Das bedeutet $\P^1$ kann nur sich selbst überlagern und nichts
anderes!

Nun wollen wir $\Aut(\C)$ etwas besser verstehen. Dazu wenden wir uns
kurz dem $\R^n$ und Gittern zu:

\begin{defin}
  \label{defin:gitter}
  Sei $V$ ein $n-$dimensionaler, reeller Vektorraum. Eine additive
  Untergruppe $\Gamma \subset V$ heißt \init{Gitter}, falls $n$ linear
  unabhängige $\gamma_1, \dots, \gamma_n \in V$ existieren, so dass
  \[
  \Gamma = \Z \gamma_1 + \dots + \Z \gamma_2
  \]
  gilt.
\end{defin}


\begin{lemma}
  \label{lemma:gitter}
  Sei $V$ ein reeller Vektorraum und $\Gamma \subset V$ eine additive
  Untergruppe. Dann ist $\Gamma$ genau dann ein Gitter, wenn
  \begin{enumerate}
  \item $\Gamma$ diskret ist und
  \item $\Gamma$ nicht in einem echten Untervektorraum von $V$
    enthalten ist.
  \end{enumerate}
\end{lemma}

\begin{proof}
  Falls $\Gamma$ ein Gitter ist, ist die Aussage aus der Definition
  klar. Erfülle also $\Gamma$ die beiden Bedingungen. Wir gehen nun
  per Induktion über $n = \dim V$ vor. Für $n = 0$ ist die Aussage klar. Gelte also die
  Aussage für ein $n \in \N_0$. Da $\Gamma \subset V$ mit $\dim V =
  n+1$ in keinem echten
  Untervektorraum von $V$ enthalten ist, existieren linear unabhängige $x_1, \dots,
  x_{n+1} \in \Gamma$. Sei $V_1 := \langle x_1, \dots, x_n \rangle$
  und $\Gamma_1 := \Gamma \cap V_1$. Nun ist $\Gamma_1$ wieder diskret
  und in keinem echten Untervektorraum von $V_1$ enthalten. Weiterhin
  ist $\dim V_1 = n$, also können wir die Induktionsvoraussetzung
  verwenden und erhalten linear unabhängige $\gamma_1, \dots, \gamma_n
  \in \Gamma_1 \subset \Gamma$, so dass $\Gamma_1 = \Z \gamma_1 +
  \dots + \Z \gamma_n$ gilt. Nun bilden die $\gamma_i$ bereits eine
  Basis von $V_1$ und fügen wir $x_{n+1}$ hinzu, erhalten wir sogar eine Basis
  von $V$, das bedeutet aber, dass wir zu jedem $x \in \Gamma \subset
  V$, eindeutig bestimmte $c_i(x), c(x) \in \R$ finden, so dass
  \[
  x = c_1(x) \gamma_1 + \dots + c_n(x) \gamma_n + c(x) x_{n+1}
  \]
  gilt. Wir betrachten nun
  \[
  P := \{ \lambda_1 \gamma_1 + \dots + \lambda_n \gamma_n + \lambda
  x_{n+1} \mid \lambda_i, \lambda \in [0,1] \}.
  \]
  Dieses $P$ ist kompakt und da $\Gamma$ diskret ist, folgt dass $P
  \cap \Gamma$ endlich ist. Weiterhin ist $( \Gamma \cap P) \setminus
  V_1$ nicht leer, da $x_{n+1}$ enthalten ist. Damit existiert ein
  $\gamma_{n+1} \in ( \Gamma \cap P) \setminus V_1$ mit
  \[
  c(\gamma_{n+1}) = \min \{ c(x) \mid x \in ( \Gamma \cap P ) \setminus
  V_1 \} \in ]0, 1].
  \]
  Wir behaupten nun, dass $\Gamma = \Gamma_1 + \Z \gamma_{n+1}$ ist. Sei
  dazu $x \in \Gamma$. Dann existieren $n_j \in \Z$, so dass
  \[
  x' := x - \sum_{j=1}^{n+1} n_j \gamma_j = \sum_{j=1}^n \lambda_j
  \gamma_j + \lambda x_{n+1}
  \]
  mit $0 \leq \lambda_j < 1$ und $0 \leq \lambda < c(\gamma_{n+1})$
  gilt. Da $x' \in \Gamma \cap P$ liegt und $c(\gamma_{n+1})$ minimal
  von 0 verschieden gewählt wurde, muss $\lambda = 0$ sein. Also liegt
  $x' \in \Gamma \cap V_1 = \Gamma_1$. Damit sind die $\lambda_i$
  ganzzahlig und die einzig verbleibende Möglichkeit ist $\lambda_i =
  0$. Dann ist aber $x'$ bereits 0 und wir erhalten 
  \[
  x = \sum_{j=1}^n n_j \gamma_j \in \Z \gamma_1 + \dots + \Z \gamma_n.
  \]
\end{proof}

\begin{prop}
  Sei $\Gamma \subset \Aut(\C)$ eine fixpunktfrei- und
  diskretoperrierende Untergruppe dann ist $\Gamma$ eine der folgenden
  Gruppen:
  \begin{itemize}
  \item $\Gamma = \{\id\}$
  \item $\Gamma = \{z \mapsto z + n\gamma \mid n \in \Z\}$ mit $\gamma \in \C^\ast$.
  \item $\Gamma = \{z \mapsto z + n\gamma_1 + m \gamma_2 \mid m,n \in
    \Z\}$ mit $\gamma_1, \gamma_2 \in \C$ linear unabhängig über $\R$.
  \end{itemize}
\end{prop}

\begin{proof}
  Sei $f \in \Gamma$. Dann existieren $a \in \C^\ast$ und $b \in
  \C$, so dass $f(z) = az +b$ für beliebige $z \in \C$
  gilt. Angenommen $a \neq 1$. Dann ist
  \[
  z = \frac{b}{1-a}
  \]
  ein Fixpunkt. Also ist $f(z) = z + b$ für alle $z \in \C$. Wir
  definieren
  \[
  \tilde \Gamma = \{f(0) \mid f \in \Gamma\} \leq \C.
  \]
  Dann ist
  \[
  \Gamma = \{z \mapsto z + \gamma \mid \gamma \in \tilde \Gamma \}
  \]
  Da $\Gamma$ diskret operiert ist $\tilde \Gamma$ eine diskrete
  Untergruppe. Die haben wir aber charakterisiert und damit ist
  $\tilde \Gamma$ entweder ein 0, 1 oder 2 dimensionales Gitter. Das
  entspricht aber gerade den obigen drei Fällen.
\end{proof}

Nun sollten wir $\Aut(\h)$ als nächste verstehen. Das ist jedoch eine
sehr schwierige Aufgabe und füllt ganze Bücher. Wir begnügen uns an
dieser Stelle mit der Aussage, dass eine fixpunktfrei- und
diskretoperierende Untergruppe die abelsch ist, automatisch zyklisch
sein muss.

\begin{thm}
  \label{thm:abelsch-zyklisch}
  Sei $G \leq \PSL(2, \R)$ eine Fuchssche Gruppe. Dann ist $G$ genau
  dann abelsch, wenn $G$ zyklisch ist.
\end{thm}

Jetzt können wir versuchen den Uniformisierungssatz zu beweisen.

\begin{proof}
  Der erste Fall ergibt sich direkt aus dem Satz von Riemann-Roch
  \cite[Kor. 16.13]{For} und müssen wir leider als gegeben annehmen.

  Sei
  nun also $g \geq 1$ und bezeichne $\tilde X$ die Universelle
  Überlagerung von $X$. Dann ist $\tilde X$ einfach zusammenhängend
  und aus dem Riemannschen Abbildungssatz \ref{thm:rmt} folgt, dass
  $\tilde X$ konform äquivalent zu $\P^1$, $\C$ oder $B$ ist. Da wir
  wissen, dass $B$ konform äquivalent zu $\h$ ist, können wir in der
  Betrachtung genau so gut $\h$ verwenden. Nun wissen wir nach Lemma
  \ref{lemma:deck-pc}, dass jeder Automorphismus von $\P^1$ einen
  Fixpunkt besitzt, d.h. es gibt keine fixpunktfrei-operierende
  Untergruppe von $\Aut(\P^1)$. Also überlagert $\P^1$ nur sich
  selbst und es müsste $X \cong \tilde X \cong \P^1$ gelten. Dies ist
  ein Widerspruch zu $g \neq 0$. Dementsprechend ist die Universelle Überlagerung
  von $X$ entweder $\C$ oder $\h$.

  Sei nun $g \geq 2$. Angenommen $\tilde X \cong \C$. In Lemma
  \ref{lemma:deck-pc} wurden alle diskreten, fixpunkfreioperierenden
  Untergruppen von $\Aut(\C)$ bestimmt. Die zugehörigen Riemannschen
  Flächen sind dann $X \cong \C$, $X \cong \C^\times$ oder $X \cong
  \quot{\C}{\Gamma}$ für ein Gitter $\Gamma \subset \C$. Nun sind aber
  die ersten beiden Möglichkeiten nicht kompakt und die Dritte hat
  Geschlecht $g =1$. Ein Widerspruch. Also muss die Universelle
  Überlagerung von $X$ konform äquivalent zu $\h$ sein und damit ist
  $X$ konform äquivalent zu $\quot{\h}{G}$ für eine Fuchssche Gruppe $G
  \leq \Aut(\h)$.

  Nun fehlt noch der Fall $g = 1$. Angenommen $\tilde X \cong \h$. Wir
  wissen aus Satz \ref{thm:geschlecht-1-z}, dass
  \begin{align}
    \label{eq:g-z}
  \Z \oplus \Z \cong G := \Deck(\tilde X \setminus X) \leq \Aut(\h)
  \end{align}
  gelten müsste. Insbesondere müssten wir eine abelsche Fuchssche
  Gruppe finden, allerdings wissen wir nach Satz
  \ref{thm:abelsch-zyklisch}, dass diese alle zyklisch sind. Damit ist
  $G$ also isomorph zu $\Z$ oder zu $\quot{\Z}{n\Z}$ für ein $n \in
  \N$. Dies ist ein Widerspruch dazu, dass Gleichung \eqref{eq:g-z}
  gelten soll. Also ist die Universelle Überlagrung von $X$ konform
  äquivalent zu $\C$. Die Gruppe der Decktransformation ist also eine
  diskrete abelsche Untergruppe von $\C$ und muss isomorph zu $\Z
  \oplus \Z$ sein. In Lemma \ref{lemma:deck-pc} sind die möglichen diskreten
  Untergruppen von $\Aut(\C)$ charakterisiert und die einzige Möglichkeit, die
  isomorph zu $\Z \oplus \Z$ ist, ist die des Gitters. Also ist die
  $\Deck(\tilde X \setminus X)$ isomorph zu einem Gitter $\Gamma
  \subset \C$. Insgesamt erhalten wir, dass $X$ konform äquvialent zu
  $\quot{\C}{\Gamma}$ ist.
\end{proof}


\end{document}