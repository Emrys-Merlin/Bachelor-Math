\documentclass[ngerman,headsepline]{scrartcl}


\usepackage{scrpage2}

\pagestyle{scrheadings}
\ofoot{\pagemark}
\lehead{Uniformisierung kompakter Riemannscher Flächen}
\rohead{\headmark}
\automark[section]{section}


\usepackage[backend=biber, style=alphabetic]{biblatex}
\bibliography{biblio}
\DefineBibliographyStrings{ngerman}{
  bibliography={Literatur}
  }

\usepackage{amssymb}
\usepackage[]{babel}
\usepackage[]{amsmath}
\usepackage{xparse}
\usepackage[colorlinks=true,linkcolor=blue,pdfborder={0 0 0}]{hyperref}
\usepackage{microtype}
%\usepackage{luacode}
\usepackage{tikz}
%\usepackage{listings}
%\usepackage{siunitx}
\usepackage{makeidx}
\usepackage{amsthm}
\usepackage{mathtools}
% \usepackage{unicode-math}
\usepackage{todonotes}


\usepackage{fontspec}
\setmainfont[]{Linux Libertine O}
\setsansfont{Linux Biolinum O}
% \setmathfont{xits-math.otf}
% \setmathfont{Asana-Math.otf}

%\fontspec[ItalicFont={Linux Libertine Italic}, BoldSlantedFont={Linux Libertine}]{Linux Libertine}

%Abkürzungen für Standardzahlmengen
\let\C\relax
\NewDocumentCommand\R{}{\mathbb{R}}
\NewDocumentCommand\Q{}{\mathbb{Q}}
\NewDocumentCommand\N{}{\mathbb{N}}
\NewDocumentCommand\C{}{\mathbb{C}}
\NewDocumentCommand\Z{}{\mathbb{Z}}
\NewDocumentCommand\A{}{\mathcal{A}}
\NewDocumentCommand\K{}{\mathbb{K}}
\NewDocumentCommand\p{}{\mathbb{P}}
\NewDocumentCommand\h{}{\mathbb{H}}
\NewDocumentCommand\F{}{\mathcal{F}}
\NewDocumentCommand\D{}{\mathcal{D}}
\NewDocumentCommand\lie{}{\mathcal{L}}
\NewDocumentCommand\jo{}{\mathfrak{J}}
\NewDocumentCommand\hol{}{\mathcal{O}}
\NewDocumentCommand\mer{}{\mathcal{M}}
\NewDocumentCommand\diff{}{\mathcal{E}}
\NewDocumentCommand\runge{}{\mathfrak{h}}
\NewDocumentCommand\fu{}{\mathfrak{U}}
\NewDocumentCommand\dist{}{\mathcal{D}}
\NewDocumentCommand\Supp{}{\operatorname{Supp}}
\NewDocumentCommand\im{}{\operatorname{im}}
\NewDocumentCommand\sm{}{\operatorname{sm}}
\NewDocumentCommand\Reg{}{\operatorname{Reg}}
\NewDocumentCommand\be{}{\mathfrak{B}}
\NewDocumentCommand\pe{}{\mathfrak{P}}
\NewDocumentCommand\res{}{\operatorname{res}}
\let\S\relax
\NewDocumentCommand\S{}{\mathcal{S}}
\let\P\relax
\NewDocumentCommand\P{}{\mathbb{P}}
\NewDocumentCommand\Fix{}{\operatorname{Fix}}
\NewDocumentCommand\Mat{}{\operatorname{Mat}}
\NewDocumentCommand\SL{}{\operatorname{SL}}
\NewDocumentCommand\GL{}{\operatorname{GL}}
\NewDocumentCommand\PSL{}{\operatorname{PSL}}



%Pfeile und Stuff
\NewDocumentCommand\Ra{}{\Rightarrow}
\NewDocumentCommand\La{}{\Leftarrow}
\NewDocumentCommand\LRa{}{\Leftrightarrow}
\NewDocumentCommand\ra{}{\rightarrow}
\NewDocumentCommand\la{}{\leftarrow}

\NewDocumentCommand\tang{ O{p} O{M}}{T_{#1}#2}
\NewDocumentCommand\cotang{ O{p} O{M}}{T^\ast_{#1}#2}
\NewDocumentCommand\del{ O{i} O{x} O{} }{\frac{\partial {#3}}{\partial {#2}^{#1}}}
\NewDocumentCommand\delat{ O{p} O{i} O{x} O{} }{\left . \del[#2][#3][#4] \right |_{#1}}
\NewDocumentCommand\christ{O{i} O{j} O{k} }{ \Gamma_{#1 #2}^{#3} }

\NewDocumentCommand\quot{m m}{\left .\raisebox{.2em}{$#1$}\middle/\raisebox{-.2em}{$#2$}\right .}

%richtiges epsilon
\let\epsilon\relax
\NewDocumentCommand\epsilon{}{\varepsilon}
\let\phi\relax
\NewDocumentCommand\phi{}{\varphi}

\let\d\relax
\NewDocumentCommand\d{ O{} }{\operatorname{d}\hspace{-0.1em}#1}
\NewDocumentCommand\arsinh{}{\operatorname{arsinh}}
\NewDocumentCommand\id{}{\operatorname{id}}
\NewDocumentCommand\supp{}{\operatorname{supp}}
\NewDocumentCommand\rank{}{\operatorname{rank}}
\NewDocumentCommand\tr{}{\operatorname{tr}}
\NewDocumentCommand\diam{}{\operatorname{diam}}
\NewDocumentCommand\ric{}{\operatorname{ric}}
\NewDocumentCommand\scal{}{\operatorname{scal}}
\NewDocumentCommand\g{m m}{\langle #1, #2 \rangle}
\NewDocumentCommand\ord{}{\operatorname{ord}}
\let\Re\relax
\NewDocumentCommand\Re{}{\operatorname{Re}}
\let\Im\relax
\NewDocumentCommand\Im{}{\operatorname{Im}}
\NewDocumentCommand\Div{}{\operatorname{Div}}
\NewDocumentCommand\Aut{}{\operatorname{Aut}}
\NewDocumentCommand\Deck{}{\operatorname{Deck}}


\NewDocumentCommand\init{m}{\emph{#1}\index{#1}}

%siunitx
%\sisetup{separate-uncertainty,exponent-product=\cdot}

%tikz
\usetikzlibrary{shadows}

%amsthm
%\theoremstyle{plain}
\theoremstyle{definition}
\newtheorem{thm}{Satz}[section]
\newtheorem{lemma}[thm]{Lemma}
\newtheorem{prop}[thm]{Proposition}
\newtheorem{cor}[thm]{Korollar}
%\theoremstyle{definition}
\newtheorem{defin}[thm]{Definition}
\newtheorem{bsp}[thm]{Beispiele}
%\theoremstyle{remark}
\newtheorem{rem}[thm]{Bemerkung}

\tikzset{node distance=3cm, auto}

\makeindex

%%% Local Variables: 
%%% mode: latex
%%% TeX-master: "Bachelor"
%%% End: 
