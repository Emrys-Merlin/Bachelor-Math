\message{ !name(../Bachelor.tex)}\documentclass[ngerman,twoside,headsepline, titlepage=true]{scrartcl}


\usepackage{scrpage2}

\pagestyle{scrheadings}
\ofoot{\pagemark}
\lehead{Uniformisierung kompakter Riemannscher Flächen}
\rohead{\headmark}
\automark[section]{section}


\usepackage[backend=biber, style=alphabetic]{biblatex}
\bibliography{biblio}
\DefineBibliographyStrings{ngerman}{
  bibliography={Literatur}
  }

\usepackage{amssymb}
\usepackage[]{babel}
\usepackage[]{amsmath}
\usepackage{xparse}
\usepackage[colorlinks=true,linkcolor=blue,pdfborder={0 0 0}]{hyperref}
\usepackage{microtype}
%\usepackage{luacode}
\usepackage{tikz}
%\usepackage{listings}
%\usepackage{siunitx}
\usepackage{makeidx}
\usepackage{amsthm}
\usepackage{mathtools}
% \usepackage{unicode-math}
\usepackage{todonotes}


\usepackage{fontspec}
\setmainfont[ItalicFont={Linux Biolinum Italic}]{Linux Libertine O}
\setsansfont{Linux Biolinum O}
% \setmathfont{xits-math.otf}
% \setmathfont{Asana-Math.otf}

%\fontspec[ItalicFont={Linux Libertine Italic}, BoldSlantedFont={Linux Libertine}]{Linux Libertine}

%Abkürzungen für Standardzahlmengen
\let\C\relax
\NewDocumentCommand\R{}{\mathbb{R}}
\NewDocumentCommand\Q{}{\mathbb{Q}}
\NewDocumentCommand\N{}{\mathbb{N}}
\NewDocumentCommand\C{}{\mathbb{C}}
\NewDocumentCommand\Z{}{\mathbb{Z}}
\NewDocumentCommand\A{}{\mathcal{A}}
\NewDocumentCommand\K{}{\mathbb{K}}
\NewDocumentCommand\p{}{\mathbb{P}}
\NewDocumentCommand\h{}{\mathbb{H}}
\NewDocumentCommand\F{}{\mathcal{F}}
\NewDocumentCommand\D{}{\mathcal{D}}
\NewDocumentCommand\lie{}{\mathcal{L}}
\NewDocumentCommand\jo{}{\mathfrak{J}}
\NewDocumentCommand\hol{}{\mathcal{O}}
\NewDocumentCommand\mer{}{\mathcal{M}}
\NewDocumentCommand\diff{}{\mathcal{E}}
\NewDocumentCommand\runge{}{\mathfrak{h}}
\NewDocumentCommand\fu{}{\mathfrak{U}}
\NewDocumentCommand\dist{}{\mathcal{D}}
\NewDocumentCommand\Supp{}{\operatorname{Supp}}
\NewDocumentCommand\im{}{\operatorname{im}}
\NewDocumentCommand\sm{}{\operatorname{sm}}
\NewDocumentCommand\Reg{}{\operatorname{Reg}}
\NewDocumentCommand\be{}{\mathfrak{B}}
\NewDocumentCommand\pe{}{\mathfrak{P}}
\NewDocumentCommand\res{}{\operatorname{res}}
\let\S\relax
\NewDocumentCommand\S{}{\mathcal{S}}
\let\P\relax
\NewDocumentCommand\P{}{\mathbb{P}}
\NewDocumentCommand\Fix{}{\operatorname{Fix}}
\NewDocumentCommand\Mat{}{\operatorname{Mat}}
\NewDocumentCommand\SL{}{\operatorname{SL}}
\NewDocumentCommand\GL{}{\operatorname{GL}}
\NewDocumentCommand\PSL{}{\operatorname{PSL}}



%Pfeile und Stuff
\NewDocumentCommand\Ra{}{\Rightarrow}
\NewDocumentCommand\La{}{\Leftarrow}
\NewDocumentCommand\LRa{}{\Leftrightarrow}
\NewDocumentCommand\ra{}{\rightarrow}
\NewDocumentCommand\la{}{\leftarrow}

\NewDocumentCommand\tang{ O{p} O{M}}{T_{#1}#2}
\NewDocumentCommand\cotang{ O{p} O{M}}{T^\ast_{#1}#2}
\NewDocumentCommand\del{ O{i} O{x} O{} }{\frac{\partial {#3}}{\partial {#2}^{#1}}}
\NewDocumentCommand\delat{ O{p} O{i} O{x} O{} }{\left . \del[#2][#3][#4] \right |_{#1}}
\NewDocumentCommand\christ{O{i} O{j} O{k} }{ \Gamma_{#1 #2}^{#3} }

\NewDocumentCommand\quot{m m}{\left .\raisebox{.2em}{$#1$}\middle/\raisebox{-.2em}{$#2$}\right .}

%richtiges epsilon
\let\epsilon\relax
\NewDocumentCommand\epsilon{}{\varepsilon}
\let\phi\relax
\NewDocumentCommand\phi{}{\varphi}

\let\d\relax
\NewDocumentCommand\d{ O{} }{\operatorname{d}\hspace{-0.1em}#1}
\NewDocumentCommand\arsinh{}{\operatorname{arsinh}}
\NewDocumentCommand\id{}{\operatorname{id}}
\NewDocumentCommand\supp{}{\operatorname{supp}}
\NewDocumentCommand\rank{}{\operatorname{rank}}
\NewDocumentCommand\tr{}{\operatorname{tr}}
\NewDocumentCommand\diam{}{\operatorname{diam}}
\NewDocumentCommand\ric{}{\operatorname{ric}}
\NewDocumentCommand\scal{}{\operatorname{scal}}
\NewDocumentCommand\g{m m}{\langle #1, #2 \rangle}
\NewDocumentCommand\ord{}{\operatorname{ord}}
\let\Re\relax
\NewDocumentCommand\Re{}{\operatorname{Re}}
\let\Im\relax
\NewDocumentCommand\Im{}{\operatorname{Im}}
\NewDocumentCommand\Div{}{\operatorname{Div}}
\NewDocumentCommand\Aut{}{\operatorname{Aut}}
\NewDocumentCommand\Deck{}{\operatorname{Deck}}


\NewDocumentCommand\init{m}{\emph{#1}\index{#1}}

%siunitx
%\sisetup{separate-uncertainty,exponent-product=\cdot}

%tikz
\usetikzlibrary{shadows}

%amsthm
\theoremstyle{plain}
\newtheorem{thm}{Satz}[section]
\newtheorem{lemma}[thm]{Lemma}
\newtheorem{prop}[thm]{Proposition}
\newtheorem{cor}[thm]{Korollar}
\theoremstyle{definition}
\newtheorem{defin}[thm]{Definition}
\newtheorem{bsp}[thm]{Beispiele}
\theoremstyle{remark}
\newtheorem{rem}[thm]{Bemerkung}

\tikzset{node distance=3cm, auto}

\makeindex

%%% Local Variables: 
%%% mode: latex
%%% TeX-master: "Bachelor"
%%% End: 


\title{Uniformisierung kompakter riemannscher Flächen}
\author{Tim Adler}


\begin{document}

\message{ !name(content/Runge.tex) !offset(-9) }

\section{Der Rungesche Approximationssatz}
\label{sec:Runge}

% TODO: Frechet-Raum und Konsequenzen aufschreiben

\begin{prop}
  Sei $X$ eine Riemannsche Fläche, $Y \subset X$ offen. \\
  Dann besitzen $\diff(Y)$ und $\diff^{0,1}(Y)$ Struktur eines Fr´echet-Raums.
\end{prop}

\begin{proof}
  % TODO: Warum genau geht das? Wieso reichen abzählbarviele Karten?
  % (Rest folgt dann)
  Sei dazu $K_j \subset Y$ kompakt, $j \in \N$, $\bigcup_{j \in \N}
  \mathring{K_j} = Y$ und $K_j \subset U_j$, wobei $(U_j, z_j)$ Karten
  sind. \\
  Für jedes $\nu = (\nu_1, \nu_2)^T \in \N^2$ eine Halbnorm
  \[
  p_{j\nu}: \diff(Y) \ra \R, \quad p_{j\nu}(f) := \|D_j^\nu f\|_{K_j},
  \]
  wobei $D^\nu_j := \left ( \frac{\partial}{\partial x_j} \right
  )^{\nu_1} \left ( \frac{\partial}{\partial y_j} \right )^{\nu_2}$
  mit $z_j = x_j + iy_j$. \\
  % TODO: Genauer
  Dies ist eine abzählbare Familie von Halbnormen. Da falls $0 \neq f
  \in \diff(Y)$ gilt, immer eine der Halbnormen von 0 verschieden ist,
  definiert diese Familie die Struktur eines Fr´echet-Raums auf
  $\diff(Y)$. \\
  Wir erhalten eine Umgebungsbasis der 0 durch endliche Schnitte von
  Mengen der Form $U(p_{j\nu}, \epsilon) := \{f \in \diff(Y) |
  p_{j\nu}(f) < \epsilon \}, \quad \epsilon > 0$. \\
  Konvergenz in dieser Topologie ist dann gleichbedeutend mit
  $C^\infty$-Konvergenz auf jedem der $K_j$. \\
  % TODO: Warum genau?
  Behauptung: Die Topologie ist unabhängig von den gewählten $K_j$.\\
  \\
  Völlig analog kann die obige Konstruktion auch für $\diff^{0,1}(Y)$
  durchgeführt werden.
\end{proof}


\begin{cor}
  Mit der obigen Topologie ist $\hol(Y) \subset \diff(Y)$ ein
  abgeschlossener Teilraum und die Konvergenz bzgl. dieser Topologie
  stimmt mit der kompakten Konvergenz überein.
\end{cor}

\begin{lemma}
  \label{lemma:kompakter-träger-funktional}
  Sei $Y \subset X$ offen, $X$ eine Riemannsche Fläche. Sei $T \in
  \diff(Y)'$. \\
  Dann hat $T$ kompakten Träger, d.h. es existiert ein $K \subset Y$
  kompakt, so dass
  \[
  T[f] = 0 \qquad \forall f \in \diff(Y) \text{ mit } \Supp(f) \subset
  Y \setminus K
  \]
  Das gleiche Resultat gilt für $\diff^{0,1}(Y)$.
\end{lemma}

\begin{proof}
  Aus der Stetigkeit von $T$ folgt die Existenz einer offenen
  Nullumgebung $U \subset \diff(Y)$, so dass
  \[
  |T[f]| < 1 \qquad \forall f \in U
  \]
  Nach Konstruktion der obigen Topologie finden wir ein $\epsilon > 0$
  und endlich viele $j_1, \dots, j_n \in \N$ und $\nu_1, \dots \nu_n
  \in \N^2$, so dass $U(p_{j_1\nu_1}, \epsilon) \cap \dots \cap
  U(p_{j_n\nu_n}, \epsilon) \subset U$. \\
  Sei nun  $K := K_{j_1} \cup \dots \cup K_{j_n}$. Wir behaupten, dass
  dies das gesuchte $K$ ist. \\
  Dazu wählen wir $f \in \diff(Y)$ mit $\Supp(f) \subset Y \setminus
  K$. Dann gilt für beliebige
  $\lambda > 0$:
  \[
  p_{j_1\nu_1}(\lambda f) = \dots = p_{j_n\nu_n}(\lambda f) = 0
  \]
  Damit liegt $\lambda f \in U$ und es folgt:
  \[
  |T[f]| < \frac1\lambda \qquad \forall \lambda >0
  \]
  und damit $T[f] = 0$.
\end{proof}

\begin{lemma}
  \label{lemma:Funktional-explizit}
  Sei $Z \subset X$ offen und $X$ eine Riemannsche Fläche. Sei $S \in
  \diff^{0,1}(X)'$ mit $S[\d[''g]] = 0$ für beliebige $g \in \diff(X)$
  mit $\Supp(g) \Subset Z$. \\
  Dann existiert ein $\sigma \in \Omega(X)$ mit $S[\omega] = \iint_Z
  \sigma \wedge \omega$ für jedes $\omega \in \diff^{0,1}(X)$ mit
  $\Supp(\omega) \Subset Z$.
\end{lemma}

\begin{proof}
  Sei $z: U \ra V \subset \C$ eine Karte von $X$ mit $U \subset
  Z$. Wir identifizieren $U$ mit $V$ und können deshalb vom Raum der
  Testfunktionen $\dist(U)$ auf $U$ sprechen. Sei also $\phi \in
  \dist(U)$. Wir bezeichnen mit $\tilde \phi$ jede 1-Form in
  $\dist^{0,1}(X)$, für die $\tilde \phi|_U \cong \phi \d[\bar z]$ und
  $\tilde \phi |_{U \setminus U} \cong 0$ gilt. \\
  Damit können wir ein $S_U \in \dist(U)'$ definieren:
  \[
  S_U: \dist(U) \ra \C, \phi \mapsto S[\tilde \phi]
  \]
  Diese verschwindet auf allen Funktionen $\phi = \frac{\partial
    g}{\partial \bar z}$, $g \in \dist(U)$, denn $S_U[\phi] = S[\tilde
  \phi] = S[ \frac{\partial g}{\partial \bar z} \d[\bar z]] =
  S[\d[''g]] = 0$ nach Voraussetzung. Das bedeutet aber, dass
  % TODO: richtige Referenz
  $\frac{\partial S_U}{\partial \bar z} = 0$ und aus (24.10) erhalten
  wir ein $h \in \hol(U)$:
  \[
  S[\tilde \phi] = \iint_U h(z) \phi(z) \d[z] \wedge \d[\bar z] \qquad
  \forall \phi \in \dist(U)
  \]
  Setzen wir nun $\sigma_U := h \d[z]$, so folgt
  \[
  S[\omega] = \iint_U \sigma_u \wedge \omega \qquad \omega \in
  \dist^{0,1}(U) \text{ und } \Supp(\omega) \Subset U
  \]
  Führen wir die gleiche Konstruktion auf einer zweiten Karte (z', U')
  aus und gilt außerdem noch, dass $\Supp(\omega) \Subset U')$, so
  erhalten wir
  \[
  \iint_U \sigma_U \wedge \omega = S[\omega] = \iint_{U'} \sigma_{U'}
  \wedge \omega \qquad \forall \omega \in \diff^{0,1}(X) \text{ und }
  \Supp(\omega) \Subset U \cap U'
  \]
  und damit $\sigma_U|_{U \cap U'} \equiv \sigma_{U'}|_{U \cap
    U'}$. Also können wir alle $\sigma_U$ zu einer globalen 1-Form
  $\sigma \in \Omega(Z)$ verkleben und es folgt
  \[
  S[\omega] = \iint_Z \sigma \wedge \omega \qquad \forall \omega \in
  \diff^{0,1}(X) \text{ mit } \Supp(\omega) \Subset U,
  \]
  wobei $U$ eine Koordinatenumgebung darstellt. \\
  Wählen wir nun ein beliebiges $\omega \in \diff^{0,1}(X)$ mit
  $\Supp(\omega) \Subset Z$, so können wir eine Partition der 1
  wählen, so dass $\omega = \sum_{j=1}^n \omega_J$ mit
  $\Supp(\omega_j) \Subset U_j$. Dabei sind die $U_j$
  Koordinatenumgebungen. \\
  Zu guter Letzt erhalten wir damit:
  \[
  S[\omega] = \sum_{j=1}^n S[\omega_j] = \sum_{j=1}^n \iint_Z \sigma
  \wedge \omega_j = \iint_Z \sigma \wedge \omega
  \]
\end{proof}

\begin{thm}
  \label{thm:runge-dicht}
  Sei $Y \Subset X$ eine offene Runge-Teilmenge und $X$ eine nicht
  kompakte Riemannsche Fläche. \\
  Dann liegt $\im( \hol(Y') \ra \hol(Y))$ für beliebige $Y' \Subset X$
  offen und $Y \subset Y'$ bzgl. kompakter Konvergenz dicht in $\hol(Y)$.
\end{thm}

\begin{proof}
  Sei $\beta: \diff(Y') \ra \diff(Y)$ die Einschränkungsabbildung. Wir
  % TODO: Referenz
  verwenden Kor B.10. Dazu müssen wir zeigen, dass falls $T \in
  \diff(Y)'$ mit $T_{\beta(\hol(Y'))} \equiv 0$ auch $T|_{\hol(Y)}
  \equiv 0$ gilt. \\
  % TODO: Referenz
  Nach 14.16 existiert zu jedem $\omega \in \diff^{0,1}(X)$ ein $f \in
  \diff(Y')$ mit $\d[''f] = \omega|_{Y'}$. \\
  Dies erlaubt uns die Definition von $S: \diff^{0,1}(X) \ra \C, \quad
  S[w] := T[f|_Y]$. Diese Definition ist unabhängig von $f$, denn für
  ein $g \in \diff(Y')$ mit $\d[''g] = \omega|_{Y'}$, dann ist $f-g
  \in \hol(Y')$ und deshalt $T[(f-g)|_Y] = 0$ nach der Voraussetzung,
  dass $T|_{\beta(\hol(Y'))} \equiv 0$. \\
  Als nächstes wollen wir zeigen, dass $S$ stetig ist. \\
  Dazu betrachten wir zunächst $V := \{ (\omega, f) \in \diff^{0,1}(X)
  \times \diff(Y'): \quad \d[''f] = \omega|_{Y'} \}$. Aus der
  % TODO: Woher Stetigkeit d''?
  Stetigkeit von $\d['']: \diff(Y') \ra \diff^{0,1}(Y')$ folgt, dass $V
  \subset \diff^{0,1}(X) \times \diff(Y')$ abgeschlossen und somit ein
  Fr´echet-Raum ist. \\
  % TODO: Referenz
  Nach 14.16 ist $\pi_1: V \ra \diff^{0,1}(X)$ surjektiv und damit
  offen. \\
  Weiterhin ist $\beta \circ \pi_2 : V \ra \diff(Y)$ stetig. Da nun
  auch noch das folgende Diagramm kommutiert:
  % TODO: Diagramm T*(beta*pi2) = S*pi1
  folgt die Stetigkeit von $S$. \\
  \\
  Das Lemma \ref{lemma:kompakter-träger-funktional} liefert
  \begin{itemize}
  \item ein kompaktes $K \subset Y$ mit $T[f] = 0$ für jedes $f \in
    \diff(Y)$ mit $\Supp(f9 \subset Y \setminus K$ und
  \item ein kompaktes $L \subset X$ mit $S[\omega] = 0$ für jedes
    $\omega \in \diff^{0,1}(X)$ mit $\Supp(\omega) \subset X\setminus L$
  \end{itemize}
   Sei $g \in \diff(X)$ mit $\Supp(g) \Subset X \setminus K$. Dann ist
   $S[\d[''g]] = T[g|_Y]  = 0$. Lemma \ref{lemma:funktional-explizit}
   gibt uns dann ein $\sigma \in \Omega(X\setminus K)$ mit
   \[
   S[\omega] = \iint_{X \setminus K}\sigma \wedge \omega \qquad
   \forall \omega \in \diff^{0,1}(X) \text{ mit } \Supp(\omega)
   \Subset X\setminus K
   \]
   Aufgrund des kompakten Trägers von $S$ muss $\sigma|_{X \setminus
     (K \cup L)} \equiv 0$ gelten. \\
   Nun ist jede Zusammenhangskomponente von $X \setminus \runge(K)$
   nicht relativ kompakt. \\
   Behauptung: Jede Zusammenhangskomponente von $X \setminus
   \runge(K)$ schneidet $X \setminus (K \cup L)$. \\
   Angenommen es gäbe eine Zusammenhangskomponente $U \subset X
   \setminus \runge(K)$ mit $U \cap X \setminus ( K \cup L) =
   \varnothing$. Dann gilt $U \subset K \cup L$, also ist $U$ relativ
   kompakt. Ein Widerspruch. \\
   Also schneiden alle Zusammenhangskomponenten $X \setminus (K \cup
   L)$ und aus de Identitätssatz folgt:
   \[
   \sigma|_{X \setminus \runge(K)} \equiv 0 \label{eq:runge}\tag{$\ast$}
   \]
   Das bedeutet, dass $S[\omega] = 0$ für jedes $\omega \in
   \diff^{0,1}(X)$ mit $\Supp(\omega) \Subset X \setminus
   \runge(K)$.\\
   \\
   Sei nun $f \in \hol(Y)$. Wir wollen zeigen, dass $T[f] = 0$. \\
   Zunächst gilt $\runge(K) \subset \runge(Y) = Y$. Durch Glättung
   % TODO: genauer
   können wir ein $g \in \diff(X)$ mit $f \equiv g$ in einer Umgebung
   von $\runge(K)$ und $\Supp(g) \Subset Y$ finden. \\
   Also liegt $\Supp(f - g|_Y) \subset Y \ K$, somit $T[f- g|_Y] = 0$
   und schlussendlich $T[f] = T[g|_Y] = S[\d[''g]]$. \\
   Nun ist aber $g$ in einer Umgebung von $\runge(K)$ holomorph, was
   nichts anderes bedeutet als $\Supp(\d[''g]) \Subset X \setminus
   \runge(K)$. \\
   Mit \eqref{eq:runge} folgt
   \[
   T[f] = T[g|_f] = S[\d[''g]] = 0
   \]
\end{proof}

\begin{thm}[Rungescher Approximationssatz]
  Sei $X$ eine nicht kompakte Riemannsche Fläche, $Y \subset X$ eine
  offene Runge-Teilmenge. \\
  Dann kann jede holomorphe Funktion auf $Y$ auf beliebigen Kompakta
  gleichmäßig durch holomorphe Funktionen auf ganz $X$ approximiert werden.
\end{thm}

\begin{proof}
  \begin{description}
  \item[Fall $Y \Subset X$:] Seien $f \in \hol(Y)$, $K \subset Y$
    kompakt und $\epsilon > 0$ gegeben. \\
    % TODO: Referenz
    Aus dem Beweis von (23.9) erhalten wir eine Ausschöpfung $Y_1
    \Subset Y_2 \Subset \dots$ von $X$ mit $Y_0 := Y \Subset Y_1$. \\
    Nach Satz \ref{thm:runge-dicht} existiert ein $f \in \hol(Y_1)$
    mit
    \[
    \|f_1 - f\|_K < 2^{-1} \epsilon
    \]
    Induktiv erhalten wir aus Satz \ref{thm:runge-dicht} eine Folge
    von Funktionen $f_n \in \hol(Y_n)$ mit
    \[
    \|f_n - f_{n-1}\|_{\bar Y_{n-2}} < 2^{-n} \epsilon \qquad \forall
    n \geq 2
    \]
    Nun gilt für beliebige $n \in \N$:
    \[
    \|f_{\nu + k} - f_{\nu} \|_{Y_n} \leq \epsilon 2^{-\nu} \sum_{\mu
      = 0}^k 2^{-\mu} \leq 2 \epsilon 2^{-\nu} \xrightarrow{\nu \ra
      \infty} 0 \qquad \forall $\nu q n$
    \]
    Also konvergiert $(f_\nu)_{\nu > n}$ gleichmäßig auf $Y_n$ für
    jedes $n \in \N$ und damit existiert ein $F \in \hol(X)$, so dass
    $f_\nu \rightrightarrow F$ auf jedem $Y_n$. \\
    Eingesetzt folgt
    \begin{align*}
      \|F - f\|_K & \leq \|F - f_n\|_{Y_n} + \sum_{k=1}^n \|f_k -
      f_{k-1}\|_{\bar Y_{k-2}} \\
      & < \|F _ f_n \|_{Y_n} + \epsilon \sum_{k=1}^n 2^{-k} \\
      & \xrightarrow{n \ra \infty} \epsilon
    \end{align*}
  \item[Fall $Y \subset X$ nicht relativ kompakt:] Auch hier wählen
    wir $f \in \hol(Y)$, $K \subset Y$ kompakt und $\epsilon > 0$. \\
    In diesem Fall können wir $L \subset Y$ kompakt finden mit $K \subset
    \mathring L$. Dies ist möglich in dem wir $K$ durch endlich viele relativ
    Kompakte Koordinatenumgebungen überdecken und L als deren
    Abschluss definieren. \\
    % TODO: Referenz
    Dann folgt aus 23.7 die Existenz einer offenen Runge-Teilmenge
    $\tilde Y \subset X$ mit $K \subset \tilde Y \subset L$. \\
    Damit ist $\tilde Y \Subset X$ und $f$ kann als Funktion von
    $\hol(\tilde Y)$ aufgefasst werden und schlussendlich können wir
    den obigen Fall anwenden und erhalten ein $F \in \hol(X)$ mit
    $\|f|_{\tilde Y} - F\|_K = \|f - F\|_K < \epsilon$.
  \end{description}
\end{proof}

%%% Local Variables: 
%%% mode: latex
%%% TeX-master: "../Bachelor"
%%% End: 

\message{ !name(../Bachelor.tex) !offset(-289) }

\end{document}

%%% Local Variables: 
%%% mode: latex
%%% TeX-master: t
%%% End: 
