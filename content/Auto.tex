
\section{Automorphismengruppen}
\label{sec:auto}

\begin{defin}
  Sei $X$ eine Riemannsche Fläche. Unter der
  \init{Automorphismengruppe} von $X$ verstehen wir
  \[
  \Aut(X) := \{ f: X \ra X \mid f,f^{-1} \text{ holomorph} \}.
  \]
  Diese ist mit der Verkettung von Abbildungen tatsächlich eine Gruppe.
\end{defin}

\begin{lemma}
  \label{lemma:kreis-halbebene}
  Die Abbildung
  \[
  \phi: \h \ra B, \quad \phi(z) := \frac{z-i}{z+i}
  \]
  ist biholomorph.
\end{lemma}

\begin{proof}
  Zunächst ist $\phi$ wohldefiniert, denn für $z \in \h$ erhalten wir,
  dass $\Im(z) > 0$ gilt und daher folgt
  \begin{align*}
    |\phi(z)|^2 & = \frac{z-i}{z+i} \cdot \frac{\bar z + i}{\bar z -
      i} \\
    & = \frac{|z|^2 - \bar z i + i z + 1}{|z|^2 +i \bar z - i z + 1}
    \\
    & = \frac{|z|^2 - 2 \Re(iz) + 1}{|z|^2 - 2 \Re(iz) + 1} \\
    & = \frac{|z|^2 +1 - 2 \Im(z)}{|z|^2 + 1 + 2\Im(z)} \\
    & < 1.
  \end{align*}
  Betrachten wir nun
  \[
  \psi: B \ra \h, \quad \psi(z) := \frac{ -iz - i}{z-1},
  \]
  so ist diese Abbildung auch wohldefiniert, denn für $z \in \C$ mit
  $|z| < 1$ erhalten wir
  \begin{align*}
    \Im(\psi(z)) & = \frac{1}{2i} ( \psi(z) - \bar \psi(z)) \\
    & = \frac{1}{2i} \left ( \frac{-iz - i}{z -1 } - \frac{ i \bar z +
        i}{\bar z - 1} \right ) \\
    & = \frac{1}{2i} \cdot \frac{2i ( 1- |z|^2 )}{|z -1|^2} \\
    & = \frac{1- |z|^2}{|z-1|^2} \\
    & > 0.
  \end{align*}
  Weiterhin zeigt eine einfache Rechnung, dass $\psi \circ \phi =
  \id_\h$ und $\phi \circ \psi = \id_B$ gelten.
\end{proof}

\begin{rem}
  Dass $\h$ und $B$ konform äquivalent sind, ist bereits eine
  Konsequenz aus dem Riemannschen Abbildungssatz \ref{thm:rmt}, allerdings benötigen
  wir für den nächsten Satz die Übergangsabbildung explizit.
\end{rem}

\begin{thm}
  \label{thm:aut}
  Es gelten
  \begin{enumerate}
  \item $\Aut(\C) = \{ f: \C \ra \C, \ z \mapsto az + b \mid a \in
    \C^\times, \ b \in \C \}$
  \item $\Aut(\P^1) = \{ f: \P^1 \ra \P^1,\ f(z) = M\langle z \rangle \mid
    M \in Sl(2, \C) \}$
  \item $\Aut(B) = \{ f: B \ra B,\ z \mapsto \frac{\alpha z +
      \beta}{\bar \beta z + \bar \alpha} \mid \alpha, \beta \in \C, \
    |\alpha |- |\beta| = 1\}$
  \item $\Aut(\h) = \{f: \h \ra \h,\ f(z) = M\langle z \rangle 
    \mid M \in Sl(2, \R) \}$
  \end{enumerate}
\end{thm}

\begin{proof}
  Wir zeigen jeweils nur $\subseteq$, denn die andere Inklusion ergibt
  sich sofort, da alle Umkehrfunktionen explizit ausgedrückt werden
  können.

  Sei also $f \in \Aut(\C)$, dann ist $f$ eine ganze Funktion, also
  existieren $a_k \in \C$, so dass
  \[
  f(z) = \sum_{k=0}^\infty a_k z^k
  \]
  gilt. Wir behaupten nun, dass $a_k = 0$ für fast alle $k \in \N_0$
  gelten muss. Angenommen dies wäre nicht der Fall, dann hätte die
  Funktion $g: \C^\ast \ra \C$ mit $g(z) := f(\frac{1}{z})$ eine
  wesentliche Singularität bei $0$. Dann wäre aber der Satz von
  Casorati-Weierstraß (vgl. \cite[Satz 6.11]{Kas}) anwendbar und wir
  erhielten zu jedem $n \in \N$
  ein $z_n \in \dot B_{\frac{1}{n}}(0)$, so dass $|g(z_n) - f(0)| <
  \frac{1}{n}$ gilt. Setzen wir $w_n := \frac{1}{z_n}$, so konvergierte
  $(w_n)_{n \in \N}$ gegen $\infty$ und $f(w_n) \ra f(0)$ für $n \ra
  \infty$. Setzen wir nun weiterhin $\tilde z_n := f(w_n)$ und $\tilde
  z = f(0)$, so erhielten wir also $\lim_{n \ra \infty} \tilde z_n =
  \tilde z$, allerdings gälte $|f^{-1}(\tilde z_n)| = |w_n| \ra \infty
  \neq |f^{-1}(\tilde z)| = 0$. Dies ist ein Widerspruch zur
  Stetigkeit von $f^{-1}$. Also ist $f$ ein Polynom. Dies hat aber
  direkt zur Folge, dass der Grad von $f$ eins sein muss, da wir
  ansonsten nach dem Haupsatz der Algebra mehrere Nullstellen
  hätten. Damit haben wir bereits die gewünschte Form gefunden.

  Als nächstes wenden wir uns $\Aut(\P^1)$ zu. Sei dazu $f \in
  \Aut(\P^1)$. Wir setzen $a:= f(\infty)$ und $b :=
  f^{-1}(\infty)$. Zunächst betrachten wir den fall $a \neq
  \infty$. Dann gilt auch $b \neq \infty$ und wir können
  \[
  \psi: \P^1 \ra \P^1, \quad \psi(z) = \frac{bz + c}{z - a}
  \]
  definieren, wobei $c \in \C$ mit $c \neq a$. Weiterhin gilt
  \[
  \psi^{-1}: \P^1 \ra \P^1, \quad \psi^{-1}(z) = \frac{az +c}{z -b}
  \]
  und damit ist $\psi \in \Aut(\P^1)$. Setzen wir nun $g := f \circ
  \psi$, so erhalten wir
  \begin{align*}
    g(\infty) = f(\psi(\infty)) = f(b) = \infty.
  \end{align*}
  Also ist $g|_\C \in \Aut(\C)$ und wie wir oben gesehen haben,
  existieren $d,e \in \C$ mit $d \neq 0$, so dass $g|_\C(z) = dz +
  e$. Nun gilt $|dz + e| \ra \infty$ für $z \ra \infty$. Also können
  wir die Notation $g(z) = dz+e$ auf ganz $\P^1$ fortsetzen. Insgesamt
  erhalten wir
  \begin{align*}
    f(z) & = g(\psi^{-1}(z)) \\
    & = d\psi^{-1}(z) + e \\
    & = d \cdot \frac{az + c}{z - b} + e \\
    & = \frac{(da + e) z + (dc - eb)}{z - b} \\
    & =
    \begin{pmatrix}
      (da + e) & (dc - eb) \\
      1 & -b
    \end{pmatrix}\langle z \rangle \\
    & =: M\langle z \rangle.
  \end{align*}
  Weiterhin ist $\det M = d( a-c) \neq 0$. Wählen wir als $\gamma$
  eine Wurzel von $\det M$ und definieren $\tilde M := \gamma^{-1} M$,
  so gilt $f(z) = M\langle z \rangle = \tilde M \langle z \rangle$ und
  $\det \tilde M = \gamma^{-2} \det M = 1$. Also liegt $\tilde M \in
  PSl(2, \C)$.

  Für den Fall, dass $a = \infty = b$, so brauchen wir die
  Hilfsfunktion $\psi$ nicht und wir können direkt unser Wissen über
  $\Aut(\C)$ anwenden. Eine analoge Rechnung liefert dann das
  Resultat.

  Sei nun $f \in \Aut(B)$ und $w := f^{-1}(0)$. Dann definieren wir
  \[
  f_w : B \ra B, \quad f_w(z) = \frac{z -w}{\bar w z - 1}.
  \]
  Es zeigt sich, dass $f_w$ selbstinvers ist und damit in $\Aut(B)$
  liegt. Setzen wir $g := f\circ f_w$, so erhalten wir
  \begin{align*}
    g(0) = f( f_w(0)) = f(w) = 0
  \end{align*}
  Damit ist das Lemma von Schwarz (vgl. \cite[Satz 5.10]{Kas})
  anwendbar und wir erhalten sowohl für $g$, als auch für $g^{-1}$
  \begin{align*}
    |g(z)| & \leq |z| \\
    |g^{-1}(z)| & \leq |z|
  \end{align*}
  Es ergibt sich, dass
  \begin{align*}
    |w| = |g^{-1}(g(w))| \leq |g(w)| \leq |w|
  \end{align*}
  gilt. Also ist $|g(w)| = |w|$ und als weitere Folgerung aus dem Lemma
  von Schwarz erhalten wir die Existenz eines $c \in \C$ mit $|c| =
  1$, so dass
  \[
  g(z) = cz
  \]
  gilt. Es folgt, dass
  \begin{align*}
    f(z) = c f_w(z) = \frac{i dz - i dw}{ - \overline{idw} z +
      \overline{i d}}
  \end{align*}
  gilt, wobei $d^2 = c$. Setzen wir $\tilde \alpha = i d$ und $\tilde
  \beta = - i d w$, so folgt $|\tilde \alpha| = 1$ und $|\tilde \beta|
  = |w| < 1$. Wir definieren
  \[
  \lambda = |\tilde \alpha| - |\tilde \beta|
  \]
  und setzen $\alpha := \lambda^{-1} \tilde \alpha$ und $\beta :=
  \lambda^{-1} \tilde \beta$. Es gilt $|\alpha| - |\beta| = 1$ und
  \[
  f(z) = \frac{\alpha z - \beta}{\bar \beta z  + \bar \alpha},
  \]
  was die Behauptung zeigt.

  Als letztes interessieren wir uns nun für $\Aut(\h)$. Diesen Fall
  können wir aber auf $\Aut(B)$ zurückführen, denn Lemma
  \ref{lemma:kreis-halbebene} gibt uns eine Bijektion
  \[
  \Aut(\h) \ra \Aut(B), \quad f \mapsto \phi \circ f \circ \phi^{-1}.
  \]
  Sei also $f \in \Aut(\h)$, dann existieren $\alpha, \beta \in \C$
  mit $|\alpha| - |\beta| = 1$, so dass
  \[
  \phi( f( \phi^{-1}(z))) = \frac{\alpha z + \beta}{\bar \beta z +
    \bar \alpha}
  \]
  Es gilt also
  \begin{align*}
    f(z) & = \phi^{-1} \left ( \frac{\alpha \phi(z) + \beta}{\bar
        \beta \phi(z) + \bar \alpha} \right ) \\
    & = \frac{-i \alpha \phi(z) - i \beta - i \bar \beta \phi(z) - i
      \bar \alpha}{\alpha \phi(z) + \beta - \bar \beta \phi(z) - \bar
      \alpha} \\
    & = \frac{-i (\alpha + \bar \beta)(z-i) -i (\beta + \bar
      \alpha)(z+i)}{(\alpha - \bar \beta)(z -i) + (\beta - \bar
      \alpha)(z+i)} \\
    & = \frac{-i ( \alpha + \bar \beta + \beta + \bar \alpha) z + (
      \bar \alpha - \bar \beta + \beta + \bar \alpha)}{(\alpha - \bar
      \beta + \beta - \bar \alpha)z + i (- \alpha + \bar \beta + \beta
      - \bar \alpha)} \\
    & = \frac{ - ( \Re(\alpha) + \Re(\beta)) z +  (\Im(\beta) -
      \Im(\alpha))}{(\Im(\alpha) + \Im(\beta))z + ( \Re(\beta) -
      \Re(\alpha))} \\
    & =: M\langle z \rangle.
  \end{align*}
  Betrachten wir nun $\det M$, so erhalten wir $\det M = |\alpha| -
  |\beta| = 1$. Außerdem hat $M$ nur reelle Einträge, also liegt $M
  \in Sl(2, \R)$.
\end{proof}

Als nächstes wenden wir uns Gittern zu. Diese benötigen wir, um die
diskreten Untergruppen von $\C$ zu beschreiben. Später wird sich dann
auch zeigen, dass wir diese diskreten Untergruppen benötigen, um die
Riemannschen Flächen zu beschreiben, deren Universelle Überlagerung
$\C$ ist.

\begin{defin}
  Sei $B$ ein $n-$dimensionaler, reeller Vektorraum. Eine additive
  Untergruppe $\Gamma \subset V$ heißt \init{Gitter}, falls $n$ linear
  unabhängige $\gamma_1, \dots, \gamma_n \in V$ existieren, so dass
  \[
  \Gamma = \Z \gamma_1 + \dots + \Z \gamma_2
  \]
\end{defin}

\begin{lemma}
  \label{lemma:gitter}
  Sei $\Gamma \subset v$ eine additive Untergruppe. $\Gamma$ ist ein
  Gitter genau dann, wenn
  \begin{enumerate}
  \item $\Gamma$ diskret ist und
  \item $\Gamma$ nicht in einem echten Untervektorraum von $V$
    enthalten ist.
  \end{enumerate}
\end{lemma}

\begin{proof}
  Falls $\Gamma$ ein Gitter ist, die Aussage aus der Definition
  klar. Erfülle also $\Gamma$ die beiden Bedingungen. Wir gehen nun
  per Induktion vor. Für $n = 0$ ist di eAussage klar. Gelte also die
  Aussage für ein $n \in \N_0$. Da $\Gamma \subset V$ mit $\dim V =
  n+1$ in keinem echten
  Untervektorraum von $V$ enthalten ist, existieren linear unabhängige $x_1, \dots,
  x_{n+1} \in \Gamma$. Sei $V_1 := \langle x_1, \dots, x_n \rangle$
  und $\Gamma_1 := \Gamma \cap V_1$. Nun ist $\Gamma_1$ wieder diskret
  und in keinem echten Untervektorraum von $V_1$ enthalten. Weiterhin
  ist $\dim V_1 = n$, also können wir die Induktionsvoraussetzung
  verwenden und erhalten linear unabhängige $\gamma_1, \dots, \gamma_n
  \in \Gamma_1 \subset \Gamma$, so dass $\Gamma_1 = \Z \gamma_1 +
  \dots + \Z \gamma_n$ gilt. Nun bildendie $\gamma_i$ bereits eine
  Basis von $V_1$ und fügen wir $x_{n+}$ erhalten wir sogar eine Basis
  von $V$, das bedeutet aber, dass wir zu jedem $x \in \Gamma \subset
  V$, eindeutig bestimmte $c_i(x), c(x) \in \R$ finden, so dass
  \[
  x = c_1(x) \gamma_1 + \dots + c_n(x) \gamma_n + c(x) x_{n+1}
  \]
  Wir betrachten nun
  \[
  P := \{ \lambda_1 \gamma_1 + \dots + \lambda_n \gamma_n + \lambda
  x_{n+1} \mid \lambda_i, \lambda \in [0,1] \}.
  \]
  Dieses $P$ ist kompakt und da $\Gamma$ diskret ist, folgt dass $P
  \cap \Gamma$ endlich ist. Weiterhin ist $( \Gamma \cap P) \setminus
  V_1$ nicht leer, da $x_{n+1}$ enthalten ist. Damit existiert ein
  $\gamma_{nü1} \in ( \Gamma \cap P) \setminus V_1$ mit
  \[
  c(\gamma_n) = \min \{ c(x) \mid x \in ( \Gamma \cap P ) \setminus
  V_1 \} \in ]0, 1].
  \]
  Wir behaupten nun, dass $\Gamma = \Gamma_1 + \Z \gamma_{n+1}$. Sei
  dazu $x \in \Gamma$. Dann existieren $n_j \in \Z$, so dass
  \[
  x' := x - \sum_{j=1}^{n+1} n_j \gamma_j = \sum_{j=1}^n \lambda_j
  \gamma_j + \lambda x_{n+1}
  \]
  mit $0 \leq \lambda_j < !$ und $0 \leq \lambda < c(\gamma_{n+1})$
  gilt. Da $x' \in \Gamma \cap P$ liegt und $c(\gamma_{n+1})$ minimal
  von 0 verschieden gewählt wurde, muss $\lambda = 0$ sein. Also liegt
  $x' \in \Gamma \cap V_1 = \Gamma_1$. Also sind die $\lambda_i$
  ganzzahlig und die einzige Möglichkeit, bleibt, dass $\lambda_i = 0$
  gilt. Dann ist aber $x'$ bereits 0 und wir erhalten
  \[
  x = \sum_{j=1}^n n_j \gamma_j \in \Z \gamma_1 + \dots + \Z \gamma_n.
  \]
\end{proof}

%%% Local Variables: 
%%% mode: latex
%%% TeX-master: "../Bachelor"
%%% End: 
