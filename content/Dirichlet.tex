
\section{Das Dirichlet Randwertproblem}
\label{sec:Dirichlet}

Es besteht ein enger Zusammenhang zwischen holomorphen und
harmonischen Abbildungen, genauer gesagt ist auf
zusammenhängenden Riemannschen Flächen jede harmonische Funktion
Realteil einer holomorphen Funktion. Dieses Resultat ist sehr
interessant, denn es ermöglicht uns holomorphe Abbildungen zu
"`konstruieren"'. Wir sind im weiteren Verlauf vor allem an
nicht-konstanten holomorphen Abbildungen interessiert und unsere
Strategie, diese Abbildungen zu finden, wird darin bestehen das
Dirichlet Randwertproblem (kurz: RWP) (vgl. Definition
\ref{defin:rwp}) zu lösen. Die Lösung eines solchen RWP ist immer
harmonisch und durch geeignete Randwerte können wir dafür sorgen, dass
sie nicht-konstant ist. Durch diesen einfachen Trick haben wir dann
eine nicht-konstante holomorphe Abbildung gefunden.

Natürlich ist per se noch nicht klar, unter welchen Umständen
überhaupt Lösungen für ein RWP existieren und ob diese eindeutig
bestimmt sind. Der Großteil des Kapitels wird sich zunächst mit etwas
Terminologie und dann der Eindeutigkeit von Lösungen des RWP
beschäftigen. Wir werden Perrons Methode (Lemma \ref{lemma:perron})
verwenden, um Lösungskandidaten zu konstruieren und schlussendlich ein
hinreichendes Kriterium angeben, um Gebiete auf denen Lösungen des RWP
existieren zu identifizieren.

Im Weiteren bezeichne $X$ immer ein Riemannsche Fläche.

\begin{defin}
  Für jedes $\omega \in \diff^{(1)}(X)$ existiert genau ein $\omega_1
  \in \diff^{1,0}(X)$ und genau ein $\omega_2 \in \diff^{0,1}(X)$, so
  dass $\omega = \omega_1 + \omega_2$ gilt. Dies ermöglicht die
  Definition der Abbildung
  \[
  \ast : \diff^{(1)}(X) \ra \diff^{(1)}(X), \quad \omega_1 + \omega_2
  \mapsto i(\bar \omega_1 - \bar \omega_2).
  \]
\end{defin}

\begin{prop}
  Die Abbildung $\ast$ ist ein $\R$-linearer Isomorphismus. Weiterhin
  bildet sie  $\diff^{1,0}(X)$ auf $\diff^{0,1}(X)$ ab und umgekehrt.
\end{prop}

\begin{proof}
  Die $\R$-Linearität von $\ast$ ist klar. Die Bijektivität folgt aus
  der Tatsache, dass $\ast^{-1} = - \ast$ gilt. Sei nun weiterhin
  $\omega \in \diff^{1,0}(X)$. Auf einer Karte $(z,U)$ nimmt $\omega$
  dann die Form $\omega = f \d[z]$ für ein $f \in \diff(U)$ an. Dann gilt
  aber $\ast \omega = i \bar \omega = i \bar f \d[\bar z] \in
  \diff^{0,1}(X)$ auf der Karte. Also wird tatsächlich
  $\diff^{1,0}(X)$ auf $\diff^{0,1}(X)$ abgebildet. Die Bijektivität
  folgt auch einfach wieder aus der Tatsache, dass $\ast^{-1} = -
  \ast$ gilt.
\end{proof}

\begin{defin}
  Eine 1-Form $\omega \in \diff^1(X)$ heißt \init{harmonisch}, falls
  \[
  \d[\omega] = 0 = \d[\ast \omega]
  \]
  gilt.
\end{defin}

\begin{thm}
  \label{thm:harm-form}
  Jede reelle 1-Form $\sigma \in \diff^1(X)$ ist Realteil genau einer
  holomorphen 1-Form $\omega \in \Omega(X)$.
\end{thm}

\begin{proof}
  Wie können $\sigma = \omega_1 + \omega_2$ mit $\omega_1 \in
  \diff^{1,0}(X)$ und $\omega_2 \in \diff^{0,1}(X)$ schreiben. Nun
  folgt aber aus $\d[\omega] = 0$, dass
  \begin{align}
    \d[''\omega_1] + \d['\omega_2] = 0 \label{eq:harm1}
  \end{align}
  und aus $\d[\ast \omega] = 0$, dass
  \begin{align}
  i(\d['\bar \omega_1] - \d[''\bar \omega_2]) = 0 \label{eq:harm2}
  \end{align}
  gelten. Durch eine einfache
  Rechnung auf Karten lässt sich \eqref{eq:harm2} zu
  \[
  0 = \overline{\d[''\omega_1]} - \overline{\d['\omega_2]}
  \]
  umformen. Dies liefert uns zunächst zusammen mit
  Gleichung\eqref{eq:harm1}, dass $\d[''\omega_1] = 0$ und $\d['\omega_2] = 0$
  gelten. Daraus folgt aber direkt, dass $\omega_1$ in $\Omega(X)$
  und unter Ausnutzung von $\overline{\d['\omega_2]} = \d[''
  \bar \omega_2]$ erhalten wir, dass $\bar \omega_2$ in $\Omega(X)$
  liegt. Nun wissen wir aber, dass $\sigma$ reell ist und wir erhalten
  die Gleichung
  \[
  \omega_1 + \omega_2 = \sigma = \bar \sigma = \bar \omega_1 + \bar
  \omega_2.
  \]
  Da die Darstellung von $\sigma$ eindeutig ist, muss $\omega_1 =
  \bar \omega_2$ gelten. Setzen wir $\omega = 2\omega_1 \in
  \Omega(X)$, so gilt
  \begin{align*}
    \Re(\omega) & = \Re(2\omega_1) = \frac{1}{2}( 2 \omega_1 + 2 \bar
    \omega_1) = \omega_1 + \omega_2 = \sigma.
  \end{align*}
  Um nun die Eindeutigkeit von $\omega$ zu beweisen, zeigen wir, dass
  $\sigma = 0$ bereits $\omega = 0$ impliziert. Dies genügt, da
  $\Re(x) = \Re(y)$ genau dann gilt, wenn $\Re(x-y) = 0$ ist. Dies
  folgt aus der $\R$-Linearität von $\Re$\footnote{Wobei $\Re$
  natürlich nicht $\C$ linear ist.}. Sei also $\omega \in \Omega(X)$
  mit $\Re(\omega) = 0$. Nun existiert lokal immer eine
  Stammfunktion für $\omega$, d.h. lokal existiert ein holomorphes
  $f$, so dass $\omega = \d[f]$ gilt. Dann folgt, dass
  $\d[\Re(f)] = 0$ sein muss, also ist $\Re(f)$ konstant. Das wiederum
  hat zur Folge, dass die holomorphe Funktion $e^f$ ihr
  Betragsmaximum annimmt, denn es gilt $|e^f| = e^{\Re(f)}$. Dann ist
  aber $e^f$ nach dem Maximumprinzip für holomorphe Funktionen
  (vgl. \cite[Kor. 2.6]{For})
  konstant. Also ist auch $f$ konstant und es gilt $\omega = \d[f] =
  0$. Da dies lokal um jeden Punkt gilt, ist somit $\omega = 0$ und
  damit ist unser $\omega$ eindeutig bestimmt.
\end{proof}

\begin{defin}
  Sei $X$ eine Riemannsche Fläche und $Y \subset X$ offen. Dann heißt
  $u \in \diff(Y)$ \init{harmonisch}, falls $\d['\d[''u]] = 0$ gilt.
  In lokalen Koordinaten $(U,z)$ mit $z = x + iy$ bedeutet das
  \begin{align*}
    \Delta u = \left ( \frac{\partial^2}{\partial x^2} +
      \frac{\partial^2}{\partial y^2} \right ) u = 4
    \frac{\partial^2}{\partial z \partial \bar z} u = 0.
  \end{align*}
\end{defin}

\begin{prop}
  \label{prop:harm-realteil-hol}
  Sei $X$ eine Riemannsche Fläche, $G \subset X$ ein Gebiet
  und $u: G \ra \R$ harmonisch. Dann existiert ein $f \in \hol(G)$, so dass
  \[
  u = \Re(f)
  \]
  gilt.
\end{prop}

\begin{proof}
  Es gilt $\d[\d[u]] = 0$ und unter Ausnutzung von $0 = \d['\d['' u]]
  = - \d[''\d['u]]$ lässt sich auf Karten $\d[\ast\d[u]] = 0$
  verifizieren. Damit ist $\d[u]$ eine harmonische 1-Form und nach
  Satz \ref{thm:harm-form} existiert ein $\omega \in \Omega(G)$, so dass $\d[u] =
  \Re(\omega)$ ist. Nun ist aber $G$ einfach zusammenhängend, d.h. es gibt
  eine Funktion $g \in \hol(G)$, so dass $\omega = \d[g]$. Damit ist
  $\d[u] = \Re(\d[g])$ und es folgt $u = \Re(g) +$const.
\end{proof}

\begin{rem}
  Die Umkehrung gilt immer, d.h. der Realteil einer holomorphen
  Funktion ist immer harmonisch.
\end{rem}

\begin{prop}[Maximumprinzip für harmonische Funktionen]
  \label{prop:max-prinzip-harm}
  Sei $Y \subset X$ ein Gebiet. Sei weiterhin $u: Y \ra \R$ harmonisch
  und $x_0 \in Y$ mit
  \[
  u(x_0) = \sup_{y \in Y} u(y).
  \]
  Dann ist $u$ konstant.
\end{prop}

\begin{proof}
  Wir verwenden für den Beweis ein
  "`Offen-Abgeschlossen"'-Argument. Sei dazu
  \[
  M:= \{ y \in Y\mid u(y) = u(x_0)\} \neq \varnothing.
  \]
  Wir behaupten zunächst, dass $M$ offen ist.
  Sei $y \in Y$. Dann existiert eine offene,
  einfachzusammenhängende Umgebung $U \subset Y$ von $y$. 
  Nach Proposition \ref{prop:harm-realteil-hol} folgt dann die
  Existenz eines $f \in \hol(U)$ mit $u|_U = \Re(f)$.
  Weiterhin ist die reelle Exponentialfunktion streng monoton
  steigend und somit nimmt die Funktion $e^u$ bei $y$ ein Maximum
  an. Allerdings gilt auch $|e^f| = e^u$, was nichts anderes
  bedeutet, als dass die holomorphe Funktion $e^f$ ein
  Betragsmaximum bei $y$ annimmt. 
  Aus dem Maximumsprinzip für holomorphe Funktionen (vgl. \cite[Kor. 2.6]{For}) folgt dann die
  Konstanz von $e^f$ und damit
  \[
  u|_U = \text{const.}
  \]
  Also gilt $y \in U \subset M$.
  
  Als nächstes behaupten wir, dass $M$ zusätzlich auch abgeschlossen
  ist. Dies ist aber eine direkte Konsequenz aus der Stetigkeit von $u$.

  Damit ist $M \subset Y$ sowohl offen als auch abgeschlossen und da
  $Y$ zusammenhängend ist, muss $M$ entweder leer oder ganz $Y$
  sein. Nun liegt $x_0$ in $M$ und somit folgt $M = Y$, also ist $u$ konstant.
\end{proof}

\begin{defin}
  \label{defin:rwp}
  Seien $X$ eine Riemannsche Fläche, $Y\subset X$ offen und $f: \partial
  Y \ra \R$ stetig. Man nennt $u \in C(\bar Y, \R)$ \init{Lösung des Dirichlet
    Randwertproblemes} (Dirichlet-RWP), falls
  \begin{enumerate}
  \item $u|_Y$ harmonisch ist und
  \item $u|_{\partial Y} = f$ gilt.
  \end{enumerate}
\end{defin}

\begin{prop}
  \label{prop:dirichlet-eindeutig}
  Sei $Y \Subset X$ offen und $\partial Y \neq \varnothing$. Dann ist
  die Lösung eines Dirichlet-RWP eindeutig, falls sie existiert.
\end{prop}

\begin{proof}
  Seien $u_1$ und $u_2$ zwei Lösungen des Dirichlet-Problems. Dann erfüllen
  die Funktionen
  \[
  (u_1 - u_2)|_{\partial Y} = (u_2 - u_1)|_{\partial Y} = 0
  \]
  und sind harmonisch. Aus dem Maximumprinzip folgt sofort, die
  Nicht-Positivität von sowohl $u_1 - u_2$ als auch von $u_2 - u_1$
  auf allen Zusammenhangskomponenten von $Y$, also folgt
  \[
  0 \leq u_1 - u_2 \leq 0
  \]
  und damit
  \[
  u_1 = u_2.
  \]
\end{proof}

\begin{thm}
  \label{thm:poisson}
  Sei $f: \partial B_R(0) \ra \R$ stetig, $R > 0$ und
\begin{align*}
    u(z) :=
    \begin{cases}
      \frac{1}{2\pi} \int_0^{2\pi} \frac{R^2 - |z|^2}{|Re^{i\phi}
        - z|^2} f(R e^{i\phi}) \d[\phi] & \text{für } |z| < R\\
      f(z) & \text{für } |z| = R
    \end{cases}.
  \end{align*}
  Dann ist $u$ stetig auf $\overline{B_R(0)}$ und harmonisch auf $B_R(0)$;
  löst also das Dirichlet-Problem auf $\overline{B_R(0)}$.
\end{thm}

\begin{proof}
  Für $z \neq \rho$ definieren wir
  \[
  P(z, \rho) := \frac{|\rho|^2 - |z|^2}{|\rho - z|^2}, \qquad
  F(z,\rho) := \frac{\rho +z }{\rho - z}.
  \]
  Dann gilt $P(z, \rho) = \Re(F(z, \rho))$ und für $u$ ergibt sich
  \begin{align*}
    u(z) & = \frac{1}{2\pi} \int_0^{2\pi} P(z, Re^{i\phi})
    f(Re^{i\phi}) \d[\phi] \\
    & = \Re \left ( \frac{1}{2\pi} \int_0^{2\pi} F(z, Re^{i\phi})
      f(Re^{i\phi}) \d[\phi] \right ).
  \end{align*}
  Nun ist $F$ als Funktion von $z$ holomorph (da $|z| < R$) 
  und damit erhalten wir aus der Leibnizregel, dass $u$ der Realteil
  einer holomorphen Funktion ist, also harmonisch.
  
  Als nächstes müssen wir die Stetigkeit von $u$ auf dem Rand
  zeigen. Zunächst zeigen wir jedoch die Gleichung
  \[
  \frac{1}{2\pi} \int_0^{2\pi} P(z, Re^{i\phi}) \d[\phi] = 1.
  \]
  Dazu verwenden wir den Residuensatz, denn es gilt
  \begin{align}
  \frac{1}{2\pi} \int_0^{2\pi} P(z, Re^{i\phi}) \d[\phi]
  = \Re \left ( \frac{1}{2\pi i} \int_{|\rho| = R} \underbrace{
      \frac{\rho + z}{(\rho - z)\rho}}_{=:h_z(\rho)} \d[\rho] \right
  ). \label{eq:residuensatz}
  \end{align}
  Weiterhin ist $\res_{\rho = z} h_z(\rho) = 2$ und $\res_{\rho=0}
  h_z(\rho) = -1$. Da jede Singularität nur einmal umlaufen wird
  erhalten wir
  \[
  \eqref{eq:residuensatz} = \Re(\res_{z = \rho}h_z(\rho) + \res_{z =
    0} h_z(\rho)) = 1.
  \]
  Also gilt für $\rho_0 \in \partial B_R(0)$ und $z \in B_R(0)$
  \[
  u(z) - f(\rho_0) = \frac{1}{2\pi} \int_0^{2\pi} P(z, \rho) (f (\rho)
  - f(\rho_0) )\d[\phi],
  \]
  wobei $\rho = Re^{i\phi}$ gesetzt wurde. Sei nun $\epsilon > 0$ beliebig. Da $f$
  stetig ist, existiert ein $\delta_0 > 0$, so dass
  \[
  |f(\rho) - f(\rho_0)| \leq \frac{\epsilon}{2} \qquad \forall |\rho -
  \rho_0| \leq \delta_0, \quad \rho, \rho_0 \in \partial B_R(0)
  \]
  gilt. Weiterhin existiert ein $M > 0$, so dass $|f(\rho)| \leq M$ für
  jedes $\rho \in \partial B_R(0)$ ist. Setzen wir $\alpha \subset [0, 2\pi]$
  so, dass
  \[
  |R e^{i\phi} - \rho_0| \leq \delta_0 \qquad \forall \phi \in \alpha
  \]
  und $\beta := [0, 2\pi]\setminus \alpha$, so erhalten wir
  \begin{align*}
    |u(z) - f(\rho_0)| & \leq \frac{1}{2\pi} \int_\alpha
    P(z, \rho) \frac{\epsilon}{2} \d[\phi] +
    \frac{1}{2\pi} \int_\beta P(z, \rho) 2 M \d[\phi] \\
    & \leq \frac{\epsilon}{2} + \frac{M}{\pi} \int_\beta P(z,
    Re^{i\phi}) \d[\phi].
  \end{align*}
  Sei nun $|z - \rho_0| =: \delta \leq \frac{\delta_0}{2}$. Dann gilt
  für $\phi \in \beta$
  \[
  |Re^{i\phi} - z| \geq |Re^{i\phi} - \rho_o| - |z - \rho_o| \geq
  \delta_0 - \frac{\delta_0}{2} = \frac{\delta_0}{2}
  \]
  und damit für $P$
  \[
  P(z, Re^{i\phi}) = \frac{(R+ |z|)(R - |z|)}{|Re^{i\phi} - z|^2} \leq
  \frac{4 \cdot 2 R \delta}{\delta_0^2}.
  \]
  Insgesamt erhalten wir
  \[
  |u(z) - f(\rho_0)| < \frac{\epsilon}{2} + \frac{8 R
    \delta}{\delta_0^2} \cdot \frac{M}{\pi} \cdot 2\pi
  \]
  und auch der rechte Summand wird kleiner als
  $\frac{\epsilon}{2}$, wenn $\delta =: |z - \rho_0|$ klein genug
  gewählt wurde.
\end{proof}

\begin{cor}
  \label{cor:harm-darstellung}
  Sei $u: B_R(0) \ra \R$ harmonisch, $ R > 0$. Dann gilt
  \[
  u(z) = \frac{1}{2\pi} \int_0^{2\pi} \frac{r^2 - |z|^2}{|re^{i\phi} -
    z|^2} u(re^{i\phi}) \d[\phi]
  \]
  für jedes $|z|<r<R$.
\end{cor}

\begin{proof}
  Die Aussage folgt direkt aus Satz \ref{thm:poisson} und Proposition \ref{prop:dirichlet-eindeutig}.
\end{proof}

\begin{cor}
  \label{cor:harm-konvergenz}
  Sei $(u_n)_{n\in \N} \subset C(B_r(0), \R)$ eine Folge harmonischer
  Funktionen, die kompakt gegen
  \[
  u: B_R(0) \ra \R
  \]
  konvergiert. Dann ist $u$ bereits harmonisch.
\end{cor}

\begin{proof}
  Nach Korollar \ref{cor:harm-darstellung} erhalten wir für jedes $|z|
  < r <R$ und $n \in \N$
  \[
  u_n(z) = \frac{1}{2\pi} \int_0^{2\pi} P(z, re^{i\phi})
  u_n(re^{i\phi}) \d[\phi].
  \]
  Da $u_n$ gleichmäßig auf $\partial B_r(0)$ konvergiert, gilt die
  Integralformel auch für $u$. Damit ist $u$ auf allen $B_r(0)$ mit $r <
  R$ und schlußendlich auf $B_R(0)$ harmonisch.
\end{proof}

\begin{thm}[Harnacksches Prinzip für Kreisscheiben]
  \label{thm:harnack}
  Sei $M \in \R$ und $u_0 \leq u_1 \leq \dots \leq M$ eine monoton
  wachsende, beschränkte Folge harmonischer Funktionen von $B_R(0)$ nach
  $\R$. Dann konvergiert $(u_n)_{n\in \N}$ kompakt gegen eine harmonische
  Funktion $u: B_R(0) \ra \R$.
\end{thm}

\begin{proof}
  Sei $K \subset B_R(0)$ kompakt. Dann existieren $\rho < r < R$, so
  dass $K \subset \overline{B_\rho(0)}$ gilt. Sei $\epsilon > 0$ gegeben und
  \[
  \epsilon' := \epsilon \frac{r - \rho}{ r+ \rho} > 0.
  \]
  Nun ist $(u_n(0))_{n \in \N}$ eine monoton wachsende Folge und
  beschränkt, d.h. es existiert ein $N \in \N$, so dass
  \[
  u_n(0) - u_m(0) \leq \epsilon' \qquad \forall n \geq m \geq N
  \]
  gilt. Für $|z| \leq \rho$ erhalten wir
  \begin{align}
    0 \leq P(z, re^{i\phi}) & = \frac{(r - |z|)(r+|z|)}{|re^{i\phi} -
      z|^2} \nonumber\\
    & \leq \frac{(r- |z|)(r + |z|)}{(r - |z|)^2} \nonumber\\
    & = \frac{r + |z|}{r - |z|} \nonumber\\
    & \leq \frac{ r + \rho}{ r - \rho}. \label{eq:P-absch} 
  \end{align}
  Für beliebige $z \in K$ gilt nun die folgende Integralformel
  \begin{align*}
    u_n(z) - u_m(z) & = \frac{1}{2\pi} \int_0^{2\pi} P(z, re^{i\phi})
    (u_n(re^{i\phi}) - u_m(re^{i\phi})) \d[\phi] \\
    & \stackrel{\eqref{eq:P-absch}}{\leq} \frac{r + \rho}{r - \rho}
    \frac{1}{2\pi} \int_0^{2\pi} (u_n(r e^{i\phi}) -  u_m(re^{i\phi})
    \d[\phi] \\
    & = \frac{r+\rho}{r - \rho} (u_n(0) - u_m(0)) \\
    & \leq \epsilon.
  \end{align*}
  Also konvergiert $(u_n)_{n \in \N}$ kompakt und nach Korollar
  \ref{cor:harm-konvergenz} ist die Grenzfunktion wieder harmonisch.
\end{proof}

\begin{prop}
  \label{prop:harm-rf}
  Sei $X$ eine Riemannsche Fläche und $D \Subset U \subset X$, so dass
  $(U, z)$ eine Karte und $z(D) \subset \C$ eine Kreisscheibe
  ist. Dann ist das Dirichlet-Problem auf $D$ wohldefiniert und
  eindeutig lösbar.
\end{prop}

\begin{proof}
  Die Aussage folgt daraus, dass die Eigenschaft harmonisch zu sein
  invariant unter biholomorphen Transformationen und damit unabhängig
  von der gewählten Kartenabbildung ist. Die Lösung lässt sich dann
  einfach auf $z(D)$ nach Satz \ref{thm:poisson} ermittlen.
\end{proof}

\begin{defin}
  Sei $X$ eine Riemannsche Fläche und $Y \subset X$ offen. Wir
  bezeichnen mit $\Reg(Y)$ die Menge aller Teilgebiete $D \Subset
  Y$, die den Voraussetzungen von Propositon \ref{prop:harm-rf}
  genügen.
  
  Für $u \in C(Y, \R)$ und $D \in \Reg(Y)$ definieren wir $P_D u: Y
  \ra \R$ durch $P_D u |_{Y \setminus D} = u_{Y \setminus D}$ und
  $P_Du|_D$ ist die Lösung des Dirichlet-Problems mit Randwerten
  $u|_{\partial D}$.
\end{defin}

\begin{cor}
  \label{cor:pd-rechenregeln}
  Mit der gleichen Notation wie in der vorherigen Definition erhalten wir
  für beliebige $u, v\in C(Y, \R)$ und $\lambda \in \R$:
  \begin{enumerate}
  \item $P_D(u+v) = P_Du + P_D v$,
  \item $P_D(\lambda u) = \lambda P_D u$ und
  \item $u \leq v \Ra P_Du \leq P_D v$.
  \end{enumerate}
\end{cor}

\begin{proof}
  Die ersten beiden Aussagen ergeben sich aus der Eindeutigkeit
  harmonischer Funktionen und 3. ist eine Folge des Maximumprinzips.
\end{proof}

\begin{cor}
  $u \in C(Y, \R)$ ist genau dann harmonisch, wenn $P_Du = u $ für
  jedes $D \in \Reg(Y)$ gilt.
\end{cor}

\begin{proof}
  \begin{description}
  \item[$\Ra$] Diese Richtung ergibt sich aus der Definition.
  \item[$\La$] Folgt aus der Tatsache, dass harmonisch zu sein eine
    lokale Eigenschaft ist und es zu jedem Punkt $x \in Y$ ein $D \in
    \Reg(Y)$ mit $x \in D$ existiert.
  \end{description}
\end{proof}

\begin{defin}
  Sei $Y \subset X$ offen und $X$ eine Riemannsche Fläche.
  Ein $u \in C(Y, \R)$ heißt
  \begin{enumerate}
  \item \init{subharmonisch}, falls $P_D u \geq u$ für jedes $D \in
    \Reg(Y)$ gilt.
  \item \init{lokal subharmonisch}, falls es zu jedem Punkt in $Y$
    eine Umgebung gibt, auf der $u$ subharmonisch ist.
  \end{enumerate}
\end{defin}
\begin{cor}
  Sei $Y \subset X$ offen und $X$ eine Riemannsche Fläche. Seien
  weiterhin $u,v \in C(Y, \R)$ subharmonisch und $\lambda \geq
  0$. Dann sind $u+v$, $\lambda u$ und $\sup(u,v)$ subharmonisch.
\end{cor}

\begin{proof}
  $u+v$ und $\lambda u$ folgen direkt aus Korollar
  \ref{cor:pd-rechenregeln}. Um die Aussage für $\sup(u,v)$ zu zeigen,
  betrachten wir zunächst eine beliebige Funktion $f \in C(\partial D,
  \R)$. Dann gilt klarerweise $f \leq |f|$. Seien nun $\tilde u$ und
  $\tilde v$ Lösungen des Randwertproblems zu $f$ bzw. $|f|$. Dann
  folgt aus dem Maximumprinzip, dass $\tilde u \leq \tilde v$ gelten
  muss. Andererseits gilt auch $-f \leq |f|$ und aus der Eindeutigkeit
  der harmonischen Funktionen und dem Maximumprinzip erhalten wird $-
  \tilde u \leq \tilde v$. Insgesamt gilt also $|\tilde u| \leq \tilde
  v$. Diese Aussage verwenden wir, um die folgende Abschätzung herzuleiten
  \begin{align*}
    P_D \sup(u,v) & = \frac{1}{2} ( P_D u + P_D v + P_D|u - v|) \\
    & \geq \frac12 (P_D u + P_D v + |P_D u - P_D v|) \\
    & = \sup(P_D u, P_D v) \\
    & \geq \sup(u,v).
  \end{align*}
  Also ist auch $\sup(u,v)$ subharmonisch.
\end{proof}

\begin{thm}[Maximumprinzip für lokal subharmonische Funktionen]
  Sei $X$ eine Riemannsche Fläche, $Y \subset X$ ein Gebiet und $u
  \in C(Y, \R)$ lokal subharmonisch, so dass ein $x_0 \in Y$ existiert
  mit
  \[
  u(x_0) = \sup_{y\in Y} u(y).
  \]
  Dann ist $u$ konstant.
\end{thm}

\begin{proof}
  Sei $M:= \{y \in Y | u(y) = u(x_0) =: c \}$. Angenommen es gälte $M \neq
  Y$. Dann existierte ein $a \in \partial M$ und aus der Stetigkeit
  von $u$ folgte $u(a) = u(x_0)$. Nun müsste in jeder Umgebung von $a$
  ein $x$ existieren, so dass $u(x) < u(x_0)$ wäre, d.h. wir könnten ein
  $D \in \Reg(y)$ finden mit $a \in D$ und $u|_{\partial D} \not
  = c$. Wenn wir nun $D$ klein genug wählten, könnten wir annehmen, dass $u$
  subharmonisch in einer Umgebung von $\bar D$ wäre. Also gälte $u \leq
  P_D u =: v$. Damit wäre $v$ harmonisch auf $D$ und $v|_{\partial D} =
  u|_{\partial_D} \leq c$ und es folgte $v \leq c$ auf $\bar D$, aber
  $c = u(a) \leq v(a)$. Die Funktion $v$ nähme also ihr Maximum im Inneren an und
  wäre nach Proposition \ref{prop:max-prinzip-harm} konstant,
  insbesondere wäre $v = c$ auf $\partial D$. Dies ist ein Widerspruch zu
  $u|_{\partial D} \not = c$. Also muss $M = Y$ gelten.
\end{proof}

\begin{cor}
  Sei $X$ eine Riemannsche Fläche, $Y\subset X$ offen und $u \in C(Y,
  \R)$ lokal subharmonisch. Dann ist $u$ bereits subharmonisch.
\end{cor}

\begin{proof}
  Sei $D \in \Reg(Y)$ beliebig. Da $P_D u$ harmonisch auf $D$ ist, ist
  $v:= u- P_Du$ lokal subharmonisch auf $D$ und $v|_{\partial D}
  = 0$. Also folgt aus dem Maximumprinzip $v \leq0$ auf $D$ und
  damit $P_Du \geq u$.
\end{proof}

\begin{lemma}
  Sei $u \in C(Y, \R)$ subharmonisch und $B \in \Reg(Y)$. Dann ist
  $P_B u$ auch subharmonisch.
\end{lemma}

\begin{proof}
  Wir setzen $v:= P_B u$ und wählen $D \in \Reg(Y)$ beliebig. Wir
  zeigen dann die Abschätzung $P_D v \geq v$. Auf $Y \setminus D$ gilt $P_Dv = v$
  und auf $Y\setminus B$ gilt
  \[
  P_Dv = P_DP_Bu = P_Du \geq u = P_Bu = v.
  \]
  Damit gilt $v - P_Dv \leq 0$ auf $Y \setminus (B \cap D)$,
  insbesondere $v- P_D v \leq 0$ auf $\partial (Y \setminus (B \cap
  D))$ und aus dem Maximumprinzip für harmonische Funktionen folgt $v
  - P_D v \leq 0$ auf $B \cap D$, da $v - P_D v$ dort harmonisch
  ist. Insgesamt ergibt sich $P_Dv \geq v$ auf ganz $Y$.
\end{proof}

\begin{lemma}[Perron]
  \label{lemma:perron}
  Sei $M \subset C(Y, \R)$ eine nicht-leere Menge subharmonischer
  Funktionen mit den folgenden Eigenschaften:
  \begin{enumerate}
  \item $u,v \in M \Ra \sup(u,v) \in M$,
  \item $u \in M, D \in \Reg(Y) \Ra P_Du \in M$ und
  \item $\exists K \in \R: u \leq K \quad \forall u \in M$.
  \end{enumerate}
  Dann ist die Funktion $u^\ast : Y \ra \R$ gegeben durch $u^\ast(x) :=
  \sup\{u(x) \mid u \in M \}$ harmonisch auf $Y$.
\end{lemma}

\begin{proof}
  Sei $a \in Y$ und $D \in \Reg(Y)$ eine Umgebung von $a$. Sei
  weiterhin $(u_n)_{n \in \N} \subset M$ mit $\lim_{n \ra \infty}
  u_n(a) = u^\ast(a)$. Aufgrund von 1. können wir ohne Einschränkung davon ausgehen, dass
  $u_0 \leq u_1 \leq u_2 \leq \dots$ ist. Setzen wir $v_n := P_D u_n$, so gelten
  \begin{align}
    & u_n \leq v_n \leq u^\ast \label{eq:v-absch} \\
    \intertext{und}
   & v_0 \leq v_1 \leq v_2 \leq \dots \ . \nonumber
  \end{align}
  Nach dem Harnackschen Prinzip konvergiert $(v_n)_{n \in \N}$ auf $D$
  gegen eine harmonische Funktion $v: D \ra \R$ und aus
  \eqref{eq:v-absch} erhalten wir $v(a) = u^\ast(a)$ und $v \leq
  u^\ast$ auf $D$.
  
  Wir zeigen zunächst, dass $v = u^\ast|_D$ gilt. Sei dazu $x \in
  D$ beliebig und $(w_n)_{n\in \N}\subset M$ mit
  $\lim_{n \ra \infty}w_n(x) = u^\ast(x)$. Aufgrund von 1. und
  2. können wir $v_n \leq w_n = P_D w_n$ und $w_n
  \leq w_{n+1}$ annehmen. Also konvergiert $(w_n)_{n \in \N}$ auf $D$
  gleichmäßig gegen $w: D \ra \R$ mit $v \leq w \leq u^\ast$. Damit
  gilt aber $u^\ast(a) = v(a) \leq w(a) \leq u^\ast(a)$ und aus
  dem Maximumprinzip angewandt auf $v-w$ ergibt sich $v = w$ auf
  $D$. Insbesondere $v(x) = w(x) = u^\ast(x)$. Also ist $u^\ast = v$
  harmonisch auf $D$ und, da $D$ beliebig gewählt
  war, auf ganz $Y$.
\end{proof}

\begin{defin}
  \label{def:perron}
  Sei $Y \subset X$ offen, $\partial Y \neq \varnothing$ und
  $f: \partial Y \ra \R$ stetig und beschränkt. Setzen wir $K :=
  \sup\{f(x) : x \in \partial Y\}$, dann bezeichnet
  $\pe_f$ die Menge aller $u \in C(\bar Y, \R)$, so dass
  \begin{enumerate}
  \item $u|_Y$ subharmonisch ist und
  \item $u|_{\partial Y} \leq f$ und $u \leq K$ gilt.
  \end{enumerate}
  $\pe_f$ wird als \init{Perronklasse} von $f$ bezeichnet.
\end{defin}

\begin{cor}
  Mit der Notation aus Definition \ref{def:perron} folgt, dass $u^\ast := \sup_{u
    \in \pe_f} u$ harmonisch auf $Y$ ist.
\end{cor}

\begin{proof}
  Wir wenden Lemma \ref{lemma:perron} auf $M := \pe_f$ an.
\end{proof}

\begin{rem}
  Damit $u^\ast$ eine Lösung des Dirichlet-Problems auf $Y$ ist,
  müsste
  \begin{align}
  \lim_{\substack{y\ra x\\y \in Y}}u^\ast(y) =
  f(x) \label{eq:dirichlet-rand} 
  \end{align}
  für jedes $x \in \partial Y$ gelten. Dies ist leider nicht immer der
  Fall. Im Folgenden wollen wir Kriterien für die Randpunkte angeben,
  die sicherstellen, dass \eqref{eq:dirichlet-rand} erfüllt ist.
\end{rem}

\begin{defin}
  Sei $X$ eine Riemannsche Fläche und  $Y \subsetneq X$ offen. Ein
  Punkt $x \in \partial Y$ heißt \init{regulär}, falls es eine
  offene Umgebung $U \subset X$ von $x$ und eine Funktion $\beta  \in
  C(\bar Y \cap U)$ gibt, so dass
  \begin{enumerate}
  \item $\beta|_{Y \cap U}$ harmonisch ist,
  \item $\beta(x) =0$ und $\beta(y) < 0$ für beliebige $y \in \bar Y
    \cap U \setminus \{x\}$ gilt.
  \end{enumerate}
  $\beta$ wird als \init{Barriere} von $x$ bezeichnet.
\end{defin}

\begin{lemma}
  \label{lemma:zsh-komp}
  Sei $X$ eine Riemannsche Fläche, $Y \subset X$ eine offene Menge und
  $Z \subset Y$ eine Zusammenhangskomponente von $Y$. Dann ist $Z$ offen.
\end{lemma}

\begin{proof}
  Sei $x \in Z$. Dann finden wir eine offene, zusammenhängende
  Koordinatenumgebung $U \subset Y$ mit $x \in U$. Nun ist aber
  $Z$ die maximale zusammenhängende Teilmenge, die $x$ enthält und da
  $U$ auch zusammenhängend gewählt wurde, muss $U \subset Z$
  gelten. Also ist $x$ ein innerer Punkt und damit $Z$ offen.
\end{proof}

\begin{cor}
  Sei $x \in \partial Y$ ein regulärer Randpunkt und $Z \subset Y$
  offen mit $x \in \partial Z$.\\
  Dann ist $x$ ein regulärer Randpunkt von $Z$. Insbesondere hat jede
  Zusammenhangskomponente von $Y$ regulären Rand.
\end{cor}

\begin{lemma}
  \label{lemma:regulär-trennen}
  Sei $x \in \partial Y$ ein regulärer Randpunkt und $V$ eine offene
  Umgebung von $x$ mit reellen Konstanten $m \leq c$. Dann existiert
  ein $v \in C( \bar Y, \R)$, so dass die Eigenschaften
  \begin{enumerate}
  \item $v|_Y$ ist subharmonisch,
  \item es gelten $v(x) = c$ und $v|_{\bar Y \cap V} \leq c$ und
  \item $v|_{\bar Y\setminus V} = m$
  \end{enumerate}
  erfüllt sind.
\end{lemma}

\begin{proof}
  Wir können ohne Einschränkung $c = 0$ annehmen. Sei $U$ eine offene Umgebung von $x$ mit Barriere $\beta \in C(\bar
  Y \cap U, \R)$. Wir können nun eine offene Umgebung $\tilde V \Subset U \cap V$
  von $x$ finden. Dort gilt
  \[
  \sup\{\beta(y) \mid y \in \partial \tilde V \cap \bar Y \} < 0.
  \]
  Also existiert ein $k > 0$, so dass $k\beta|_{\partial \tilde V \cap
    \bar Y} < m$ ist.
  Schlußendlich definieren wir
  \[
  v :=
  \begin{cases}
    \sup(m, k\beta) & \text{auf } \bar Y \cap \tilde V\\
    m & \text{auf } \bar Y \setminus V
  \end{cases}.
  \]
  Damit ist $v$ stetig und genügt den Bedingungen 1 bis 3.
\end{proof}

\begin{lemma}
  Sei $Y \subsetneq X$ offen, $f \in C(\partial Y, \R)$ beschränkt und
  $u^\ast := \sup_{u \in \pe_f}u$, wobei $\pe_f$ die Perronklasse von
  $f$ bezeichnet. Dann gilt für jeden regulären Randpunkt $x \in \partial Y$
  \[
  \lim_{\substack{y \ra x\\y \in Y}} u^\ast(y) = f(x).
  \]
\end{lemma}

\begin{proof}
  Sei $\epsilon > 0$ beliebig. Dann existiert eine relativ kompakte offene
  Umgebung $V$ von $x$ mit
  \[
  f(x) - \epsilon \leq f(y) \leq f(x) + \epsilon \qquad \forall y
  \in \partial Y \cap V.
  \]
  Seien $k, K \in \R$ mit $k \leq f(y) \leq K$ für jedes $y
  \in \partial Y$.
  \emph{Fall a)} Nach Lemma \ref{lemma:regulär-trennen} können wir eine
  Funktion $v \in C(\bar Y, \R)$ finden, die subharmonisch auf $Y$ ist und
  \begin{align*}
    v(x) & = f(x) - \epsilon, \\
    v|_{\bar Y \cap V} & \leq f(x) - \epsilon \quad \text{und} \\
    v|_{\bar Y \setminus V} & = k - \epsilon
  \end{align*}
  erfüllt. Damit ist $v|_{\partial Y} \leq f$ und $v \leq K$. Also ist $v \in
  \pe_f$ und es folgt $v \leq u^\ast$. Wir erhalten also
  \[
  \liminf_{\substack{y \ra x\\y \in Y}} u^\ast(y) \geq v(x) = f(x) - \epsilon.
  \]
  
  \emph{Fall b)} Erneut durch Lemma \ref{lemma:regulär-trennen} erhalten
  wir ein $w \in C(\bar Y, \R)$, das subharmonisch auf $Y$ ist und
  \begin{align*}
    w(x) & = - f(x), \\
    w|_{\bar Y \cap V} & \leq -f(x) \quad \text{und} \\
    w|_{\bar Y \setminus V}  & = -K
  \end{align*}
  erfüllt. Also gilt für alle $u \in \pe_f$ und $y \in \partial Y \cap
  V$ die Abschätzung
  $u(y) \leq f(x) + \epsilon$ und wir erhalten
  \[
  u(y) + w(y) \leq \epsilon \qquad \forall y \in \partial Y \cap V.
  \]
  Weiterhin erhalten wir für jedes $z \in \bar Y \cap \partial V$
  \[
  u(z) + w(z) \leq K + w(z) = K - K  = 0.
  \]
  Das Maximumprinzip angewandt auf die subharmonischen Funktion $u+w$
  auf $Y \cap V$ ergibt $u + w \leq \epsilon $ auf $\bar Y \cap
  V$. Also ist $u|_{\bar Y \cap V} \leq \epsilon - w|_{\bar Y \cap V}$
  für jedes $u \in \pe_f$ und wir folgern
  \[
  \limsup_{\substack{y \ra x\\y \in Y}}u^\ast \leq \epsilon - u(x) = f(x)
  + \epsilon.
  \]
  
  Fall a) und b) zusammen ergeben dann die Behauptung.
\end{proof}

\begin{thm}
  \label{thm:dirichlet}
  Sei $X$ eine Riemannsche Fläche und  $Y \subsetneq X$ offen mit
  regulärem Rand. Dann ist das Dirichlet-Problem für jedes
  beschränkte $f \in C(\partial Y, \R)$ lösbar auf $Y$.
\end{thm}

Nun geben wir noch ein hinreichendes Kriterium für reguläre Randpunkte an.

\begin{thm}
  \label{thm:reg-rand}
  Sei $Y \subset \C$ offen und $a \in \partial Y$. Existieren $m \in \C$
  und $r > 0$, so dass $a \in \partial B_r(m)$ und $\bar
  B_r(m) \cap Y = \varnothing$ gelten. Dann ist $a$ ein regulärer Randpunkt von $Y$.
\end{thm}

\begin{proof}
  Setzen wir $c := \frac{a+m}{2}$, dann stellt $\beta(z) := \log
  \frac{r}{2} - \log|z-c|$ eine Barriere von $a$ dar. Denn $|a - c| =
  \frac{r}{2}$ und damit gilt $\beta(a) = 0$. Weiterhin gilt für $z
  \in Y$, dass $|z-m| > r$ ist. Aus der Dreiecksungleichung erhalten
  wir $|z-c| > r - |c-m| = \frac{r}{2}$. Also ist $\beta(y) < 0$ für
  beliebige $y \in Y$. Es bleibt also noch zu zeigen, dass $\beta|_Y$
  harmonisch ist. Für diese Berechnung nehmen wir ohne Einschränkung
  $c = 0$ an und betrachten $\beta$ als Funktion von $x$ und $y$,
  wobei $z = x + iy$ gilt. Wir werden in den folgenden Gleichungen
  aber dennoch (inkonsistenter Weise) $z$ verwenden und fassen dieses
  dann als Funktion von $x$ und $y$ auf. Wir erhalten für die
  Ableitungen die Gleichungen 
  \begin{align*}
    \frac{\partial \beta(x,y)}{\partial x} & = - \frac{x}{|z|^2}, \\
    \frac{\partial^2 \beta(x,y)}{\partial x^2} & = \frac{2x^2 -
      |z|^2}{|z|^4}, \\
    \frac{\partial^2 \beta(x,y)}{\partial y^2} & = \frac{2y^2 -
      |z|^2}{|z|^4} && \text{und} \\
    \Delta \beta(x,y) & = \frac{2 x^2 - |z|^2 + 2y^2 -|z|^2}{|z|^4} \\
    & = \frac{2 |z|^2 - 2|z|^2}{|z|^4} \\
    & = 0.
  \end{align*}
  Also ist $\beta$ auch harmonisch auf $Y$. Wir erhalten, dass $\beta$
  eine Barriere in $a$ ist.
\end{proof}

\begin{rem}
  \label{rem:reg-rand}
  Satz \ref{thm:reg-rand} gibt uns nur eine Beschreibung für reguläre
  Randpunkte für Gebiete in $\C$. Dies ist jedoch ausreichend, da die
  Eigenschaft ein regulärer Randpunkt zu sein lokal ist und deshalb
  auf Karten nachgerechnet werden kann. Weiterhin ist die Eigenschaft
  harmonisch zu sein invariant unter biholomorphen Transformationen,
  so dass wir das obige $\beta$ einfach auf einer Karte angeben
  können. Somit überträgt sich Satz \ref{thm:reg-rand} direkt auf
  Gebiete beliebiger Riemannscher Flächen.
\end{rem}


%%% Local Variables: 
%%% mode: latex
%%% TeX-master: "../Bachelor"
%%% End: 
