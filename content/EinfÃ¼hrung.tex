
\section{Einführung}
\label{sec:einführung}

\subsection{Notation}
\label{sec:notation}

In dieser Arbeit bezeichnet die Menge $\N = \{1, 2, 3, \dots\}$ die natürlichen
Zahlen und die Menge $\N_0 = \{0 \} \cup \N$ die natürlichen Zahlen mit der
0. Wir bezeichnen mit $\Z, \Q, \R$ und $\C$ die Mengen der ganzen,
rationalen, reellen und komplexen Zahlen. Die 
obere Halbebene ist definiert durch
\[
\h := \{ z \in \C \mid \Im(z) > 0 \}
\]
und die Riemannsche Zahlenkugel wird gegeben durch
\[
\P^1 := \C \cup \{\infty\}.
\]

Weiterhin bezeichnen wir mit $\hol, \mer$ bzw. $\diff$
die Garben der holomorphen, meromorphen bzw. glatten Funktionen und mit
$\Omega, \mer^{(1)}$ bzw. $\diff^{(1)}$ die Garben der holomorphen,
meromorphen bzw. glatten 1-Formen einer Riemannschen Fläche. Auf Karten $(U,z)$ haben die
Elemente $\omega \in \diff^{(1)}$ die Form
\[
\omega|_U = f_z \d[z] + g_z\d[\bar z]
\]
mit glatten Funktionen $f_z,g_z \in \diff(U)$. $\diff^{1,0}$ bezeichnet
dann die Garbe der glatten 1-Formen, so dass $g_z$ auf jeder Karte
verschwindet. $\diff^{0,1}$ wird analog definiert.

Auf einer Riemannschen Fläche $X$ kann die abelsche Gruppe der
Divisoren $\Div(X)$ durch Abbildungen der Form $D: X \ra \Z$, wobei
auf jeder kompakten Teilmenge $K \subset X$ jeder Divisor $D|_K$ fast
überall verschwindet, definiert werden. $\Div(X)$ ist durch
\[
D \leq D' \quad :\LRa \quad D(x) \leq D'(x) \quad \forall x \in X
\]
halbgeordnet. Zusätzlich kann jeder meromorphen Funktion $f \in \mer(Y)$ mit $Y
\subset X$ offen via
\[
(f): Y \ra \Z, y \mapsto \ord_y(f)
\]
ein Divisor zugeordnet werden. Dies ermöglicht die Definition der
Garbe der meromorphen Vielfachen des Divisors $-D$, d.h. für offene
Teilmengen $U \subset X$ definieren wir
\[
\hol_D(U) := \{f \in \mer(U) \mid (f) \geq D|_U \}.
\]

Für jede Abbildung $f$ bezeichnen wir mit $\im(f)$ deren Bild und für
jedes $z \in \C$ bezeichnen wir mit $\Im(z)$ seinen Imaginärteil. Diese
Notationsphilosophie birgt die Gefahr einiger Verwechslungen,
allerdings ist sie allgemein üblich, weshalb sie auch in diesem Text
zum Einsatz kommt. Es sollte hoffentlich aus dem jeweiligen Kontext
klar werden, welche der beiden Bedeutungen gemeint ist.

Für einen topologischen Raum $X$ bedeutet $Y \Subset X$, dass $Y$
eine Teilmenge von $X$ und zusätzlich relativ kompakt ist.

\subsection{Einleitung}
\label{sec:einleitung}

Diese Bachelor-Arbeit wuchs aus einem Seminarvortrag im Wintersemester
2012/2013 zum Thema "`Riemannsche Flächen"' betreut durch Herrn Hendrik
Kasten. Herr Kasten hat mir auch die Bearbeitung dieses Themas
ermöglicht und dafür möchte ich ihm an dieser Stelle sehr herzlich
danken. Genau so wie das Seminar ist auch diese Arbeit an Otto Forsters
Buch "`Lectures on Riemann surfaces"' \cite{For} angelehnt. Da diese
Arbeit direkt auf das Seminar aufbaut und um den Umfang derselben im
Rahmen zu halten, wurden die ersten 15 Kapitel
als bekannt vorausgesetzt. Jedoch sollten zumindest die grundlegenden
Begriffe geklärt und etwas Motivation für die Fragestellung an
die Hand gegeben werden.

Riemannsche Flächen sind interessante Gebilde, die in verschiedenen
Bereichen der Mathematik ein Rolle spielen. In der Funktionentheorie
selbst beschäftigt man sich mit der Fragestellung unter welchen
Voraussetzungen man eine eindeutige holomorphe Logarithmus- oder
Wurzel-Funktion definieren kann, in der Theorie der algebraischen
Kurven stellt man fest, dass normale projektive Kurven
kompakten Riemannschen Flächen entsprechen und in der Differentialgeometrie
beobachtet man, dass kompakte Riemannsche Flächen Beispiele für
sphärische, euklidische und hyperbolische Geometrien liefern und es
wird außerdem deutlich, dass es (in einem geeigneten Sinn) "`viel
mehr"' kompakte hyperbolische Flächen gibt, als sphärische und euklidische.

Zunächst wollen wir jedoch das zentrale Objekt dieser Arbeit
definieren.

\begin{defin}[Riemannsche Fläche]
  \label{def:rf}
  Sei $X$ ein zusammenhängender Hausdorffraum. Ein \init{Atlas}
  ist eine Menge $\{(U_i, z_i) \mid i \in I\}$, bestehend aus
  offenen Teilmengen $U_i$, die eine Überdeckung von $X$ bilden und
  Homöomorphismen
  \[
  z_i: U_i \ra V_i \subset \C,
  \]
  wobei $V_i$ offene Teilmengen von $\C$ bezeichnen. Ein
  \init{komplexer Atlas} ist ein Atlas, so dass alle
  Kartenwechselabbildungen
  \[
  z_i \circ z_j^{-1} : z_j(U_i \cap U_j) \ra z_i(U_i \cap U_j)
  \]
  holomorph sind.

  Eine Karte $(z,U)$ heißt \init{verträglich} mit einem
  komplexen Atlas, falls sie zusammen mit dem bisherigen Atlas wieder
  einen komplexen Atlas ergibt. Vereinigt man einen komplexen Atlas mit allen
  verträglichen Karten, so erhält man einen maximalen komplexen
  Atlas, der auch als \init{komplexe Struktur} bezeichnet wird.

  Eine \init{Riemannsche Fläche} $X$ ist ein zusammenhängender
  Hausdorffraum zusammen mit einer komplexen Struktur.
\end{defin}

Diese Definition erinnert stark an die Definition einer glatten
zweidimensionalen reellen Mannigfaltigkeit. Ein Unterschied ist
jedoch, dass wir hier nicht die Existenz einer abzählbaren Basis der
Topologie fordern. Wie wir im Kapitel \ref{sec:Topologie}
sehen werden, wäre diese Forderung im Falle von Riemannschen Flächen
redundant, denn wir bekommen sie gratis aus der zusätzlichen Forderung
der Komplexen Struktur.

Das Ziel dieser Arbeit ist es nun Riemannsche Flächen zu
klassifizieren. Dies für alle Riemannschen Flächen zu erreichen wäre
ein sehr ambitioniertes Ziel, deshalb werden wir uns hier auf den Fall
der kompakten Riemannschen Flächen beschränken\footnote{Immerhin
  erhalten wir auf dem Weg dahin noch eine vollständige
  Charakterisierung aller einfach zusammenhängenden Riemannschen
  Flächen (vgl. Kapitel \ref{sec:RMT}).}. Nun stellt sich nur noch die
Frage, was Klassifikation im Fall von Riemannschen Flächen
bedeutet. Dazu muss, um auf die Sprache der Kategorientheorie
zurückzugreifen, gesagt werden, was die Morphismen und vor allem die
Isomorphismen zwischen Riemannschen Flächen sind. Diese stellen sich
dann als holomorphe bzw. biholomorphe Abbildungen heraus, die wir nun
definieren wollen.

\begin{defin}
  Seien $X,Y$ Riemannsche Flächen und $f: X \ra Y$ eine Abbildung. $f$
  heißt \init{holomorph}, falls für jede Karte $(U,z)$ von $X$ und
  jede Karte $(V, w)$ von $Y$ mit $f(U) \subset W$ die Abbildung
  \[
  w \circ f \circ z^{-1} : z(U) \subset \C \ra w(V) \subset \C
  \]
  holomorph im Sinne der Funktionentheorie 1 ist. $f$ heißt
  \init{biholomorph}, falls $f$ bijektiv und sowohl $f$ als auch $f^{-1}$
  holomorph sind. Zwei Riemannsche Flächen $X$ und $Y$ heißen \init{konform
  äquivalent}, falls es eine biholomorphe Abbildung zwischen ihnen gibt.
\end{defin}

Nun sind wir in der Lage das Ziel der Arbeit sauber zu
formlieren. Wir wollen alle kompakten Riemannschen Flächen bis auf
Konforme Äquivalenz klassifizieren. Dieses Ziel werden wir mit Satz
\ref{thm:uniformisierung} tatsächlich erreichen können. 

In den folgenden Kapitel wird nun aufbauend auf dem Stoff der
Grundvorlesungen, der Funktionentheorie 1 und des Seminars die Sprache
und das Handwerkszeug entwickelt, um den Uniformisierungssatz für
kompakte Riemannsche Flächen beweisen zu können.


%%% Local Variables: 
%%% mode: latex
%%% TeX-master: "../Bachelor"
%%% End: 
