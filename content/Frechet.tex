
\section{Fr\'echet Räume}
\label{sec:frechet}

Dieses recht kurze Kapitel trägt rudimentäre Eigenschaften von
Fr\'echet-Räumen zusammen, wie sie in Kapitel \ref{sec:Runge} benötigt
werden. Der Satz von Hahn-Banach \ref{thm:hahn-banach} wird genannt
und es wird auf eine Quelle für den Beweis verwiesen, der aufgrund
seiner Umfänglichkeit leider nicht im Rahmen dieser Arbeit geführt
werden konnte.

\begin{defin}
  \label{def:frechet}
  Sei $E$ ein topologischer Vektorraum. $E$ heißt
  \init{Fr\'echet-Raum}, falls die folgenden Bedingungen erfüllt sind:
  \begin{enumerate}
  \item Es existiert eine abzählbare Familie von Seminormen $(p_n)_{n
      \in \N}$, die die Topologie von $E$ erzeugen,
  \item $E$ ist Hausdorffsch und
  \item $E$ ist vollständig.
  \end{enumerate}
  Weiterhin definieren wir für $\epsilon > 0$ und $x \in E$ die Menge
  \[
  U(p_k, \epsilon, x) := \{ y \in E \mid p_k(x-y) < \epsilon\}
  \]
\end{defin}


\begin{rem}
  \label{rem:frechet}
  \begin{enumerate}
  \item Konvergenz bezüglich der Topologie von $E$ bedeutet dann
    nichts anderes, als Konvergenz bezüglich aller Halbnormen $p_k$.
  \item Die erste Bedingung bedeutet, dass $U\subset X$ genau dann
    offen ist, falls zu jedem $x \in U$ ein $K \in \N$ und ein
    $\epsilon > 0$ existiert, so dass
    \[
    \cap_{k=1}^K U(p_k, \epsilon, x) \subset U
    \]
    gilt.
  \item Aus der Bedingung, dass $E$ Hausdorffsch ist, folgt, dass $x =
    y$ dann und nur dann gilt, wenn $p_k(x-y) = 0$ für jedes $k \in
    \N$ gilt. Dies folgt leicht aus der Tatsache, dass falls $p_k(x-y)
    = $ für beliebige $k \in \N$ gilt, $x \in U(p_k, \epsilon, y)$ für
    alle $k \in \N$ und $\epsilon > 0$ gilt. Also ist $x$ in jeder
    offenen Umgebung von $y$ enthalten. Da $E$ Hausdorffsch ist, ist
    dies nur möglich für $x = y$. Die Umkehrung ist natürlich immer wahr.
  \end{enumerate}
\end{rem}

\begin{prop}
  \label{prop:frechet-metrik}
  Sei $E$ ein Fr\'echet-Raum mit Halbnormen $(p_k)_{k \in \N}$. Dann
  ist
  \[
  d: E \times E \ra \R, \quad d(x,y) := \sum_{k=1}^\infty 2^{-k}
  \frac{p_k(x-y)}{1 + p_k(x-y)}
  \]
  eine Metrik und die von $d$ induzierte Topologie stimmt mit der
  Topologie von $E$ überein. Also ist $E$ insbesondere
  metrisierbar. $d$ wird auch als \init{Fr\'echet-Metrik} bezeichnet.
\end{prop}

\begin{proof}
  Zunächst ist die Abbildung $d$ wohldefiniert, denn die Abbildung,
  die durch $x \mapsto \frac{x}{1+x}$ gegeben ist, bildet das
  Intervall $[0, \infty[$ auf das Intervall $[0,1[$ ab. Der Rest folgt
  aus der Konvergenz der geometrischen Reihe. Wir bezeichnen
  diese  Abbildung mit $f$.

  Die Symmetrie der Metrik ist klar und für die Definitheit wählen wir
  $x\neq y \in E$. Dann existiert nach der dritten Bemerkung ein $k \in
  \N$, so dass $p_k(x-y) >0$ ist. Dann ist aber $d(x,y) \geq
  2^{-k}p_k(x-y) > 0$. Da für $x =y$ klarerweise $d(x,y) =0$ gilt,
  folgt die Definitheit von $d$. Es bleibt also nur noch die die
  Dreiecksungleichung zu zeigen. Dazu stellen wir zunächst fest, dass
  $f$ streng monoton wachsend ist, denn die Ableitung $f'(x) =
  \frac{1}{(1+x)^2}$ ist positiv auf dem gesamten Intervall $[0,
  \infty[$. Nun folgt aus der Dreicksungleichung für die Halbnormen,
  dass $p_k(x+y) \leq p_k(x) + p_k(y)$ für alle $k \in \N$ gilt. Aus
  der Monotonie von $f$ erhalten wir dann
  \begin{align*}
    f(p_k(x+y)) & \leq f(p_k(x) + p_k(y)) \\
    & = \frac{ p_k(x) + p_k(y)}{1 + p_k(x) + p_k(y)}\\
    & = \frac{p_k(x)}{1+ p_k(x)+p_k(y)} + \frac{p_k(y)}{1+p_k(x) +
      p_k(y)} \\
    & \leq \frac{p_k(x)}{1+p_k(x)} + \frac{p_k(y)}{1+ p_k(y)} \\
    & = f(p_k(x)) + f(p_k(y)).
  \end{align*}
  Aus dieser Dreiecksungleichung für $f \circ p_k$ folgt direkt die
  Dreieckungleichung für $d$, da die Gleichung $d(x,y) = \sum_{k=1}^\infty
  2^{-k}f(p_k(x-z))$ gilt.
  
  Nun wenden wir uns der Topologie zu. Wir wissen, dass die Bälle
  \[
  B_\epsilon(x) := \{ y \in E \mid d(x,y) < \epsilon \} \qquad \forall
  x \in E, \quad \epsilon > 0
  \]
  die Topologie der Metrik $d$ erzeugen. Um nun zu zeigen, dass die
  Topologie von $E$ mindestens genau so fein ist, wie die von $d$
  erzeugte, müssen wir zeigen, dass die metrischen Bälle bezüglich der
  Topologie von $E$ offen sind. Genauer reicht es uns sogar
  zu zeigen, dass $x$ ein innerer Punkt von $B_\epsilon(x)$ bezüglich
  der eigentlichen Topologie von $E$ ist, denn
  für jeden anderen Punkt $y \in B_\epsilon(x)$ finden wir ein
  $\delta >0$, so dass $B_\delta(y) \subset B_\epsilon(x)$ und dann
  brauchen wir nur eine (bzgl. $E$) offene Umgebung $U$ von $y$ mit $U
  \subset B_\delta(y)$ zu finden. Also zeigen wir auch hier, dass $y$
  ein innerer Punkt von $B_\delta(y)$ ist. Sei nun also
  $B_\epsilon(x)$ gegeben. Dann existiert ein $K \in \N$, so dass
  $\sum_{k=K+1}^\infty 2^{-k} < \frac{\epsilon}{2}$ gilt. Wählen wir
  nun ein $\tilde \epsilon := \frac{\epsilon}{2K}$, so  existiert ein
  $\delta > 0$, so dass für $0 \leq x < \delta$ $0 \leq f(x) < \tilde \epsilon$
  gilt. Dies folgt aus der Stetigkeit von $f$. Wir behaupten nun, dass
  $U := \bigcap_{k=1}^{K-1} U(p_k, \delta, x) \subset B_\epsilon(x)$
  gilt. Dieses $U$ ist dann nach der zweiten Bemerkung offen und wir
  haben die Behauptung gezeigt. Sei nun also $y \in U$, dann gilt
  $p_k(x-y) < \delta$ für alle $k \leq K-1$ und damit
  \begin{align*}
    d(x,y) & = \sum_{k=1}^\infty 2^{-k} f(p_k(x-y)) \\
    & \leq \sum_{k=1}^{K-1}2^{-k} f(p_k(x-y)) + \sum_{k=K}^\infty 2^{-k} \\
    & < \sum_{k=1}^{K-1} f(p_k(x-y)) + \frac{\epsilon}{2} \\
    & < \sum_{k=1}^{K-1} \tilde \epsilon + \frac{\epsilon}{2} \\
    & = (K-1) \tilde \epsilon + \frac{\epsilon}{2}\\
    & < \epsilon.
  \end{align*}
  Nun müssen wir noch zeigen, dass die von $d$ induzierte Topologie
  mindestens so fein ist, wie die Topologie von $E$. Sei dazu $U \subset
  E$ offen. Wählen wir $x \in U$, so existiert ein $K \in \N$ und ein
  $\epsilon > 0$, so dass $\bigcap_{k=1}^K U(p_k, \epsilon, x) \subset
  U$. Weiterhin können wir zu $f$ explizit eine Umkehrfunktion
  angeben. Für diese gilt
  \[
  f^{-1} := [0, 1[\ra [0, \infty[, \quad f^{-1}(x) := \frac{x}{1-x}
  \]
  und damit ist $f^{-1}$ auch stetig und wir finden ein $\delta > 0$,
  so dass $0 \leq f^{-1}(x) < \epsilon$ für alle $0 \leq x <
  \delta$. Setzen wir $\tilde \delta := 2^{-K} \delta$, so behaupten
  wir, dass $B_{\tilde \delta}(x) \subset \cap_{j=1}^K U(p_j,
    \epsilon, x) \subset U$ gilt. Sei dazu $y \in B_{\tilde
      \delta}(x)$. Aus $d(x,y) < \tilde \delta$ folgt dann, dass
    $2^{-k} f(p_k(x-y)) < 2^{-K} \delta$ gilt. Insbesonder erhalten
    wir für alle $k \leq K$, dass $f(p_k(x-y)) < 2^{-(K-k)} \delta
    \leq \delta$ gilt. Und aus der Stetigkeit von $f^{-1}$ folgt, dass
    $p_k(x-y) < \epsilon$ ist für alle $k \leq K$. Also ist $y \in
    \bigcap_{k=1}^K U(p_k, \epsilon, x) \subset U$. Damit ist $U$ auch
    bezüglich der von $d$ induzierten Topologie offen. Insgesamt haben
    wir die Äquivalenz der Topologien gezeigt.
\end{proof}
  
\begin{thm}
  \label{thm:frechet-abgeschlossen}
  Sei $F \subset E$ ein abgeschlossener Untervektorraum. Dann ist $F$
  ein Fr\'echet-Raum.
\end{thm}

\begin{proof}
  Wir statten $F$ mit der Teilraumtopologie aus. Dann ist $F$
  als abgeschlossener Teilraum automatisch vollständig. Weiterhin wird
  die Topologie wieder durch eine abzählbare Familie von Halbnormen
  erzeugt, denn wir können die Familie von $E$ auf $F$
  einschränken. Dass $F$ Hausdorffsch ist, ergibt sich daraus, dass
  wir die Topolgie von $F$ auch dadurch erhalten, dass wir die Metrik
  von $E$ auf $F$ einschränken. Dann ist $F$ aber auch ein metrischer
  Raum und damit insbesondere Hausdorffsch. Insgesamt ist $F$ also ein
  Fr\'echet-Raum.
\end{proof}

\begin{thm}
  \label{thm:frechet-summe}
  Seien $E, F$ Fr\'echet-Räume. Dann ist $E \times F$ mit der
  Produkttopologie auch ein Fr\'echet-Raum.
\end{thm}

\begin{proof}
  Da $E$ und $F$ jeweils eine Metrik besitzen, wird die
  Produkttopologie von $d(x,y) = d_E(x_1, y_1) + d_Y(x_2, y_2)$, wobei
  $x = (x_1,x_2)$ und $y = (y_1, y_2)$ gilt, induziert. Dann ist $E
  \times F$ aber direkt vollständig und hausdorffsch. Weiterhin
  erhalten wir die gesuchte abzählbare Familie von Halbnormen, in dem
  wir die beiden Familien von $E$ und $F$ verwenden. Diese erzeugen
  die gleiche Topologie, wie die Produkttopologie. Dies lässt sich auf
  analoge Art und Weise wie im Beweis von Proposition
  \ref{prop:frechet-metrik} zeigen.
\end{proof}

\begin{thm}[von Hahn-Banach]
  \label{thm:hahn-banach}
  Sei $E$ ein Fr\'echet-Raum, $E_0 \subset E$ ein Untervektorraum und
  $\phi_0 \in E_0'$. Dann existiert ein $\phi \in E'$ mit $\phi|_{E_0}
  \equiv \phi_0$
\end{thm}

\begin{proof}
  Wir verweisen auf die Standardliteratur zur Funktionalanalysis z.B. \cite[Satz3.6]{Rud}.
\end{proof}

\begin{cor}
  \label{cor:frechet-dicht}
  Sei $E$ ein Fr\'echet-Raum. Seien weiterhin $A \subset B \subset E$
  Untervektorräume. Falls für jedes $\phi \in E'$ mit $\phi|_A \equiv
  0$ auch $\phi|_B \equiv 0$ ist, so liegt $A$ dicht in $B$.
\end{cor}

\begin{proof}
  Sei $A$ nicht dicht in $B$. Wir versuchen nun ein stetiges, lineares
  Funktional auf $E$ zu konstruieren, dass zwar auf $A$ verschwindet,
  jedoch nicht auf $B$. Wir finden nun ein $b_0 \in B \setminus
  \bar A$. Setzen wir $E_0 := \bar A \oplus \C b_0$ und
  \[
  \phi_0: E_0 \ra \C, \quad a + \lambda b_0 \mapsto \lambda,
  \]
  so ist $\phi_0$ stetig. Angenommen $\phi_0$ wäre nicht stetig, dann
  wäre es aufgrund der Linearität insbesondere nicht in $0$
  stetig. Also fänden wir eine Folge $(a_n + \lambda_n b_0)_{n \in
    \N} \subset E_0$ mit Grenzwert $0$, wobei aber $(\lambda_n)_{n \in
    \N}$ nicht gegen 0 konvergiert. Wir können nun ohne Einschränkung
  annehmen, dass $\lambda_n \geq 0$, denn wir finden auf jeden Fall
  eine Teilfolge mit $\lambda_n \geq 0$ oder eine Teilfolge mit
  $\lambda_n \leq 0$. Im zweiten Fall betrachten wir $(-\lambda_n)_{n
    \in \N}$ und $(-(a_n + \lambda_n b_0))_{n \in \N}$, wobei dann auch
  diese Folge gegen $0$ konvergiert. Dies folgt sofort durch die
  Halbnormen. Unter der Voraussetzung, dass $(\lambda_n)_{n\in\N}$ nicht gegen $0$
  konvergiert, fänden wir eine Teilfolge $(\lambda_{n_l})_{l\in \N}$
  und ein $C > 0$, so dass $\lambda_{n_l} > C$ für beliebige $l \in
  \N$ gelten würde. Dann setzten wir $\lambda := \liminf_{l \ra \infty}
  \lambda_{n_l} \in [C, \infty]$.

  Wäre nun $\lambda = \infty$. Dann müsste bereits
  $\lambda_{n_l} \ra \infty$ für $l \ra \infty$ gelten. Nun
  konvergierte die Folge $(a_n + \lambda_n b_0)_{n \in \N}$ bezüglich
  jeder Halbnorm $p_k$ gegen 0, also fänden wir $K_k > 0$, so dass
  \[
  p_k(a_{n_l} + \lambda_{n_l}b_0) < K_k \qquad \forall k \in \N
  \]
  gelten würde. Wir erhielten dann, dass weiterhin
  \[
  p_k(\lambda_{n_l}^{-1}a_{n_l} + b_0) < \lambda_{n_l}^{-1} K_k
  \xrightarrow{l \ra \infty} 0
  \]
  für jedes $k \in \N$ wäre. Damit wäre aber $b_0$ in $\bar A$ enthalten. Ein
  Widerspruch.

  Für den Fall, dass $\lambda < \infty$ wäre, fänden wir eine weitere
  Teilfolge, die gegen $\lambda$ konvergierte. Wir gehen ohne
  Einschränkung davon aus, dass bereits $(\lambda_{n_l})_{l \in \N}$
  gegen $\lambda$ konvergierte. Dann erhielten wir aber aus der
  Dreiecksungleichung
  \[
  p_k(a_{n_l} + \lambda b_0) \leq p_k(a_{n_l} + \lambda_{n_l}) +
  |\lambda - \lambda_{n_l}| p_k(b_0) \ra 0
  \]
  für $l \ra \infty$ und alle $k \in \N$. Damit würde aber $\lim_{l
    \ra \infty}(-a_{n_l}) = \lambda b_0$ gelten und es wäre wieder
  $b_0 \in \bar A$. Also ist $\phi_0$ stetig.
  
  $\phi_0$ lässt sich nun nach Satz \ref{thm:hahn-banach}
  auf ganz $E$ fortsetzen. Also existiert ein $\phi \in E'$ mit
  $\phi|_{E_0} \cong \phi$. Dann ist aber $\phi|_A$ konstant 0, aber
  dies gilt nicht für $\phi|_B$.
\end{proof}

%%% Local Variables: 
%%% mode: latex
%%% TeX-master: "../Bachelor"
%%% End: 
