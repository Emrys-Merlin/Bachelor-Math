
\section{Abelsche Fuchssche Gruppen}
\label{sec:fuchs}

\todo{Sl durch PSl geeignet ersetzen}
\todo{$\id$ ausschließen}

\begin{defin}
  Die diskreten Untergruppen von $PSl(2,\R)$ werden als Fuchssche
  Gruppen bezeichnet.
\end{defin}

\begin{rem}
  Um über Diskretheit sprechen zu können, benötigt $Sl(2, \R)$
    eine Topolgie. Hier verwenden wir einfach die Teilraumtopolgie,
    die durch die Standardtopologie auf $\Mat_{4\times4}(\R) \cong
    \R^4$ gegeben ist.
\end{rem}

\begin{thm}
  \label{thm:fixpkt}
  Sei $S \in Sl(w, \R)$. Dann gelten
  \begin{itemize}
  \item Für $|\tr S| < 2$ besitzt $S$ einen Fixpunkt in $\h$.
  \item Für $|\tr S| = 2$ besitzt $S$ einen Fixpunkt auf $\R \cup \{\infty\}$.
  \item Für $|\tr S| > 2$ besitzt $S$ zwei Fixpunkte auf $\R \cup \{\infty\}$.
  \end{itemize}
\end{thm}

\begin{proof}
  Gelte
  \[
  S =
  \begin{pmatrix}
    a & b\\
    c & d
  \end{pmatrix} \in Sl(2, \R)
  \]
  Für einen Fixpunkt $S \langle z \rangle = z$ muss die Gleichung
  \begin{align}
    \label{eq:fixpkt}
    c z^2 + (d-a)z - b = 0
  \end{align}
  Falls $c \neq 0$ gilt erhalten wir unter Ausnutzung von $\det S = 0$
  \begin{align*}
    z_{1,2} & = \frac{a-d}{2c} \pm \sqrt{\frac{(d-a)^2}{4c^2} +
      \frac{b}{c}} \\
    & = \frac{a-d}{2c} \pm \frac{1}{2c} \sqrt{(\tr S)^2 - 4}
  \end{align*}
  Nun hängt die Anzahl und Art der Lösungen von der Diskriminante
  ab. Wir erhalten also die folgenden Fälle:
  \begin{itemize}
  \item Für $|\tr S| < 2$ erhalten wir zwei Lösungen in $\C$, wobei
    der Imaginärteil von Null verschieden ist. Aufgrund des
    Vorzeichens liegt nur eine dieser Lösungen in $\h$. Also finden
    wir einen Fixpunkt in $\h$.
  \item Für $|\tr S| = 2$ erahlten wir eine Lösung in $\R$.
  \item Für $|\tr S| > 2$ erhalten wir zwei Lösungen in $\R$.
  \end{itemize}
  Nun müssen wir noch den Fall $c = 0$ betrachten. Dann vereinfacht
  sich die Gleichung \eqref{eq:fixpkt} unter Ausnutzung von $1 = \det
  S = ad$ zu
  \[
  z = a^2 z - ba.
  \]
  Weiterhin wissen wir, dass $\tr S = a + \frac{1}{a}$ erfüllt. Damit
  wissen wir aber bereits, dass $|\tr S| \geq 2$ gilt, denn die
  Funktion, die durch $x \mapsto \frac{1}{x} + x$ gegeben ist, hat
  zwei globale Betragsminima bei $\pm 1$, was eine einfache
  Kurvendiskussion zeigt. Betrachten wir zunächst den Fall $|\tr S| =
  2$, so muss $a = \pm 1$ gelten. Dann ist aber $\infty$ die einzige
  Lösung der obigen Gleichung. Für $|\tr S| > 2$ ist $\infty$ auch
  eine Lösung, allerdings erhalten wir als zweite Lösung $z =
  \frac{ba}{1-a^2} \in \R$, da $a^2 \neq 1$ gelten muss. Dies zeigt
  die verbleibenden Fälle.
\end{proof}

\begin{defin}
  Die Elemente aus $Sl(2,\R)$ mit $|\tr S| < 2$ ($ = 2$ oder $ > 2$)
  heißen elliptisch (parabolisch, hyperbolisch).
\end{defin}

\begin{lemma}
  \label{lemma:psl-trafo}
  Sei $S \in Sl(2, \R)$. Dann gelten
  \begin{enumerate}
  \item Ist $S$ elliptisch mit Fixpunkt $w \in \h$, so existiert ein $C \in Sl(2, \R)$, so
    dass $C\langle w \rangle = i$ gilt. Weiterhin gilt dann für ein
    $\phi \in \R$ die Gleichung
    \[
    C S C^{-1} =
    \begin{pmatrix}
      \cos(\phi) & \sin(\phi) \\
      -\sin(\phi) & \cos(\phi)
    \end{pmatrix}.
    \]
  \item Ist $S$ parabolisch mit Fixpunkt $w \in \R \cup \{\infty\}$,
    so existiert ein $C \in Sl(2, \R)$, so dass $C\langle w\rangle =
    \infty$ erfüllt. Weiterhin gilt dann für ein $b \in \R$ die Gleichung
    \[
    C S C^{-1} =
    \begin{pmatrix}
      1 & b\\
      0 & 1
    \end{pmatrix}.
    \]
  \item Ist $S$ hyperbolisch mit Fixpunkte $w_1 \neq w_2 \in \R \cup
    \{\infty\}$, so existiert ein $C \in Sl(2, \R)$, so dass $C\langle
    w_1 \rangle = 0$ und $C\langle w_2 \rangle = \infty$ erfüllt
    sind. Weiterhin folgt dann, dass ein $\lambda > 0$ existiert, das
    die folgende Gleichung erfüllt
    \[
    C S C^{-1} = 
    \begin{pmatrix}
      \lambda & 0\\
      0 & \lambda^{-1}
    \end{pmatrix}.
    \]
  \end{enumerate}
\end{lemma}

\begin{proof}
  \emph{Fall 1:} Sei $S \in Sl(2, \R)$ elliptisch und $w \in \h$ der
  Fixpunkt von $S$. Nun gilt nach Satz \ref{thm:aut}, dass ein $C \in
  Sl(2,\R)$ existiert, so dass $C \langle w \rangle = i$ ist. Setzen
  wir
  \[
  \tilde S = CSC^{-1}
  \begin{pmatrix}
    a & b\\
    c & d
  \end{pmatrix},
  \]
  so gilt $\tilde S\langle i \rangle = i$ und wir erhalten die
  Gleichung
  \[
  ai +b = -c + di
  \]
  Nun sind die $a,b,c,d \in \R$ und wir erhalten, dass $a = d$ und $b
  = - c$ gelten muss. Dann erhalten wir aber, dass
  \[
  1 = \det S = ad - bc = a^2 + b^2
  \]
  gelten muss. Damit finden wir aber geeignete $\phi \in \R$, so dass
  $a = \cos(\phi)$ und $b = \sin(\phi)$ erfüllen und damit hat $\tilde
  S$ die gewünschte Form.

  \emph{Fall 2:} Sei $S \in Sl(2, \R)$ parabolisch und sei $w \in \R
  \cup \{\infty\}$ der Fixpunkt von $S$. Wir setzen
  \[
  S =
  \begin{pmatrix}
    a & b\\
    c & d
  \end{pmatrix}.
  \]
  Falls nun $w = \infty$ gilt, muss $c =0$ sein. Dann folgt aber aus
  $|\tr 2 = 1\|$ und $\det S = 1$, dass $a = b = \pm 1$ gelten
  muss. Da wir uns aber in $PSl(2,\R)$ befinden, können wir ohne
  Einschränkung $a = b = 1$ annehmen. Also ist $S$ bereits in der
  gewünschten Form. Falls $w \neq \infty$ gilt, so betrachten wir
  \[
  C =
  \begin{pmatrix}
    0 & 1 \\
    -1 & w
  \end{pmatrix}
  \in PSl(2,\R)
  \].
  $C$ erfüllt die Gleichung $C \langle w \rangle > = \infty$. Also
  erfüllt $\tilde S = C S C^{-1}$ die Gleichung $\tilde S\langle \infty
  \rangle = \infty$. Nach dem Fall den wir bereits gezeigt haben, hat
  dann $\tilde S$ die gewünschte Form.

  \emph{Fall 3:} Sei $S \in PSl(2, \R)$ hyperbolisch und seien $w_1
  \neq w_2 \in \R \cup \{\infty\}$ die Fixpunkte von $S$. Wir
  betrachten zunächst den Fall $w_1 = 0$ und $w_2 = \infty$ und wählen
  die obige Darstellung für $S$. Dann folgt aber aus $S\langle 0
  \rangle = 0$ und $S\langle \infty \rangle = \infty$, dass $b = c = 0$
  gelten muss. Aus $\det S = 1$ folgt dann, dass $a = \frac{1}{d}$
  gilt. Da $S \in PSl(2,\R)$ liegt, können wir ohne Einschränkung
  $\lambda a > 0$ annehmen, was die Behauptung in diesem Fall
  zeigt. Als nächstes sei $w_1 = \infty$ oder $w_2 = \infty$. Wir
  nehmen ohne Einschränkunt $w_2 = \infty$ an. Dann ist $w_1 \neq
  \infty$ und wir setzen
  \[
  C =
  \begin{pmatrix}
    1 & - w_1\\
    0 & 1
  \end{pmatrix}
  \in PSl(2, \R)
  \]
  Dann erfüllt $C\langle \infty \rangle = \infty$ und $C\langle w_1
  \rangle = 0$ und wir erhalten für $\tilde S = C S C^{-1}$ die beiden
  Gleichungen $\tilde S \langle \infty \rangle = \infty$ und $\tilde S
  \langle 0 \rangle = 0$ und wir haben es auf den obigen Fall
  zurückgeführt. Sind nun sowohl $w_1$ als auch $w_2$ von $\infty$
  verschieden, so führen wir die gleiche Argumentation mit der Matrix
  \[
  C =
  \begin{pmatrix}
    (w_1 - w_2)^{-1} & - w_1(w_1-w_2)^{-1}\\
    1 & -w_2
  \end{pmatrix}
  \in PSl(2, \R)
  \]
  durch und erhalten auch hier, dass $\tilde S = C S C^{-1}$ die
  gewünschten Eigenschaften erfüllt.
\end{proof}

Nun wollen wir die abelschen Fuchsschen Gruppen charakterisieren. Wir
beginnen mit der folgenden Proposition.

\begin{prop}
  \label{prop:komm-fix}
  Seien $T, S \in Sl(2, \R)$ mit $TS = ST$. Dann bildet $S$ die
  Fixpunktmenge von $T$ injektiv auf sich selbst ab.
\end{prop}

\begin{proof}
  Sei $w$ ein Fixpunkt von $T$, dann gilt
  \[
  TS \langle p \rangle = ST\langle p \rangle = S\langle p \rangle.
  \]
  Also ist auch $S\langle p \rangle$ ein Fixpunkt von $T$. Da nun $S$
  auch in $PSl(2, \C) = \Aut(\P^1)$ liegt, ist $S$ insbesondere auf
  der gesamten Fixpunktmenge von $T$ injektiv. $S$ bildet also
  Fixpunkte von $T$ injektiv auf Fixpunkte von $T$ ab.
\end{proof}

\begin{thm}
  \label{thm:komm-fix}
  Seien $T,S \in PSl(2, \R)\setminus \{\id\}$. Dann kommutieren $T$ und $S$ genau dann,
  wenn die Fixpunktmengen übereinstimmen.
\end{thm}

\begin{proof}
  Gelte zunächst $TS = ST$. Dann bildet nach Proposition
  \ref{prop:komm-fix} die Abbildung $ST$ die Fixpunktmenge von $T$
  injektiv auf die Fixpunktmenge von $T$ ab. Da diese nach Satz
  \ref{thm:fixpkt} endlich ist, wirkt $ST$ auch surjektiv auf dieser
  Menge. Analog wirkt $TS$ sowohl injektiv, als auch surjektiv auf der
  Fixpunktmenge von $S$. Dann gilt aber direkt $\# \Fix(S) = \#
  \Fix(T)$. Aus Satz \ref{thm:fixpkt} folgt nun auch, dass wir nur die
  beiden Fälle ein bzw. zwei Fixpunkte überprüfen müssen. Gelte also
  zunächst $\# \Fix(S) = 1$. Sei $p$ der Fixpunkt von $S$, dann ist
  nach Proposition \ref{prop:komm-fix} $T\langle p \rangle$ auch ein
  Fixpunkt von $S$ und da nur ein Fixpunkt existiert, gilt $ p = T
  \langle p \rangle$. Also ist $p$ auch der einzige Fixpunkt von $T$
  und die Fixpunktmengen stimmen überein. Dann ist aber $T$ (und auch
  $S$) parabaolisch und nach Lemma \ref{lemma:psl-trafo} existiert ein
  $C \in PSl(2,\R)$, so dass
  \[
  \tilde T = C T C^{-1} =
  \begin{pmatrix}
    \lambda & 0 \\
    0 & \lambda^{-1}
  \end{pmatrix}
  \]
  für ein $\lambda > 0$ gilt. Setzen wir
  \[
  \tilde S = C S C^{-1} =
  \begin{pmatrix}
    a & b \\
    c & d
  \end{pmatrix},
  \]
  so folgt aus $ST = TS$ auch $\tilde S \tilde T = \tilde T \tilde
  S$. Diese Gleichung liefert uns $\lambda^{-1} c = \lambda c$ und
  $\lambda^{-1} b = \lambda$. Da $T \neq \id$ ist $\lambda \neq 1$ und
  wir erhalten $b = c = 0$ und es folgt $a = d^{-1}$. Also
  erhalten wir
  \[
  \Fix(\tilde S) = \{0, \infty\} = \Fix(\tilde T).
  \]
  Nach Rücktransformation folgt die Aussage für $S$ und $T$.

  Gelte nun $\Fix(S) = \Fix(T)$, so stimmen die Typen von $S$ und $T$
  überein weiterhin finden wir nach Lemma \ref{lemma:psl-trafo} ein $C
  \in PSl(2, \R)$, so dass $C S C^{-1}$ und $C T C^{-1}$ gemeinsam
  eine der Formen aus dem obigen Lemma annehmen. Nun haben wir aber im
  elliptischen Fall Rotationen und die Gruppe der Rotationen im $\R^2$
  ist kommutativ. Im parabolischen Fall handelt es sich um
  Translationen und diese bilden auch eine kommutative Gruppe. Der
  hyperbolische Fall liefert Diagonalmatrizen, diese kommutieren immer
  mit allen anderen Matrizen, kommutieren also insbesondere
  untereinander. Aus der Kommutativität von $C S C^{-1}$ und $C T
  C^{-1}$ folgt dann dirket die Kommutativität von $S$ und $T$.
\end{proof}

\begin{thm}
  \label{thm:abelsch-zyklisch}
  Sei $G \subset PSl(2, \R)$ eine Fuchssche Gruppe. Dann ist $G$ genau
  dann abelsch, wenn $G$ zyklisch ist.
\end{thm}

\begin{proof}
  Falls $G$ zyklisch ist, ist $G$ klarerweise kommutativ. Betrachten
  wir also den Fall, dass $G$ kommutativ ist. Nach Satz
  \ref{thm:komm-fix} sind die Fixpunktmengen aller Elemente aus $G$
  identisch. Sie haben also alle den gleichen Typ.

  Betrachten wir zunächst den elliptischen Fall. Dann können alle
  Elemente von $G$ durch Konjugation mit einem festen $C \in PSl(2,\R)$ auf
  die Form
  \[
  \begin{pmatrix}
    \cos(\phi) & \sin(\phi) \\
    -\sin(\phi) & \cos(\phi)
  \end{pmatrix}
  \]
  gebracht werden. Betrachten wir nun die Abbildungen von $G$ als
  lineare Abbildungen auf $\R^2$, so stellen wir nach Identifizierung
  mit $\C$ fest, dass es sich bei den Rotationen einfach nur um
  Multiplikation mit $e^{i\phi}$ handelt. Wir können also $G$ mit
  einer diskreten Untergruppe von $S^1$ identifizieren. Angenommen
  diese Untergruppe wäre nicht diskret. Dann gäbe es $\phi \neq \psi
  \in \R$, so dass $e^{in\phi} \neq e^{im\phi}$ für beliebige $n, m
  \in \Z$. Nun wissen wir aber nach Lemma \ref{lemma:gitter}, dass
  eine nichttriviale additive Untergruppe von $\R$ ein Gitter sein
  muss, d.h. die additive Untergruppe $L := \{ n \phi + m \psi \mid n,m \in
  \Z\}$ kann nicht diskret sein. Weiterhin ist die Abbildung
  \[
  p: \R \ra S^1, \quad \phi \mapsto e^{i\phi}
  \]
  stetig, also kann auch $p(L) \cong G$ nicht diskret sein. Ein
  Widerspruch. Also muss ein $\phi \in \R$ existieren, so dass
  \[
  G \cong \left \{ e^{in\phi} \mid n \in \Z \right \}
  \]
  gilt. Damit ist $G$ zyklisch.

  Gehen wir nun zum parabolischen Fall über, so können wir alle
  Elemente von $G$ durch Konjugation mit einem festen $C \in PSl(w,
  \R)$ auf die Form
  \begin{align}
    \label{eq:parabolisch}
  \begin{pmatrix}
    1 & b\\
    0 & 1
  \end{pmatrix}
  \end{align}
  bringen, wobei $b \in \R$ gilt. Hier erhalten wir direkt einen
  Homöomorphismus von der Menge der Matrizen der Form
  \eqref{eq:parabolisch} nach $\R$, in dem wir einfach die Komponente
  $b$ auswählen. Es ist also das Bild von $G$ unter diesem
  Homöomorphismus eine diskrete, abelsche
  Untergruppe von $\R$. Nun wissen wir aber nach Lemma \ref{lemma:gitter},
  dass die nichttrivialen diskreten, abelschen Untergruppen von $\R$ Gitter und
  damit (im eindimensionalen Fall) zyklisch sind. Also ist auch $G$
  zyklisch.

  Wenden wir uns dem letzten Fall zu. Seien also alle Elemente von $G$
  hyperbolisch. Dann finden wir erneut ein feste $C \in PSl(2, \R)$,
  da alle Element von $G$ auf die Form
  \begin{align}
    \label{eq:hyperbolisch}
  \begin{pmatrix}
    \lambda & 0 \\
    0 & \lambda^{-1}
  \end{pmatrix}
  \end{align}
  bringt. Auch hier finden wir einen Homöomorphismus von den Matrizen
  der Form \eqref{eq:hyperbolisch} und in diesem Fall $\R_{> 0}$, in
  dem wir die Matrizen auf $\lambda^2$ abbilden. Das es sich
  tatsächlich um einen Homöomorphismus handelt, folgt aus der
  Tatsache, dass wir das Vorzeichen von $\lambda$ nach Definition von
  $PSl(2, \R)$ vernachlässigen können. Nun können wir aber $\R_{> 0}$
  durch $\log$ homöomorph auf $\R$ abbilden. Also finden wir wieder
  eine abelsche, diskrete Untergruppe von $\R$. Diese ist wieder
  zyklisch und da wir nur mit Homöomorphismen gearbeitet haben, folgt
  direkt auch, dass $G$ zyklisch sein muss.
\end{proof}

%%% Local Variables: 
%%% mode: latex
%%% TeX-master: "../Bachelor"
%%% End: 
