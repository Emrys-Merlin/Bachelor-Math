\section{Der Riemannsche Abbildungssatz}
\label{sec:RMT}

In diesem Abschnitt wollen wir den zentralen Satz der Bachelor-Arbeit
beweisen: den Riemannschen Abbildungssatz. Er charakterisiert alle
einfach zusammenhängenden Riemannschen Flächen und besagt,
dass es bis auf Biholomorphie nur drei verschiedene gibt: die
Riemannsche Zahlenkugel $\P^1$, die komplexe Ebene $\C$
und die Einheitskreisscheibe $B$.

Später werden wir dieses Resultat im Zusammenhang mit der
Überlagerungstheorie verwenden, denn hier besagt das Resultat nichts
anderes, als dass es nur drei mögliche Universelle Überlagerungen zu
jeder beliebigen Riemannschen Fläche gibt. Das erleichtert deren
Untersuchung natürlich ungemein.

Zu Beginn des Kapitels werden einige Hilfssätze bewiesen, bevor wir am
Ende dann die Früchte unserer Arbeit ernten können.

Zunächst einmal beweisen wir den Riemannschen Abbildungssatz nicht
direkt für einfach zusammenhängende Riemannsche Flächen, sondern für
Riemannschle Flächen mit verschwindender holomorpher deRham Gruppe
(vgl. \ref{def:hol-deRham}). Wie wir aber am Ende sehen werden ist dies
keine Einschränkung, allerdings müssen wir erst ein paar Eigenschaften
für diese Riemannschen Flächen zusammentragen.

Der restliche Teil dieses Kapitels besteht dann darin Funktionenfolgen
zu konstruieren, die schlussendlich gegen unsere gesuchte biholomorphe
Funktion konvergieren.

\begin{defin}
  \begin{align*}
    B_\infty(0) & := \C \\
    B & := B_1(0)
  \end{align*}
\end{defin}

\begin{defin}
  \label{def:hol-deRham}
  Sei $X$ eine Riemannsche Fläche, dann heißt
  \[
  Rh_\hol^1(X) := \Omega(X)/\d[\hol(X)]
  \]
  \init{holomorphe deRham Gruppe}.
\end{defin}

\begin{lemma}
  \label{lemma:ex-log}
  Sei $X$ eine Riemannsche Fläche mit $Rh^1_\hol(X) = 0$. Dann gilt:
  \begin{enumerate}
  \item Zu jeder holomorphen Funktionen $f: X \ra C^\ast$ existieren
    $g, h \in \hol(X)$ mit $e^g = f$ und $h^2 = f$.
  \item Zu jeder harmonischen Funktion $u: X \ra \R$ existiert ein $f
    \in \hol(X)$ mit $u = \Re(f)$.
  \end{enumerate}
\end{lemma}

\begin{proof}
  \begin{enumerate}
  \item Es gilt $f^{-1}\d[f] \in \Omega(X)$ und da $Rh^1_\hol(X) = 0$
    finden wir ein $g \in \hol(X)$ mit $\d[g] = f^{-1} \d[f]$.\\
    Wir können annehmen, dass ein $a \in X$ existiert mit $ e^{g(a)} =
    f(a)$, ansonsten betrachten wir $g +$ const. Weiterhin erhalten wir:
    \begin{align*}
      \d[(f e^{-g})] & = \d[(f)] e^{-g} + f e^{-g} (-1) f^{-1}\d[f] \\
      & = 0
    \end{align*}
    Also ist $fe^{-g} \equiv $const und aus $f(a) e^{-g(a)} = 1$
    erhalten wir:
    \[
    f = e^g
    \]
    Damit ist der erste Teil der Aussage gezeigt. Für den zweiten
    setzen wir nun einfach
    \[
    h := e^{\frac{g}{2}}
    \]
  \item Nach (19.4) \footnote{Jede harmonische 1-Form ist Realteil von genau einer holomorphen
    1-Form} existiert genau ein $\omega \in \Omega(X)$, so dass $\d[u]
    = \Re(\omega)$ gilt. Da nun wieder $\Omega(X) = \d[\hol(X)]$ ist
    erhalten wir zunächst $\d[g] = \omega$ für ein $g \in \hol(X)$ und
    daraus weiterhin: $\d[u] = \Re(\d[g])$. Insgesamt erhalten wir:
    \[
    u = \Re(g) + \text{const.}
    \]
  \end{enumerate}
\end{proof}

\begin{thm}
  \label{thm:Gebiet-Kreis}
  Sei $X$ eine nicht kompakte Riemannsche Fläche und $Y \Subset X$ ein
  Gebiet mit $Rh^1_{\hol}(Y) = 0$ und regulärem Rand. \\
  Dann existiert eine biholomorphe Abbildung auf den Einheitskreis $B$.
\end{thm}

\begin{proof}
  Sei $a \in Y$. Nach Weierstrass \ref{thm:Lösung-Divisor} existiert ein $g \in
  \hol(X)$, das eine Nullstelle erster Ordnung bei $a$ besitzt und
  ansonsten $g(x) \neq 0$ für alle $x \neq a$ gilt. \\
  Nach Satz \ref{thm:dirichlet} existiert ein stetiges $u: \bar Y \ra
  \R$, s.d. $u|_Y$ harmonisch ist und
  \[
  u(y) = \log(g(y)) \qquad \forall y \in \partial Y
  \]
  erfüllt. \\
  Nach Lemma \ref{lemma:ex-log} existiert ein $h \in \hol(Y)$ mit $u =
  \Re(h)$. \\
  Wir setzen nun $f := e^{-h} g \in \hol(Y)$ und behaupten $f: Y \ra
  B$ ist biholomorph.
  \begin{itemize}
  \item $f(Y) \subset B$: Sei $y \in Y$ beliebig. Dann gilt:
    \begin{align*}
      |f(y)| & = | e^{-h(y)} | \cdot | g(y)| \\
      & = e^{-u(y)} e^{\log(|g(y)|)} \\
      & = \exp(\log|g(y)| - u(y))
    \end{align*}
    Daran erkennen wir, dass $|f|$ auf $\bar Y$ fortgesetzt werden
    kann und dort nach Konstruktion konstant den Wert 1 annimmt. Da
    $f$ eine holomorphe Funktion ist werden Extrema auf dem Rand
    angenommen und da $f$ weiterhin nicht konstant ist ($f(a) = 0$)
    erhalten wir aus dem Maximum-Prinzip:
    \[
    |f(y)| < 1 \qquad \forall y \in Y
    \]
    und damit $f(Y) \subset B$.
  \item $f$ ist eigentlich: Sei $0 < r < 1$ und $Y_r :=
    f^{-1}(\overline{B_r(0)})$. \\
    Dann ist $Y_r$ kompakt als abgeschlossene Teilmenge von $\bar Y$.
  \item $f$ ist biholomorph: $f$ nimmt jeden Wert gleich häufig an, da
    es sich um eine eigentlich Abbildung handelt. Weiterhin wird $0$
    genau einmal angenommen. Also ist $f: Y \ra B$ bijektiv und
    holomorph. Das genügt bereits, um zu zeigen, dass $f$ biholomorph
    ist. 

  \end{itemize}

\end{proof}

\begin{prop}
  \label{prop:kreis}
  Sei $f: B_r(0) \ra B_{r'}(0)$ holomorph. Dann gilt: $|f'(0)| \leq \frac{r'}{r}$
\end{prop}

\begin{proof}
  Unter Ausnutzung der Cauchyschen Integralformel erhalten wir:
  \begin{align*}
    |f'(0)| & = \left | \frac{1}{2\pi i} \int_{|z| = r} \frac{f(z)}{z^2}
      \d[z] \right | \\
    & = \left | \frac{1}{2 \pi i} \int_0^1 \frac{f(r e^{2\pi it})}{r^2
        e^{4\pi i t}} r 2\pi i e^{2\pi i t} \d[t] \right | \\
    & \leq \int_0^1 \frac{|f(r e^{2\pi it})|}{r} \d[t] \\
    & \leq \frac{r'}{r} \int_0^1 \d[t] \\
    & = \frac{r'}{r}
  \end{align*}
\end{proof}

\begin{lemma}
  \label{lemma:funktion-kompakt}
  Sei $G \subset \C$ ein Gebiet, so dass $\C\setminus G$ einen inneren
  Punkt besitzt. Sei $\omega_0 \in G$. Dann ist
  \[
  \{f \in \hol(B): f(B) \subset G \text{ und } f(0) = \omega_0
  \} \label{eq:menge} \tag{$\ast$}
  \]
  eine kompakte Teilmenge von $\hol(B)$ bzgl. der Konvergenz auf
  kompakten Teilmengen.
\end{lemma}

\begin{proof}
  Sei $a  \in \mathring{(\C \setminus G)}$ und wir bezeichnen die
  Menge in  \eqref{eq:menge} mit $M$.\\
  Behauptung: $z \mapsto \frac{1}{z-a}$
  bildet $G$ biholomorph auf ein Teilgebiet eines $B_r(0), \quad r <
  \infty$ ab.\\
  $d:= \operatorname{dist}(\{a\},G)> 0$ existiert, da $a \notin \bar
  G$, denn $a$ ist ein innerer Punkt von $\C \setminus G$. Deshalb
  folgt:
  \[
  \left | \frac{1}{z-a} \right |< \frac1d \qquad \forall z \in G
  \]
  Also sind die $f \in M$ gleichmäßig beschränkt und aus dem Satz von
  Montel folgt $M$ ist kompakt.
  %TODO Satz von Montel aufschreiben oder so.
\end{proof}

\begin{defin}
  Sei $\S$\index{$\S$} die Menge aller injektiven holomorphen Funktionen $f: B \ra
  \C$ mit $f(0) = 0$ und $f'(0)= 1$.
\end{defin}

\begin{lemma}
  \label{lemma:schlicht-max-rad}
  Sei $f \in \S$. \\
  Dann existiert ein maximales $r \geq 0$, so dass $B_r(0) \subseteq
  f(B)$.
\end{lemma}

\begin{proof}
  Sei $M:= \{ r\geq 0| B_r(0) \subseteq f(B) \}$ und $r \in M$. Da $f$ injektiv und
  holomorph ist, existiert eine holomorphe Funktion $\phi: B_r(0) \ra
  B$ mit $f \circ \phi = \id_{B_r(0)}$ und $\phi'(0) = 1$. Damit wissen
  wir aber
  \[
  1 = \phi'(0) \leq \frac{1}{r}
  \]
  Also ist $r \leq 1$ und somit $M$ beschränkt. Setzen wir nun $\tilde
  r := \sup M$, so müssen wir noch zeigen, dass $B_{\tilde r}(0) \subseteq
  f(B)$.\\
  Sei $(r_n)_{n \in \N}\subseteq M$ mit $r_n \xrightarrow{n \ra
    \infty} \tilde r$ und $z \in B_{\tilde r}(0)$. Dann existiert ein $n \in \N$,
  so dass $z \in B_{r_n}(0)\subseteq f(B)$ und erhalten somit direkt
  $B_{\tilde r}(0) \subseteq f(B)$.
\end{proof}

\begin{thm}
  \label{thm:schlicht-kompakt}
  $\S$ ist kompakt in $\hol(B)$.
\end{thm}

\begin{proof}
  Sei $(f_n)_{n \in \N} \subset \S$. \\
  Nach Lemma \ref{lemma:schlicht-max-rad}  finden wir zu jedem $n \in \N$
  ein maximales $r_n \geq 0$ mit $B_{r_n}(0) \subseteq f_n(B)$. Setzen wir
  wieder $\phi_n:=f_n^{-1} : B_{r_n}(0) \ra B$ so erhalten wir $1 =
  |\phi'(0)| \leq \frac{1}{r_n}$ und damit $r_n \leq 1$. \\
  Sei $a_n \in \partial B_{r_n}(0)$, so dass $a_n \notin f_n(B)$. Dies ist
  aufgrund der Maximalität von $r_n$ immer möglich. Damit definieren
  wir nun $g_n := \frac{f_n}{a_n}$. Wir erhalten $B \subseteq g_n(B)$
  und
  \[
  1 \notin g_n(B) \label{nicht-1}\tag{$\ast$}
  \]
  Nun ist $g_n(B)$ über $g_n$ homöomorph zu $B$, also einfach
  zusammenhängend. Dies hat zur Folge dass $Rh_\hol^1(g_n(B))$
  verschwindet.\\
  Nach Lemma \ref{lemma:ex-log} existiert ein $\psi: g_n(B) \ra \C^\ast$ mit
  $\psi(0) = i$ und $\psi^2(z) = z - 1 \quad \forall z \in g_n(B)$. \\
  Definiere $h_n := \psi \circ g_n$, also $h_n^2 = g_n - 1$. \\
  Behauptung: $w \in h_n(B) \Ra -w \notin h_n(B)$. \\
  Angenommen: es existierten $z_1,z_2 \in B$ mit $h_n(z_1) = w$ und
  $h_n(z_2) = -w$, dann hätten wir
  \begin{align*}
    g_n(z_1) & = h_n^2(z_1) +1 \\
    & = w^2 +1 \\
    & = (-w)^2 +1 \\
    & = h_n^2(z_2) +1 \\
    & = g(z_2)
  \end{align*}
  Da nun $g$ aber injektiv ist folgt $z_1 = z_2$ und damit $w =
  0$. Wir erhalten $g(z_1) = 1$ ein Widerspruch zu Gleichung \eqref{nicht-1}. \\
  Desweiteren führt $B \subseteq g_n(B)$ zu $U:= \psi(B) \subseteq
  \psi(g_n(B)) = h_n(B)$. Also $(-U) \cap h_n(B) = \varnothing$. \\
  Wir definieren $G := \bigcup_{n \in \N} h_n(B)$ und behaupten
  zunächst: $G$ ist ein Gebiet. \\
  Nach dem Satz über die Gebietstreue sind alle $h_n(B)$ offen und
  somit auch $G$. Es bleibt also zu zeigen, dass $G$ zusammenhängend
  ist. Da wir uns in $\C$ befinden fallen die Begriffe des
  Zusammenhangs und des Wegzusammenhangs zusammen und wir zeigen, dass
  $G$ wegzusammenhängend ist.\\
  Seien dazu $z, w \in G$. Dann existieren $n, \tilde n \in \N$, so
  dass $z \in h_n(B)$ und $w \in h_{\tilde n} (B)$. Weiterhin ist
  $h_l(0) = i \quad \forall l \in \N$, d.h. $i \in h_l(B) \quad
  \forall l \in \N$. Nun sind die $h_l(B)$ Gebiete nach dem Satz über
  die Gebietstreue, also finden wir zwei stetige Kurven $c_1: I \ra
  h_n(B)$ und $c_2: I \ra h_{\tilde n}(B)$ mit $c_1(0) = z$, $c_1(1) =
  i$, $c_2(0) = i$ und $c_2(1) = w$. Definieren wir nun:
  \[
  c: I \ra G, \quad c(t) :=
  \begin{cases}
    c_1(2t) & \text{für } 0 \leq t < \frac12\\
    c_2(2t-1) & \text{für } \frac12 \leq t \leq 1
  \end{cases}
  \]
  so finden wir eine Kurve in $G$ mit $c(0) = z$ und $c(1) = w$. Damit
  ist $G$ wegzusammenhängend, also ein Gebiet. \\
  Weiterhin gilt $G \cap -U = \varnothing$, da $h_n(B) \cap -U =
  \varnothing$. Also besitzt das Komplement von $G$ innere Punkte.
  Schlußendlich verwenden wir noch, dass $h_n(0) = i$ ist und
  erhalten, dass Lemma \ref{lemma:funktion-kompakt} anwendbar ist. $(h_n)_{n
    \in \N}$ besitzt demnach
  eine konvergente Teilfolge. Aufgelöst nach $f_n =
  a_n(1 + h_n^2)$ mit $|a_n| \leq 1$ erhalten wir also eine
  konvergente Teilfolge
  \[
  f_{n_k} \xrightarrow{k \ra \infty} f: B \ra \C
  \]
  Dann ist $f$ klarerweise holomorph und es gilt $f(0) = 0$ und $f'(0)
  = 1$, also ist $f$ insbesondere nicht konstant.\\
  Zu zeigen bleibt nun nur noch die Injektivität von $f$. \\
  Nehmen wir  also an $f$ ist nicht injektiv. Dann existiert ein $a \in \C$,
  so dass $f-a$ mindestens zwei Nullstellen besitzt. Nun können wir
  einen Radius $r < 1$ finden, so dass $f-a$ mindestens $k\geq 2$
  Nullstellen besitzt und auf $\partial B_r(0)$ nicht
  verschwindet. \footnote{Könnten wir kein solches $r$ finden, wäre $f-a$
  bereits konstant Null nach dem Identitätssatz.} Unter Verwendung des
  Null- und Polstellenzählenden Integrals erhalten wir:
  \[
  k = \frac1{2\pi i} \int_{z = |r|} \frac{f'(z)}{f(z) - a} \d[z]
  \]
  Aber das Integral ist stetig bzgl. gleichmäßiger Konvergenz,
  d.h. insbesondere, dass für jedes $f_{n_l}$ nahe genug an $f$ das
  Integral auch $k$ ergeben müsste. Dies ist ein Widerspruch zur
  Injektivität der $f_{n_l}$, also ist $f$ auch injektiv. \\
  Dies vervollständigt den Beweis.
\end{proof}

\begin{lemma}
  \label{lemma:bihol-kreis}
  Sei $R \in ]0, \infty]$, $Y \subsetneq B_R(0)$ ein Gebiet, $0 \in Y$
  und $Rh_\hol^1(Y) = 0$. \\
  Dann existiert ein $0< r <R$ und eine holomorphe Funktion $f: Y \ra B_r(0)$ mit $f(0) =
  0$ und $f'(0) = 1$.
\end{lemma}

\begin{proof}
  \begin{description}
  \item[Fall 1: $R < \infty$:] Wir können ohne Einschränkung annehmen,
    dass $Y \subsetneq B$, ansonsten betrachten wir $\tilde Y =
    \frac{Y}{R}$. \\
    Nach Voraussetzung existiert ein $a \in B \setminus Y$. Wir
    betrachten
    \[
    \phi: B \ra B, \quad \phi(z) := \frac{z-a}{1-\bar a z}
    \]
    % TODO: Wohldefiniertheit von phi?
    Es folgt $0 \notin \phi(Y)$. Nach Lemma \ref{lemma:ex-log} existiert ein
    $g \in \hol(Y)$ mit $g^2 = \phi|_Y$. Also ist $g(Y) \subseteq
    B$. \\
    Wir definieren nun eine weitere Hilfsfunktion
    \[
    \psi(z) := \frac{z-b}{1 - \bar b z}, \qquad b:= g(0)
    \]
    Zusammen mit $g$ ergibt dies unsere letzte Hilfsfunktion $h :=
    \psi \circ g: Y \ra B$ mit $h(0) = 0$.
    \begin{align*}
      \gamma := h'(0) & = \psi'(b) g'(0) \\
      & = \psi'(b) \cdot \frac{\phi'(0)}{2 g(0)} \\
      & = \frac{1}{1-|b|^2} \frac{1-|a|^2}{2b} \\
      & = \frac{1 + |b|^2}{2b}
    \end{align*}
    Damit erhalten wir $|\gamma| > 1$.\footnote{Die Funktion $x \mapsto
    \frac12 \left ( \frac1x + x \right )$ nimmt bei $x = 1$ ihr
    striktes Minimum an und ist für $x< 1$ echt größer als 1.} Setzen
    wir nun $r := \frac{1}{\gamma} < 1 = R$ und $f :=
    \frac{h}{\gamma}$ haben wir die gewünschte Funktion gefunden.
  \item[Fall 2: $R = \infty$:] Nach Voraussetzung existiert ein $a \in
    \C \setminus Y$. Wir betrachten
    \[
    \phi: \C \ra \C, \quad \phi(z) = z -a
    \]
    Nun gilt aber $0 \notin \phi(Y)$. Nach Lemma \ref{lemma:ex-log}
    existiert ein $g \in \hol(Y)$ mit $g^2 = \phi|_Y$. Völlig analog
    zum Beweis von Satz \ref{thm:schlicht-kompakt} erhalten wir für $w
    \in g(Y)$, dass $-w \notin g(Y)$. \\
    $g$ ist nicht konstant und damit eine offene Abbbildung. Sei
    $b \in g(Y)$. Dann existiert ein $\epsilon > 0$,
    s.d. $B_{\epsilon}(b) \subseteq g(Y)$. Da nun für $w \in g(Y)$ $-w
    \notin g(Y)$, gilt insbesondere $g \notin B_\epsilon(b)$. Also
    gilt:
    \[
    |-w - b| = |w + b| \geq r
    \]
    D.h. wir können
    \[
    \psi: g(Y) \ra \C, \quad \psi(w) = \frac{1}{w+b}
    \]
    definieren. Dann ist $\psi$ holomorph und es gilt $|\psi(w)| \leq
    \frac{1}{\epsilon}$ für jedes $w \in g(Y)$. \\
    Damit können wir eine holomorphe Abbildung $\tilde \psi: Y \ra B$
    finden. Setzen wir nun noch $c:= \tilde \psi(0)$ und betrachten
    \[
    \eta: B \ra B, \quad \eta(z) = \frac{z-c}{1-\bar c z}
    \]
    so können wir $\tilde f := \eta \circ \tilde \psi: Y \ra D$
    definieren mit den Eigenschaften: $\tilde f(0) = 0$ und
    \begin{align*}
      \tilde f'(0) & = \eta'(c) \cdot \tilde \psi'(0) \\
      & = \frac{1}{1- |c|^2} \cdot r \cdot \psi'(g(0)) \cdot g'(0) \\
      & = \frac{r}{1 - |c|^2} \cdot \frac{-1}{(g(0) + b)^2} \cdot
      \frac{\phi'(0)}{2 g(0)} \\
      & = \frac{r}{1- |c|^2} \cdot \frac{-1}{(g(0) + b)^2} \cdot
      \frac{1}{2 g(0)} \\
      & \neq 0
    \end{align*}
    Das heißt nach dem gleichen Skalierungsargument, wie im ersten
    Fall erhalten wir eine holomorphe Abbildung $f: Y \ra B_r(0)$, wobei
    $r < R = \infty$.
  \end{description}
\end{proof}

\begin{lemma}
  \label{lemma:runge-keine-kohomo}
  Sei $X$ eine nicht kompakte Riemannsche Fläche, $Rh_\hol^1(X) = 0$,
  $Y \subseteq X$ ein Runge-Gebiet.\\
  Dann gilt $Rh_\hol^1(Y) = 0$.
\end{lemma}

\begin{proof}
  Sei $\omega \in \Omega(Y)$ nach Korollar
  \ref{cor:nicht-verschwindende-1-form} existiert ein $w_0 \in
  \Omega(X)$ ohne Nullstellen. Daher
  existiert ein $f \in \hol(Y)$ mit $\omega = f \omega_0$. \\
  Der Runge Approximationssatz besagt nun, dass eine Folge $(f_n)_{n
    \in \N} \subseteq \hol(X)$ existiert, so dass $f_n \ra f$ kompakt
  auf $Y$ konvergiert. \\
  Sei nun $\alpha$ eine beliebige geschlossene Kurve in $Y$. 
  % TODO: Kurven immer stetig.
  Dann gilt:
  \[
  0 = \int_\alpha f_n \omega_0 \xrightarrow{n \ra \infty} \int_\alpha
  \omega
  \]
  Also verschwindet $\int_\alpha \omega$ für jede geschlossene
  Kurve. Demnach besitzt $\omega$ nach (10.15) eine Stammfunktion und
  damit verschwindet $Rh_\hol^1(Y)$.
\end{proof}


\begin{thm}[Der Riemannsche Abbildungssatz]
  Sei $X$ eine Riemannsche Fläche mit $Rh_\hol^1(X) = 0$.\\
  Dann ist $X$ entweder konform äquivalent
  \begin{enumerate}
  \item zur Riemannschen Zahlenkugel $\P^1$,
  \item zur Komplexen Ebene $\C$ oder
  \item zur Einheitskreisscheibe $B$.
  \end{enumerate}
\end{thm}

\begin{proof}
  \begin{description}
  \item[Fall 1: $X$ ist kompakt:] Dann wissen wir zunächst, dass $\d
    \hol(X) = 0$ und damit auch $\Omega(X) = 0$. Nun liefert uns der
    Serresche Dualitätssatz, dass $H^0(X, \Omega) \cong H^1(X,
    \hol)^\ast$. Da nun immer $\Omega(x) \cong H^0(X, \Omega)$ gilt
    und im kompakten Fall $H^1(X, \hol)$ ein endlich dimensionaler
    Vektorraum ist, erhalten wir
    \[
    H^1(X, \hol) \cong H^1(X, \hol)^\ast = 0
    \]
    Also ist das Geschlecht unserer kompakten Riemanschen Fläche 0 und
    aus dem Satz von Riemann-Roch erhalten wir, dass $X$ konform
    äquivalent zu $\P^1$ ist.
  \item[Fall 2: $X$ ist nicht kompakt:] Nach (23.9) existiert dann
    eine Ausschöpfung $Y_0 \Subset Y_1 \Subset Y_2 \Subset \dots $ von
    $X$, wobei die $Y_i$ Rungegebiete mit regulärem Rand.\\
    Nach Lemma \ref{lemma:runge-keine-kohomo} gilt für jedes $i \in \N$
    $Rh_\hol^1(Y_i) = 0$.
    Aus Satz \ref{thm:Gebiet-Kreis} erhalten wir, dass die $Y_i$
    konform äquivalent zu $B$ sind. \\
    Sei $a \in Y_0$ und $(U,z)$ eine Koordinatenumgebung von $a$. Dann
    existieren $r_n <0$ und eine biholomorphe Abbildung $f_n: Y_n \ra
    B_{r_n}(0)$ mit $f_n(a) = 0$ und $\frac{\d[f_n]}{\d[z]}(a) = 1$, denn
    betrachten wir die Abbildung $\tilde f_n : Y_n \ra B$ die nach
    Satz \ref{thm:Gebiet-Kreis} existiert und die Abbildung
    \[
    \phi: B \ra B, \quad\phi_n(z) := \frac{z- \tilde f_n(a)}{1 - z \overline{\tilde
        f_n(a)}},
    \]
    so erhalten wir zunächts $\bar f_n := \phi_n \circ \tilde
    f_n$. Diese erfüllt $\bar f_n : Y_n \ra B$ und $\bar f_n(a) =
    0$. Setzen wir nun $r_n^{-1} := \left | \frac{\d[\bar f(a)]}{\d[z]}
    \right |$ und $f_n := r_n \cdot \bar f_n : Y_n \ra B_{r_n}(0)$, so
    haben wir die gewünschte Abbildung gefunden. \\
    Weiterhin gilt $r_n \leq r_{n+1}$ für jedes $n \in \N$, denn $h:
    f_{n+1} \circ f_n^{-1} : B_{r_n}(0) \ra B_{r_{n+1}}(0)$ erfüllt $h(0)=0$
    und $h'(0) = 1$, also $1 = |h'(0)| \leq \frac{r_{n+1}}{r_n}$ nach
    Proposition \ref{prop:kreis}. \\
    Wir setzen nun
    \[
    R := \lim_{n \ra \infty} r_n \in ]0, \infty]
    \]
    Behauptung: $X$ wird biholomorph auf $B_R(0)$ abgebildet. Dazu
    betrachten wir die Funktionenfolge $(f_n)_{n \in \N}$. Wir
    behaupten, dass eine Teilfolge $(f_{n_k})_{k \in \N}$ existiert,
    so dass für beliebige $m \in \N$ $(f_{n_k}|_{Y_m})_{k \geq m}$
    kompakt auf $Y_m$ konvergiert. \\
    Dazu definieren wir $g_n(z) := \frac{1}{r_0} f_n(f_0^{-1}(r_0 z))$
    für alle $n \geq 0$. Damit sind die $g_n : B \ra \C$ injektive
    holomorphe Funktionen mit $g_n(0) = 0$ und $g_n'(0) = 1$, also
    $g_n \in \S$ und aus Satz \ref{thm:schlicht-kompakt}
    erhalten wir eine konvergente Teilfolge $(g_{n_k})_{k \in\N}$.
    Diese Teilfolge gibt uns direkt eine konvergente Teilfolge
    $(f_{n_{0k}})_{k\in\N}$.
    Wenden wir die gleiche Konstruktion nun auf
    $\frac{1}{r_1}f_{n_{0k}}(f_1^{-1}(r_1z))$ erahlten wir eine
    Teilfolge $f_{n_{1k}}$. Induktiv erhalten wir so zu jedem $Y_m$
    eine konvergente Teilfolge $(f_{n_{mk}})$. \\
    Setzen wir $f_{n_k} := f_{n_{kk}}$, so erhalten wir die gewünschte
    Teilfolge.
    Sei $f \in \hol(X)$ der Grenzwert von $(f_{n_k})_{k \in \N}$. Dann
    können wir analog zum Beweis von Satz \ref{thm:schlicht-kompakt}
    zeigen, dass $f$ injektiv ist und $f(a) = 0$ und
    $\frac{\d[f(a)]}{\d[z]} = 1$ genügt.\\
    Behauptung: $f$ bildet $X$ biholomorph auf $B_R(0)$ ab.  \\
    Zunächst folgt aus $|f_{n_k}(x)| < r_{n_k}$ für beliebige $x \in X$
    und $k \in \N$ groß genug, dass $|f(x)| \leq R$. Da $f$
    holomorph und nicht-konstant, also eine offene Abbildung ist erhalten wir direkt,
    dass $f(x) \subset B_R(0)$. Ansonsten gäbe es ein $z \in f(x)$ mit
    $z \in \partial B_R(0)$. Dieses $z$ kann aber kein innerer Punkt von
    $f(B)$ sein. Ein Widerspruch. \\
    Es bleibt also nur noch die Surjektivität zu zeigen. \\
    Angenommen $f$ wäre nicht surjektiv. Dann existiert nach Lemma
    \ref{lemma:bihol-kreis} ein $0 < r < R$ und eine holomorphe
    Abbildung $g: f(X) \ra B_r(0)$ mit $g(0) = 0$ und $g'(0) = 1$. \\
    Sei nun $n$ so groß gewählt, dass $r_n > r$. Dann gilt für
    \[
    h:= g \circ f \circ f^{-1}_n : B_{r_n}(0) \ra B_r(0)
    \]
    $h(0)=0$, $h'(0) = 1$, aber $r < r_n$. Dies ist ein Widerspruch zu
    Proposition \ref{prop:kreis}. \\
    Also ist $X$ konform äquivalent zu $D_R(0)$. Für $R = \infty$ also
    zu $\C$ und für $R < \infty$ zu $B$.
  \end{description}
\end{proof}

\begin{rem}
  Wir wissen bereits, dass jede einfachzusammenhängende Riemannsche
  Fläche eine verschwindende deRham-Gruppe hat und dementsprechend
  konform äquivalent zu $\P^1$, $\C$ oder $B$ ist, andererseits zeigt
  der Riemannsche Abbildungssatz, dass jede Riemannsche Fläche mit
  verschwindender deRham-Gruppe einfach zusammenhängend ist. Also
  haben wir unser eigentliche Resultat bewiesen, nämlich, dass jede
  einfachzusammenhängende Riemansche Fläche konform äquivalent zu
  einer der obigen drei ist.
\end{rem}

%%% Local Variables: 
%%% mode: latex
%%% TeX-master: "../Bachelor"
%%% End: 
