
\section{Abzählbare Topologie}
\label{sec:Topologie}

\begin{lemma}
  \label{lemma:surj-offen-top}
  Seien $X,Y$ topologische Räume und $f: X \ra Y$ stet, offen und
  surjektiv. \\
  Dann besitzt $Y$ eine abzählbare Basis, falls $X$ eine besitzt.
\end{lemma}

\begin{proof}
  Sei $\fu$ eine abzählbare Basis der Topologie von $X$ und
  \[
  \be := \{ f(U) : U \in \fu \}
  \]
  eine abzählbare Familie von offenen Teilmengen von $Y$. \\
  Behauptung: $\be$ ist eine Basis der Topologie von $Y$. \\
  Um dies zu beweisen wir $D \subset Y$ offen und $y \in D$ und
  müssen nun zeigen, dass ein $V \in \be$ existiert mit $y \in V
  \subset D$. \\
  Aus der Surjektivität von $f$ erhalten wir ein $x \in X$ mit $f(x) =
  y$. Weiterhin ist $f^{-1}(D)$ offen und eine Umgebung von $x$, da
  $f$ stetig ist. \\
  Da $\fu$ eine Basis der Topologie von $X$ ist existiert ein $U \in
  \fu$ mit $x \in U \subset f^{-1}(D)$. Also genügt $V:=f(U)$ den
  geforderten Eigenschaften.
\end{proof}

\begin{lemma}[Poincar\'e-Volterra]
  \label{lemma:poincare-volterra}
  Sei $X$ eine zusammenhängende Mannigfaltigkeit und $Y$ ein
  Hausdorff-Raum mit abzählbarer Basis der Topologie. Sei weiterhin
  $f: X \ra Y$ stetig und diskret. \\
  Dann besitzt $X$ eine abzählbare Basis der Topologie.
\end{lemma}

\begin{proof}
  Sei $\fu$ eine abzählbare Basis der Topologie von $Y$. \\
  Bezeichne mit $\be$ die Menge aller offenen Teilmengen $V \subset X$
  mit
  \begin{enumerate}
  \item $V$ besitzt eine abzählbare Basis der Topologie
  \item $V$ ist eine Zusammenhangskomponente eines $f^{-1}(U)$ für ein
    $U \in \fu$.
  \end{enumerate}
  \begin{enumerate}
  \item Behauptung: $\be$ ist eine Basis der Topologie von $X$. \\
    Sei $D \subset X$ offen mit $x \in D$. Zu zeigen ist, dass ein $V
    \in \be$ existiert, so dass $x \in V \subset D$. \\
    Da $f$ diskret ist existiert ein $W \subset D$ offen und relativ
    kompakt, $x \in W$ und $\partial W \cap f^{-1}(f(x)) =
    \varnothing$. Dies lässt sich einfach auf einer Karte um $x$
    verifizieren. \\
    Nun ist $f(\partial w)$ kompakt, also abgeschlossen und $f(x)
    \notin f(\partial W)$, d.h. $f(x)$ liegt in der offenen Menge $\C
    \setminus f(\partial W)$, also existiert ein $U \in \fu$, so dass
    $f(x) \in U$ und $U \cap f(\partial W) = \varnothing$. Sei  $V$
    die Zusammenhangskomponente von $f^{-1}(U)$, die $x$ enthält. Dann
    gilt $V \cap \partial W = \varnothing$ und da $V$ zusammenhängend
    ist folgt $V \subset W$. Also besitzt $V$ eine abzählbare Basis
    der Topologie als Teilmenge einer relativ kompakten Menge und
    damit $V \in be$.
  \item Behauptung: Jedes $V_0 \in \be$ schneidet höchstens abzählbar
    viele $V \in \be$. \\
    Sei $U \in \fu$, dann sind per Definition die
    Zusammenhangskomponenten von $f^{-1}(U)$ disjunkt und da $V_0$
    eine abzählbare Basis der Topologie besitzt, kann $V_0$ höchstens
    abzählbarviele Zusammenhangskomponenten von $f^{-1}(U)$
    treffen. \\
    Da es weiterhin nur abzählbar viele $f^{-1}(U)$ gibt, folgt, dass
    $V_0$ höchstens abzählbar viele $V \in \be$ schneidet.
  \item Behauptung: $\be$ ist abzählbar.\\
    Fixiere dazu $V^\ast \in \be$ und definiere für jedes $n \in \N$
    $\be_n \subset \be$ wie folgt: \\
    $\be_n$ besteht asu allen $V \in \be$, so dass $V_0, \dots, V_n
    \in \be$ existieren mit $V_0 = V^\ast$, $V_n = V$ und $V_{k-1}
    \cap V_k \neq \varnothing$ für $k = 1, \dots, n$. \\
    Da $X$ zusammenhängend ist folgt $\bigcup_{n \in \N} \be_n =
    \be$, denn für ein $V \in \be$ wählen wir ein $y \in V$ und ein $x
    \in V^\ast$. Dann existiert eine Kurve $c: [0,1] \ra X$ mit $c(0)
    = x$ und $c(1) = y$. Weiterhin ist $c([0,1])$ kompakt und damit
    finden wir eine endliche offene Überdeckung $V_0, \dots, V_n$ von
    $c([0,1])$, wobei $V_o = V^\ast$ und $V_n = V$ ist. Weiterhin
    können wir nach einer Umsortierung ohne Einschränkung annehmen,
    dass $V_{i-1} \cap V_{i} \neq \varnothing$ für alle $i= 1, \dots,
    n$ gilt. Also liegt $V \in \be_n$.
    Damit reduziert sich unser Problem darauf zu zeigen, dass $\be_n$
    für jedes $n \in \N$ abzählbar ist; dies tun wir induktiv. \\
    Klarerweise ist $\be_0 = \{V^\ast\}$ abzählbar. Angenommen $\be_n$
    ist abzählbar. $\be_{n+1}$ besteht dann aus allen $V \in \be$, so
    dass ein $\tilde V \in \be_n$ existiert mit $V \cap \tilde V \neq
    \varnothing$. Nach 2. existieren zu jedem $\tilde V \in be_n$ nur
    abzählbar viele $V$, die nicht-leeren Schnitt damit besitzen. \\
    Da $\be_n$ nun abzählbar ist und pro $\tilde V$ nur abzählbar
    viele $V$ existieren, ist auch $\be_{n+1}$ abzählbar.
  \end{enumerate}
  Damit besitzt $X$ eine abzählbare Basis der Topologie.
\end{proof}

\begin{thm}[Rad\'o]
  Jede Riemannsche Fläche $X$ besitzt eine abzählbare Basis der
  Topologie.
  \label{thm:rado}
\end{thm}

\begin{proof}
  Sei $U$ eine Koordinatenumgebung von $X$. Wähle $K_0, K_1 \subset
  U$, so dass $z(K_0)$ und $z(K_1)$ disjunkte, kompakte Kreisscheiben
  sind und setze $Y := X \setminus (K_0 \cup K_1)$. \\
  Nun genügt $\partial Y = \partial K_0 \cap \partial K_1$ dem
  Regulariätskriterium von Satz \ref{thm:reg-rand}. Also existiert
  nacht Satz \ref{thm:dirichlet} eine stetige
  Funktion $u: \bar Y \ra \R$, die harmonisch auf $Y$ ist und
  $u|_{\partial K_0} \equiv 0$ und $u|_{\partial K_1} \equiv 1$
  genügt. \\
  Da $u$ aufgrund der Randwerte nicht konstant sein kan erahlten wir
  eine nicht-triviale holomorphe 1-Form $\omega := d' U$ auf $Y$. \\
  Nach \cite[Kor. 10.6]{For} existiert eine holomorphe Stammfunktion
  $f$ von $p^\ast\omega$ auf der universellen Überlagerung $p: \tilde
  Y \ra Y$.\\
  Da $f$ nicht konstant ist, genügt $f$ den Voraussetzungen von
  Lemma \ref{lemma:poincare-volterra} (denn $\C$ ist hausdorffsch
  und besitzt eine abzählbare Basis der Topologie). Also besitzt auch
  $\tilde y$ eine abzählbare Basis der Topologie und aus
  Lemma \ref{lemma:surj-offen-top} folgt, dass $Y$ eine abzählbare
  Basis der Topologie. \\
  Nun ist $X = Y \cup U$ und $U$ ist homöomorph zu einer offenen
  Teilmenge in $\C$, besitzt also auch eine abzählbare Basis der
  Topologie. \\
  Insgesamt ergibt sich also, dass auch $X$ eine abzählbare Basis der
  Topologie besitzt.
\end{proof}

\begin{defin}
  Sei $X$ eine Riemannsche Fläche. Für alle $Y \subset X$ definieren
  wir $\runge(Y)$ durch $Y$ vereinigt mit allen relativ kompakten
  Zusammenhangskomponenten von $X \setminus Y$. \\
  $Y \subset X$ heißt \init{Runge-Teilmenge} falls $Y = \runge(Y)$. \\
  Es gilt:
  \begin{enumerate}
  \item $\runge(\runge(Y)) = \runge(Y) \quad \forall Y \subset X$ \todo{ausrechnen}
  \item $Y_1 \subset Y_2$ impliziert $\runge(Y_1) \subset \runge(Y_2)$
  \end{enumerate}
\end{defin}

\begin{thm}
  Sei $Y \subset X$ und $X$ eine Riemannsche Fläche. Dann gilt:
  \begin{enumerate}
  \item $Y$ abgeschlossen $\Ra$ $\runge(Y)$ abgeschlossen
  \item $Y$ kompakt $\Ra$ $\runge(Y)$ kompakt
  \end{enumerate}
\end{thm}

\begin{proof}
  \begin{enumerate}
  \item Seien $C_j, j \in J$ die Zusammenhangskomponenten von $X
    \setminus Y$ Da $X \setminus Y$ ist offen und $X$ ist eine
    Mannigfaltigkeit, deshalb sind alle $C_j$ offen. \\
    Sei $J_0 \subset J$, die Menge der $j \in J$, so dass $C_j$
    relativ kompakt ist. Dann erhalten wir
    \[
    X \setminus \runge(Y) = \bigcup_{j \in J\setminus J_0} C_j
    \]
    und ist damit offen, als Vereinigung offener Mengen und folglich
    ist $\runge(Y)$ abgeschlossen.
  \item Wir können ohne Einschränkung $Y \neq \varnothing$
    annehmen. Sei $U$ eine offene, relativ kompakte Umgebung von
    $Y$. Diese existiert, denn wir können $Y$ durch endlich viele
    relativ kompakte Koordinatenumgebungen überdecken. Die Vereinigung
    dieser Koordinatenumgebungen ergibt dann unser $U$. Sei weiterhin
    $C_j, j\in J$ wie oben.
    \begin{enumerate}
    \item Behauptung: Jedes $C_j$ trifft $\bar U$. \\
      Ansonsten wäre $C_j \subset X \setminus \bar U$, also $\bar C_j
      \subset X \setminus U \setminus X \setminus Y$. Da $C_j$ eine
      Zusammenhangskomponente von $X \setminus Y$ ist, folgt $C_j =
      \bar C_j$. Also ist $C_j$ offen und abgeschlossen, also $C_j =
      \varnothing$ oder $C_j = X$. Beides führt zum Widerspruch
    \item Behauptung: Nur endlich viele $C_j$ treffen $\partial U$. \\
      Dies folgt daraus, dass $\partial U$ kompakt ist und durch die
      disjunkten $C_j$ überdeckt wird.
    \end{enumerate}
    Seien nun wieder $C_j, j \in J_0$ die relativ kompakten
    Zusammenhangskomponente von $X \setminus Y$ und $C_{j_1}, \dots,
    C_{j_n}$ diejenigen, die $\partial U$ schneiden. Nach a)
    liegen alle anderen relativ kompakten Zusammenhangskomponenten in
    $U$. Also ist
    \[
    \runge(Y) \subset U \cup C_{j_1} \cup \dots \cup C_{j_n}
    \]
    und damit relativ kompakt. Da $\runge(Y)$ nach 1. aber auch
    abgeschlossen ist, ist es tatsächlich kompakt
  \end{enumerate}
\end{proof}


\begin{thm}
  \label{thm:kompakte-ausschöpfung}
  Sei $X$ eine Riemannsche Fläche. Dann existiert eine Folge kompakter Mengen
  $(K_n)_{n \in \N}$, die folgende Eigenschaften erfüllt:
  \begin{enumerate}
  \item Es gilt $K_n \subset \mathring K_{n+1}$ für jedes $n \in \N$ und
  \item $X = \bigcup_{n \in \N}$.
  \end{enumerate}
\end{thm}

\begin{proof}
  Nach Satz \ref{thm:rado} besitzt $X$ eine abzählbare Basis der
  Topologie. Wir bezeichnen Diese mit $\be$. Dann definieren wir
  \[
  \fu := \{ B \in \be \mid \bar B \text{ ist kompakt}\}.
  \]
  Wir behaupten nun, dass bereits $\fu$ eine Basis der Topologie von
  $X$ ist. Sei dazu $\varnothing \neq U \subset X$ und $x \in U$. Dann
  können wir eine relativ kompakte Koordinatenumgebung $V \subset U$
  von $x$ finden. Da $\be$ eine Basis der Topologie ist, finden wir
  ein $B \in \be$, so dass $x \in B \subset V$. Also ist $\bar B
  \subset \bar V$ kompakt und damit liegt $B \in \fu$ und es gilt $x
  \in B \subset U$. Also ist $\fu$ eine weiter abzählbare Basis der
  Topologie von $X$.

  Wir wählen nun eine Aufzählung für $\fu = \{B_i \mid i \in \N\}$ und
  setzen $K_1 := \bar B_1$. Wir konstruieren nun induktiv die
  gewünschte Folge von kompakten Teilmengen. $K_1$ wurde bereits
  konstruiert und wir setzen $k_1 = 1$. Gehen wir davon aus, dass auch
  $K_n$ und $k_n$ bereits
  konstruiert wurden, so gilt $K_n \subset \bigcup_{i=1}^\infty
  B_i$. Da $K_n$ kompakt ist, existiert ein $k_{n+1} \in \N$, so dass $K_n
  \subset \bigcup_{i=1}^{k_{n+1}} B_i$. Setzen wir $K_{n+1} :=
  \bigcup_{i=1}^{k_{n+1}}\bar B_i$, so gilt $K_n \subset \mathring
  K_{n+1}$ und weiterhin ist $K_{n+1}$ kompakt. Weiterhin gilt
  $k_{n+1} > k_n$.

  Es bleibt nun nur noch zu zeigen, dass $X = \bigcup_{n \in \N} K_n$
  gilt. Sei dazu $x \in X$ beliebig gewählt. Dann existiert ein $B_j \in
  \fu$, so dass $x \in B_j$. Da $(k_n)_{n \in\N}$ streng monoton
  wachsend ist, existiert ein $N \in \N$, so dass $x \in K_N =
  \bigcup_{i=1}^{k_N} \bar B_i$ gilt.
\end{proof}

\begin{lemma}
  \label{lemma:kompakt-in-ausschöpfung}
  Sei $X$ eine Riemannsche Fläche und $(K_n)_{n \in \N}$ eine kompakte
  Ausschöpfung von $X$ mit $K_n \subset \mathring K_{n+1}$ für alle $n
  \in \N$. Sei weiterhin $K \subset X$ kompakt. Dann existiert ein $n
  \in \N$, so dass $K \subset K_n$.
\end{lemma}

\begin{proof}
  Angenommen es gälte $K \not \subset K_n$ für jedes $n \in \N$. Dann
  fänden wir zu jedem $n \in \N$ ein $x_n \in K\setminus K_n$. Da $K$
  kompakt ist, existierte eine konvergente Teilfolge $(x_{n_k})_{k\in
    \N}$ mit Grenzwert $x \in K$. Nun existiert aber ein $n \in \N$,
  so dass $x \in \mathring K_n$ gilt. Damit wäre $\mathring K_n$ eine offene
  Umgebung von $x$ und wir fänden ein $K \in \N$, so dass $x_{n_k} \in
  K_n$ für jedes $k\geq K$ gälte. Das aber bedeutete, dass eine
  zweites $\tilde K \in \N$ existieren müsste, so dass $x_{n_k} \in K_{n_k}$
  für alle $k \geq \tilde K$ gälte. Dies ist aber ein Widerspruch dazu,
  dass $x_n \in K \setminus K_n$ gilt.
\end{proof}


\begin{lemma}
  \label{lemma:komp-enthalten}
  Sei $X$ eine Riemannsche Fläche und $K \subset X$ kompakt. Dann
  existiert ein offenes $M \Subset X$ mit $K \subset M$.
\end{lemma}

\begin{proof}
  Zu jedem $z \in K$ finden wir eine relativ kompakte
  Koordinatenumgebung $U_z$. Dann ist aber $K \subset \bigcup_{z \in
    K}U_z$ und aus der Kompaktheit von $K$ folgt, dass endlich viele
  $z_1, \dots, z_n \in K$ existieren, so dass $K \subset
  \bigcup_{i=1}^n U_{z_i}$. Setzen wir $M := \bigcup_{i=1}^n U_{z_i}$,
  so ist $M$ offen und relativ kompakt, weiterhin gilt $K \subset M$,
  was die Behauptung zeigt.
\end{proof}

\begin{cor}
  \label{cor:ausschöpfung-kompakt}
  Sei $X$ eine nicht kompakte Riemannsche Fläche. \\
  Dann existieren kompakte Runge-Teilmenge $K_j \subset X, j \in \N$ mit
  \begin{enumerate}
  \item $K_{j-1} \subset \mathring{K_j} \quad \forall j \geq 1$
  \item $\bigcup_{j\in \N} K_j = X$
  \end{enumerate}
\end{cor}

\begin{proof}
  Nach Sat \ref{thm:kompakte-ausschöpfung} existiert eine 
  Ausschöpfung
  \[
  K_0' \subset K_1' \subset \dots
  \]
  durch Kompakta von $X$. \\
  Setze nun $K_0 := \runge(K_0')$ \\
  Seien $K_0, \dots, K_n$ bereits konstruiert. Dann existiert ein
  kompaktes $M \subset X$ mit $K_n' \cup K_n \subset \mathring
  M$. Dieses $M$ erhalten wir, in dem wir $K_n'$ und $K_n$ durch
  endlich viele, relativ kompakte Koordinatenumgebungen
  überdecken. $M$ ist dann deren Vereinigung. Anschließend setzen wir
  $K_{n+1} := \runge(M)$.
\end{proof}

\begin{lemma}
  \label{lemma:zwischen-runge}
  Seien $K_1, K_2 \subset X$ kompakte Teilmengen und $X$ eine
  Riemannsche Fläche mit $K_1 \subset \mathring{K_2}$ und $\runge(K_2)
  = K_2$. \\
  Dann existiert eine offene Runge-Teilmenge $Y \subset X$ mit $K_1
  \subset Y \subset K_2$. \\
  Weiterhin kann $Y$ so gewählt werden, dass sein Rand regulär ist.
\end{lemma}

\begin{proof}
  Zu jedem $x \in \partial K_2$ existiert eine Koordinatenumgebung $U$
  von $x$, so dass $K_1 \cap U = \varnothing$. Wähle eine kompakte
  Scheibe $D$, die $x$ im Inneren enthält. Da $\partial K_2$ auch
  kompakt ist, wird es durch endlich viele $D_1, \dots, D_k$
  überdeckt. Setze $Y := K_2 \setminus (D_1 \cup \dots \cup D_k)$. \\
  Dann ist $Y$ offen und liegt zwischen $K_1$ und $K_2$. \\
  Seien $c_j, j \in J$ die Zusammenhangskomponenten von $X \setminus
  K_2$. Nach Voraussetzund sind diese nicht relativ kompakt. Nun
  treffen alle $D_i$ mindestens ein $C_j$, da $D_i \cap (Y \setminus
  K_2) \neq \varnothing$ sein muss. DA alle $D_i$ zusammenhängend
  sind, sind die $D_i \cap C_j$ zusammenhängend un nicht relativ
  kompakt. Also existieren keine relativ kompakten
  Zusammenhangskomponenten in $X\setminus Y$. Also $Y = \runge(Y)$. \\
  Weiterhin sind nach (22.18) alle Randpunkte von $Y$ regulär.
\end{proof}

\begin{lemma}
  \label{lemma:zsh-komp}
  Sei $X$ eine Riemannsche Fläche, $Y \subset X$ eine offene Menge und
  $Z \subset Y$ eine Zusammenhangskomponente von $Y$. Dann ist $Z$ offen.
\end{lemma}

\begin{proof}
  Sei $x \in Z$. Dann finden wir eine offene, zusammenhängende
  Koordinatenumgebung $U \subset Y$ mit mit $x \in U$. Nun ist aber
  $Z$ die maximale zusammenhängende Teilmenge, die $x$ enthält und da
  $U$ auch zusammenhängend gewählt wurde, muss $U \subset Z$
  gelten. Also ist $x$ ein innerer Punkt und damit $Z$ offen.
\end{proof}

\begin{thm}
  \label{thm:runge-zshkomp}
  Sei $Y \subset X$ eine offene Runge-Teilmenge und $X$ eine
  Riemannsche Fläche. \\
  Dann ist jede Zusammenhangskomponente runge.
\end{thm}

\begin{proof}
  \begin{enumerate}
  \item Seien $Y_i, i \in I$, die Zusammenhangskomponenten von
    $Y$. Nach Lemma \ref{lemma:zsh-komp} sind alle $Y_i$ offen.
    Setze nun $A := X \setminus Y$ und $A_k, k\in \N$ die
    Zusammenhangskomponenten von $A$. Dann sind alle $A_k$
    abgeschlossen, aber nicht kompakt.
  \item Behauptung: Für jedes $i \in I$ gilt: $\bar Y_i \cap A \neq
    \varnothing$.\\
    Ansonsten wäre $\bar Y_i \subset Y$. Da $\bar Y_i \cap \bigcup_{j
      \neq i} Y_j = \varnothing$. Also ist $Y_i = \bar Y_i$. Dies
    ist ein Widerspruch dazu, dass $X$ zusammenhängend ist.
  \item Behauptung: $C \cap A \neq \varnothing$ für jede
    Zusammenhangskomponente $C$ von $X \setminus Y_i$. \\
    Ansonsten gäbe es ein $j \neq i$, so dass $C \cap Y_j \neq
    \varnothing$. \\
    Nun ist aber $Y_j \subset X \setminus Y_i$ zusammenhängend und aus
    dem nicht leeren Schnitt folgte $Y_j \subset C$, da $C$ maximal
    zusammenhängend ist. Weiterhin folgt dann auch $\bar Y_j \subset
    \bar C = C$, da $C$ bereits abgeschlossen ist. Nach 2. würde dies
    bedeuten, dass $A \cap C \neq \varnothing$ gelten müsste. Also
    genau, was wir zeigen wollten.
  \item Sei nun $C$ eine Zusammenhangskomponente von $X \setminus
    Y_i$. Dann trifft $C$ nach 3. mindestens ein $A_k$. Also ist $A_k
    \subset C$ und da $A_k$ nicht kompakt ist, ist $C$ auch nicht kompakt.
  \end{enumerate}
  Damit ist $Y_i$ runge.
\end{proof}

\begin{thm}
  \label{thm:Ausschöpfung-Runge}
  Sei $X$ eine nicht kompakte Riemannsche Fläche. \\
  Dann existiert eine Ausschöpfung $Y_0 \Subset Y_1 \Subset \dots$ von
  $X$ durch relativ kompakte Runge-Gebiete. Weiterhin hat jedes $Y_i$
  regulären Rand.
\end{thm}

\begin{proof}
  Sei $K \subset X$ kompakt. Da $K$ kompakt ist, kann es nur endlich
  viele Zusammenhangskomponenten besitzen, ansonsten könnten wir eine
  Folge ohne konvergente Teilfolgen konstruieren. Wählen wir nun aus
  jeder dieser endlich vielen Zusammenhangskomponenten einen Punkt
  aus, so können wir diese durch endlich viele Kurven
  verbinden. Vereinigen wir nun $K$ mit den Bildern der Kurven, die
  auch kompakt sind, so ist dies eine endliche Vereinigung und das
  Resultat ist wieder kompakt und nach Konstruktion
  zusammenhängend. Wir bezeichnen es mit $K_1$. Es folgt $K \subset
  K_1$. Nach \ref{lemma:komp-enthalten} finden wir ein kompaktes $K_2
  \subset X$ mit $K_1 \subset \mathring{K_2}$. Nach Lemma
  \ref{lemma:zwischen-runge}
  existiert eine offene Runge-Teilmenge $\tilde Y_1 \subset X$ mit
  $K_1 \subset \tilde Y_1 \subset \runge(K_2)$. mit regulärem Rand. \\
  Wählen wir nun die Zusammenhangskomponente $Y_1$, die $K_1$ enthält,
  so ist $Y_1$ nach Satz \label{thm:runge-zshkomp} ein Runge-Gebiet
  und hat nach Bem nach Satz \ref{thm:reg-rand} auch regulären Rand. \\
  Nun finden wir eine Ausschöpfung von $X$ durch Kompakta $K_1 \subset
  K_2 \subset \dots$. \\
  Durch das obige vorgehen erhalten wir ein $Y_1$, das $K_1$
  enthält.\\
  Seien nun $Y_1, \dots, Y_k$ bereits konstruiert. Setzen wir $\tilde
  K_k = K_k \cup \bar Y_k$. Dann ist $\tilde K_k$ kompakt und wir
  können wieder wie oben ein Runge-Gebiet $Y_k \supset \tilde K_k$
  finden.\\
  Damit ist die gewünschte Ausschöpfung konstruiert.
\end{proof}

%%% Local Variables: 
%%% mode: latex
%%% TeX-master: "../Bachelor"
%%% End: 
