
\section{Die Topologie Riemannscher Flächen}
\label{sec:Topologie}

Im Gegensatz zu reellen Mannigfaltigkeiten haben wir bei der
Definition einer Riemannschen Fläche nicht gefordert, dass diese eine
abzählbare Basis der Topologie besitzt. Dies ist tatsächlich nicht
erforderlich, sondern ergibt sich als Konsequenz aus der Komplexen
Struktur.

Der erste Teil des Kapitels gibt uns verschiedene Kriterien, um die
Abzählbarkeit einer Topologie zu zeigen und kulminiert im Satz von
Rad\'o (Satz \ref{thm:rado}), der zeigt, dass jede Riemannsche Fläche
eine abzählbare Basis der Topologie besitzt.

Im zweiten Teil des Kapitels werden einige Eigenschaften abzählbarer
topologischer Räume zusammengetragen. Wir verzichten jedoch auf volle
Allgemeinheit und formulieren die Aussagen nur für Riemannsche Flächen.
Weiterhin wird der Begriff der Runge-Teilmenge eingeführt, der eine
zentrale Rolle in der Formulierung des Rungeschen Approximationssatzes
spielt. Dieser wird in Kapitel \ref{sec:Runge} ausführlich behandelt.

\begin{lemma}
  \label{lemma:surj-offen-top}
  Seien $X,Y$ topologische Räume und $f: X \ra Y$ stetig, offen und
  surjektiv. Dann besitzt $Y$ eine abzählbare Basis, falls $X$ eine
  besitzt.
\end{lemma}

\begin{proof}
  Sei $\fu$ eine abzählbare Basis der Topologie von $X$ und
  \[
  \be := \{ f(U) : U \in \fu \}
  \]
  eine abzählbare Familie von offenen Teilmengen von $Y$. Wir
  behaupten, dass $\be$ eine Basis der Topologie von $Y$ ist. Um dies
  zu beweisen, sei $D \subset Y$ offen und $y \in D$. Wir
  müssen nun zeigen, dass ein $V \in \be$ existiert mit $y \in V
  \subset D$. Aus der Surjektivität von $f$ erhalten wir ein $x \in X$ mit $f(x) =
  y$. Weiterhin ist $f^{-1}(D)$ offen und eine Umgebung von $x$, da
  $f$ stetig ist. Nun ist $\fu$ eine Basis der Topologie von
  $X$. Demnach existiert ein $U \in \fu$ mit $x \in U \subset f^{-1}(D)$. Also
  genügt $V:=f(U)$ den geforderten Eigenschaften.
\end{proof}

\begin{lemma}[Poincar\'e-Volterra]
  \label{lemma:poincare-volterra}
  Sei $X$ eine zusammenhängende Mannigfaltigkeit und $Y$ ein
  Hausdorffraum mit abzählbarer Basis der Topologie. Sei weiterhin
  $f: X \ra Y$ stetig und diskret. Dann besitzt $X$ eine abzählbare
  Basis der Topologie.
\end{lemma}

\begin{proof}
  Sei $\fu$ eine abzählbare Basis der Topologie von $Y$. 
  Wir bezeichnen mit $\be$ die Menge aller offenen Teilmengen $V \subset X$
  mit
  \begin{enumerate}
  \item $V$ besitzt eine abzählbare Basis der Topologie und 
  \item $V$ ist eine Zusammenhangskomponente eines $f^{-1}(U)$ für ein
    $U \in \fu$.
  \end{enumerate}
  Zunächst zeigen wir, dass $\be$ eine Basis der Topologie von $X$ ist.
  Sei $D \subset X$ offen mit $x \in D$. Zu zeigen ist, dass ein $V
  \in \be$ existiert, so dass $x \in V \subset D$ gilt.
  Da $f$ diskret ist, existiert ein offenes, zusammenhängendes $W \Subset D$  mit $x \in W$ und
  \[
  \partial W \cap f^{-1}(f(x)) = \varnothing.
  \]
  Dies lässt sich einfach auf einer Karte um $x$
  verifizieren. 
  Nun ist $f(\partial W)$ kompakt, also abgeschlossen und $f(x)
  \notin f(\partial W)$, d.h. $f(x)$ liegt in der offenen Menge $\C
  \setminus f(\partial W)$, also existiert ein $U \in \fu$, so dass
  $f(x) \in U$ und $U \cap f(\partial W) = \varnothing$. Sei  $V$
  die Zusammenhangskomponente von $f^{-1}(U)$, die $x$ enthält. Dann
  gilt $V \cap \partial W = \varnothing$ und da $V$ zusammenhängend
  ist, folgt $V \subset W$. Also besitzt $V$ eine abzählbare Basis
  der Topologie als Teilmenge einer relativ kompakten Menge und
  damit folgt $V \in \be$.
  
  Als nächstes beweisen wir, dass jedes $V_0 \in \be$ höchstens abzählbar
  viele $V \in \be$ schneidet.
  Sei $U \in \fu$, dann sind per Definition die
  Zusammenhangskomponenten von $f^{-1}(U)$ disjunkt und da $V_0$
  eine abzählbare Basis der Topologie besitzt, kann $V_0$ höchstens
  abzählbar viele Zusammenhangskomponenten von $f^{-1}(U)$
  treffen. Da es weiterhin nur abzählbar viele $f^{-1}(U)$ gibt, folgt, dass
  $V_0$ höchstens abzählbar viele $V \in \be$ schneidet.
  
  Zu guter Letzt ist $\be$ abzählbar. 
  Wir fixieren dazu $V^\ast \in \be$ und bezeichnen für jedes $n \in \N$
  $\be_n \subset \be$ die Menge bestehend aus allen $V \in \be$, so
  dass $V_0, \dots, V_n \in \be$ existieren mit $V_0 = V^\ast$, $V_n = V$ und $V_{k-1}
  \cap V_k \neq \varnothing$ für $k = 1, \dots, n$. 
  Da $X$ zusammenhängend ist, folgt $\bigcup_{n \in \N} \be_n =
  \be$, denn für ein $V \in \be$ wählen wir ein $y \in V$ und ein $x
  \in V^\ast$. Dann existiert eine Kurve $c: [0,1] \ra X$ mit $c(0)
  = x$ und $c(1) = y$. Weiterhin ist $c([0,1])$ kompakt und damit
  finden wir eine endliche offene Überdeckung $V_0, \dots, V_n$ von
  $c([0,1])$, wobei $V_0 = V^\ast$ und $V_n = V$ ist. Weiterhin
  können wir nach einer Umsordnung ohne Einschränkung annehmen,
  dass $V_{i-1} \cap V_{i} \neq \varnothing$ für alle $i= 1, \dots,
  n$ gilt. Also liegt $V \in \be_n$.
  Damit reduziert sich unser Problem darauf zu zeigen, dass $\be_n$
  für jedes $n \in \N$ abzählbar ist. Wir gehen dabei induktiv vor.
  Klarerweise ist $\be_0 = \{V^\ast\}$ abzählbar. Angenommen $\be_n$
  ist abzählbar. $\be_{n+1}$ besteht dann aus allen $V \in \be$, so
  dass ein $\tilde V \in \be_n$ existiert mit $V \cap \tilde V \neq
  \varnothing$. Nach der obigen Ausführung trifft $\tilde V \in \be_n$
  nur abzählbar viele $V$. Da $\be_n$ nun abzählbar ist und pro
  $\tilde V$ nur abzählbar viele $V$ existieren, ist auch $\be_{n+1}$
  abzählbar. Damit besitzt $X$ eine abzählbare Basis der Topologie.
\end{proof}

\begin{thm}[Rad\'o]
  Jede Riemannsche Fläche $X$ besitzt eine abzählbare Basis der
  Topologie.
  \label{thm:rado}
\end{thm}

\begin{proof}
  Sei $(U,z)$ eine Karte von $X$. Wir wählen $K_0, K_1 \subset
  U$, so dass $z(K_0)$ und $z(K_1)$ disjunkte, kompakte Kreisscheiben
  sind und setzen $Y := X \setminus (K_0 \cup K_1)$. 
  Nun genügt $\partial Y = \partial K_0 \cap \partial K_1$ dem
  Regularitätskriterium von Satz \ref{thm:reg-rand}. Also existiert
  nach Satz \ref{thm:dirichlet} eine stetige
  Funktion $u: \bar Y \ra \R$, die harmonisch auf $Y$ ist und
  $u|_{\partial K_0} = 0$ und $u|_{\partial K_1} = 1$
  genügt. Da $u$ aufgrund der Randwerte nicht konstant sein kann, erhalten wir
  eine nicht-triviale holomorphe 1-Form $\omega := \d['u]$ auf $Y$. 
  Nach \cite[Kor. 10.6]{For} existiert eine holomorphe Stammfunktion
  $f$ von $p^\ast\omega$ auf der universellen Überlagerung $p: \tilde
  Y \ra Y$. Da $f$ nicht konstant ist, genügt $f$ den Voraussetzungen von
  Lemma \ref{lemma:poincare-volterra}\footnote{Denn $\C$ ist hausdorffsch
  und besitzt eine abzählbare Basis der Topologie.}. Also besitzt
  $\tilde Y$ eine abzählbare Basis der Topologie und durch
  Lemma \ref{lemma:surj-offen-top} überträgt sich dies direkt auf
  $Y$. Nun ist $X = Y \cup U$ und $U$ ist homöomorph
  zu einer offenen Teilmenge in $\C$, besitzt also auch eine
  abzählbare Basis der Topologie. 
  Insgesamt ergibt sich also, dass auch $X$ eine abzählbare Basis der
  Topologie besitzt.
\end{proof}

Nun wenden wir uns einigen zentralen Eigenschaften abzählbarer
topologischer Räume zu. Wie bereits erwähnt, werden wir die
Eigenschaften jedoch nur für Riemannsche Flächen formulieren.

\begin{lemma}
  \label{lemma:komp-enthalten}
  Sei $X$ eine Riemannsche Fläche und $K \subset X$ kompakt. Dann
  existiert ein offenes $M \Subset X$ mit $K \subset M$.
\end{lemma}

\begin{proof}
  Zu jedem $z \in K$ finden wir eine relativ kompakte
  Koordinatenumgebung $U_z$. Dann ist aber $K \subset \bigcup_{z \in
    K}U_z$ und aus der Kompaktheit von $K$ folgt, dass endlich viele
  $z_1, \dots, z_n \in K$ existieren, so dass $K \subset
  \bigcup_{i=1}^n U_{z_i}$ gilt. Setzen wir $M := \bigcup_{i=1}^n U_{z_i}$,
  so ist $M$ offen und relativ kompakt. Weiterhin gilt $K \subset M$,
  was die Behauptung zeigt.
\end{proof}

\begin{defin}
  Sei $X$ eine Riemannsche Fläche. Für alle $Y \subset X$ definieren
  wir $\runge(Y)$ durch $Y$ vereinigt mit allen relativ kompakten
  Zusammenhangskomponenten von $X \setminus Y$.

  $Y \subset X$ heißt \init{Runge-Teilmenge}, falls $Y = \runge(Y)$ ist. Es gilt:
  \begin{enumerate}
  \item $\runge(\runge(Y)) = \runge(Y) \quad \forall Y \subset X$
  \item $Y_1 \subset Y_2$ impliziert $\runge(Y_1) \subset
    \runge(Y_2)$, denn entweder $x \in Y_1$ liegt bereits in $Y_2$
    oder aber in einer relativ kompakten Zusammenhangskomponente $Z
    \subset X \setminus Y_1$. Da $x$ weiterhin nicht in $Y_2$ liegt,
    liegt $x \in Z \setminus Y_2 \subset Z$. Sei $\tilde Z \ni x$ eine
    Zusammenhangskomponente von $Z \setminus Y_2$, dann ist $\tilde Z$
    als Teilmenge von $Z$ relativ kompakt und $\tilde Z$ ist eine
    Zusammenhangskomponente von $X \setminus Y_2$, da alle
    Zusammenhangskomponenten von $X \setminus Y_2$ in den
    Zusammenhangskomponenten von $X \setminus Y_1$ enthalten sein
    müssen. Damit liegt aber $x$ in $\tilde Z \subset \runge(Y_2)$.
  \end{enumerate}
\end{defin}

\begin{thm}
  Sei $X$ eine Riemannsche Fläche und $Y \subset X$. Dann gilt:
  \begin{enumerate}
  \item $Y$ abgeschlossen $\Ra$ $\runge(Y)$ abgeschlossen
  \item $Y$ kompakt $\Ra$ $\runge(Y)$ kompakt
  \end{enumerate}
\end{thm}

\begin{proof}
  \begin{enumerate}
  \item Seien $C_j, j \in J,$ die Zusammenhangskomponenten von $X
    \setminus Y$. Nach Lemma \ref{lemma:zsh-komp} sind alle $C_j$ offen.
    Sei $J_0 \subset J$, die Menge der $j \in J$, so dass $C_j$
    relativ kompakt ist. Dann erhalten wir
    \[
    X \setminus \runge(Y) = \bigcup_{j \in J\setminus J_0} C_j.
    \]
    $X \setminus \runge(Y)$ ist damit offen, als Vereinigung offener Mengen und folglich
    ist $\runge(Y)$ abgeschlossen.
  \item Wir können ohne Einschränkung $Y \neq \varnothing$
    annehmen. Sei $U$ eine offene, relativ kompakte Umgebung von
    $Y$. Diese existiert nach Lemma \ref{lemma:komp-enthalten}.
    Sei weiterhin $C_j, j \in J,$ wie oben.

    Jedes $C_j$ trifft nun aber $\bar U$. Denn andernfalls wäre $C_j
    \subset X \setminus \bar U$, also
    \[
    \bar C_j \subset X \setminus U \subset X \setminus Y.
    \]
    Da $C_j$ eine
    Zusammenhangskomponente von $X \setminus Y$ ist, folgt $C_j =
    \bar C_j$. Also ist $C_j$ offen und abgeschlossen, also $C_j =
    \varnothing$ oder $C_j = X$. Beides führt zum Widerspruch.

    Nun behaupten wir, dass nur endlich viele $C_j$ den Rand $\partial
    U$ treffen. 
    Dies folgt bereits daraus, dass $\partial U$ kompakt ist und durch die
    disjunkten $C_j$ überdeckt wird.
    Seien nun wieder $C_j, j \in J_0,$ die relativ kompakten
    Zusammenhangskomponenten von $X \setminus Y$ und $C_{j_1}, \dots,
    C_{j_n}$ diejenigen, die $\partial U$ schneiden. Der Schnitt $C_j
    \cap \bar U$ ist aber für jedes $j$ nicht leer, also liegen alle
    anderen relativ kompakten Zusammenhangskomponenten in
    $U$. Es folgt
    \[
    \runge(Y) \subset U \cup C_{j_1} \cup \dots \cup C_{j_n}
    \]
    und damit ist $\runge(Y)$ relativ kompakt. Da $\runge(Y)$ nach 1. aber auch
    abgeschlossen ist, ist es bereits kompakt.
  \end{enumerate}
\end{proof}


\begin{thm}
  \label{thm:kompakte-ausschöpfung}
  Sei $X$ eine Riemannsche Fläche. Dann existiert eine Folge kompakter Mengen
  $(K_n)_{n \in \N}$, die die folgenden Eigenschaften erfüllt:
  \begin{enumerate}
  \item Es gilt $K_n \subset \mathring K_{n+1}$ für jedes $n \in \N$ und
  \item $X = \bigcup_{n \in \N} K_n$.
  \end{enumerate}
\end{thm}

\begin{proof}
  Nach Satz \ref{thm:rado} besitzt $X$ eine abzählbare Basis der
  Topologie. Wir bezeichnen diese mit $\be$. Dann definieren wir
  \[
  \fu := \{ B \in \be \mid \bar B \text{ ist kompakt}\}.
  \]
  Wir behaupten nun, dass bereits $\fu$ eine Basis der Topologie von
  $X$ ist. Sei dazu $\varnothing \neq U \subset X$ offen und $x \in U$. Dann
  können wir eine relativ kompakte Koordinatenumgebung $V \subset U$
  von $x$ finden. Da $\be$ eine Basis der Topologie ist, finden wir
  ein $B \in \be$, so dass $x \in B \subset V$. Also ist $\bar B
  \subset \bar V$ kompakt und damit liegt $B$ in $\fu$ und es gilt $x
  \in B \subset U$. Also ist $\fu$ eine weitere abzählbare Basis der
  Topologie von $X$.

  Wir wählen nun eine Aufzählung für $\fu = \{B_i \mid i \in \N\}$ und
  setzen $K_1 := \bar B_1$. Wir konstruieren nun induktiv die
  gewünschte Folge von kompakten Teilmengen. $K_1$ wurde bereits
  konstruiert und wir setzen $k_1 := 1$. Gehen wir davon aus, dass auch
  $K_n$ und $k_n$ bereits
  konstruiert wurden, so gilt $K_n \subset \bigcup_{i=1}^\infty
  B_i$. Da $K_n$ kompakt ist, existiert ein \break$k_{n+1} \in \N$, so dass $K_n
  \subset \bigcup_{i=1}^{k_{n+1}} B_i$. Setzen wir $K_{n+1} :=
  \bigcup_{i=1}^{k_{n+1}}\bar B_i$, so gilt $K_n \subset \mathring
  K_{n+1}$ und weiterhin ist $K_{n+1}$ kompakt. Außerdem gilt
  $k_{n+1} > k_n$.

  Es bleibt nun nur noch zu zeigen, dass $X = \bigcup_{n \in \N} K_n$
  gilt. Sei dazu $x \in X$ beliebig gewählt. Dann existiert ein $B_j \in
  \fu$, so dass $x \in B_j$ ist. Da $(k_n)_{n \in\N}$ streng monoton
  wachsend ist, existiert ein $N \in \N$, so dass $x \in K_N =
  \bigcup_{i=1}^{k_N} \bar B_i$ gilt.
\end{proof}

\begin{lemma}
  \label{lemma:kompakt-in-ausschöpfung}
  Sei $X$ eine Riemannsche Fläche und $(K_n)_{n \in \N}$ eine kompakte
  Ausschöpfung von $X$ mit $K_n \subset \mathring K_{n+1}$ für alle $n
  \in \N$. Sei weiterhin $K \subset X$ kompakt. Dann existiert ein $n
  \in \N$, so dass $K \subset K_n$ gilt.
\end{lemma}

\begin{proof}
  Angenommen es gälte $K \not \subset K_n$ für jedes $n \in \N$. Dann
  fänden wir zu jedem $n \in \N$ ein $x_n \in K\setminus K_n$. Da $K$
  kompakt ist, existierte eine konvergente Teilfolge $(x_{n_k})_{k\in
    \N}$ mit Grenzwert $x \in K$. Nun existiert aber ein $n \in \N$,
  so dass $x \in \mathring K_n$ gilt. Damit wäre $\mathring K_n$ eine offene
  Umgebung von $x$ und wir fänden ein $L \in \N$, so dass $x_{n_k} \in
  K_n$ für jedes $k\geq L$ gälte. Das aber bedeutete, dass eine
  zweites $\tilde L \in \N$ existieren müsste, so dass $x_{n_k} \in K_{n_k}$
  für alle $k \geq \tilde L$ gälte. Dies ist aber ein Widerspruch dazu,
  dass $x_n \in K \setminus K_n$ gilt.
\end{proof}


\begin{cor}
  \label{cor:ausschöpfung-kompakt}
  Sei $X$ eine nicht kompakte Riemannsche Fläche. Dann existieren
  kompakte Runge-Teilmenge $K_j \subset X, j \in \N,$ mit
  \begin{enumerate}
  \item $K_{j-1} \subset \mathring{K_j} \quad \forall j \geq 1$ und
  \item $\bigcup_{j\in \N} K_j = X$.
  \end{enumerate}
\end{cor}

\begin{proof}
  Nach Sat \ref{thm:kompakte-ausschöpfung} existiert eine 
  Ausschöpfung
  \[
  K_0' \subset K_1' \subset \dots
  \]
  durch Kompakta von $X$. Wir Setzen $K_0 := \runge(K_0')$. Seien
  $K_0, \dots, K_n$ bereits konstruiert. Dann existiert nach Lemma
  \ref{lemma:komp-enthalten} ein
  kompaktes $M \subset X$ mit $K_n' \cup K_n \subset \mathring
  M$. Wir setzen $K_{n+1} = \runge(M)$ und haben die gewünschte Folge
  gefunden.
\end{proof}

\begin{lemma}
  \label{lemma:zwischen-runge}
  Seien $X$ eine Riemannsche Fläche und $K_1, K_2 \subset X$ kompakte
  Teilmengen mit $K_1 \subset \mathring{K_2}$ und $\runge(K_2) =
  K_2$. Dann existiert eine offene Runge-Teilmenge $Y \subset X$ mit
  $K_1 \subset Y \subset K_2$.

  Weiterhin kann $Y$ so gewählt werden, dass sein Rand regulär ist.
\end{lemma}

\begin{proof}
  Zu jedem $x \in \partial K_2$ existiert eine Koordinatenumgebung $U$
  von $x$, so dass \break$K_1 \cap U = \varnothing$ gilt. Wir wählen eine kompakte
  Kreisscheibe $D$\footnote{Das heißt $z(D)$ ist eine Kreisscheibe in
    $\C$.}, die $x$ im Inneren enthält. Da $\partial K_2$auch
  kompakt ist, wird es durch endlich viele $D_1, \dots, D_k$
  überdeckt. Wir setzen $Y := K_2 \setminus (D_1 \cup \dots \cup D_k)$. 
  Dann ist $Y$ offen und liegt zwischen $K_1$ und $K_2$. 
  Seien $C_j, j \in J,$ die Zusammenhangskomponenten von $X \setminus
  K_2$. Nach Voraussetzung sind diese nicht relativ kompakt. Nun
  treffen alle $D_i$ mindestens ein $C_j$, da $D_i \cap (Y \setminus
  K_2) \neq \varnothing$ sein muss. Da alle $D_i$ zusammenhängend
  sind, sind die $D_i \cap C_j$ zusammenhängend und nicht relativ
  kompakt. Also existieren keine relativ kompakten
  Zusammenhangskomponenten in $X\setminus Y$. Es folgt$Y = \runge(Y)$.
  Weiterhin sind nach Satz \ref{thm:reg-rand} alle Randpunkte von $Y$ regulär.
\end{proof}


\begin{thm}
  \label{thm:runge-zshkomp}
  Sei $X$ eine Riemannsche Fläche und $Y \subset X$ eine offene
  Runge-Teilmenge. Dann ist jede Zusammenhangskomponente von $Y$ eine
  Runge-Teilmenge. 
\end{thm}

\begin{proof}
  Seien $Y_i, i \in I$, die Zusammenhangskomponenten von
  $Y$. Nach Lemma \ref{lemma:zsh-komp} sind alle $Y_i$ offen.
  Wir setzen nun $A := X \setminus Y$ und $A_k, k\in \N,$ die
  Zusammenhangskomponenten von $A$. Dann sind alle $A_k$
  abgeschlossen, aber nicht kompakt.
  
  Für jedes $i \in I$ gilt $\bar Y_i \cap A \neq
  \varnothing$. 
  Ansonsten wäre $\bar Y_i \subset Y$. Da dann $\bar Y_i \cap \bigcup_{j
    \neq i} Y_j = \varnothing$ gilt, müsste $Y_i = \bar Y_i$ gelten. Dies
  ist ein Widerspruch dazu, dass $X$ zusammenhängend ist.
  
  Es gilt $C \cap A \neq \varnothing$ für jede
  Zusammenhangskomponente $C$ von $X \setminus Y_i$. 
  Ansonsten gäbe es ein $j \neq i$, so dass $C \cap Y_j \neq
  \varnothing$ gälte.
  Nun ist aber $Y_j \subset X \setminus Y_i$ zusammenhängend und aus
  dem nicht leeren Schnitt folgte $Y_j \subset C$, da $C$ maximal
  zusammenhängend ist. Weiterhin folgte dann auch $\bar Y_j \subset
  \bar C = C$, da $C$ bereits abgeschlossen ist. Da $\bar Y_i \cap A
  \neq \varnothing$ gilt,  würde dies
  bedeuten, dass $A \cap C \neq \varnothing$ gelten müsste. Also
  genau, was wir zeigen wollten.
  
  Sei nun $C$ eine Zusammenhangskomponente von $X \setminus
  Y_i$. Dann trifft $C$ mindestens ein $A_k$. Also ist $A_k
  \subset C$ und da $A_k$ nicht kompakt ist, ist auch $C$ nicht kompakt.
  Damit ist $Y_i$ eine Runge-Teilmenge.
\end{proof}

\begin{thm}
  \label{thm:Ausschöpfung-Runge}
  Sei $X$ eine nicht kompakte Riemannsche Fläche. Dann existiert eine
  Ausschöpfung $Y_0 \Subset Y_1 \Subset \dots$ von $X$ durch relativ
  kompakte Runge-Gebiete. Weiterhin hat jedes $Y_i$ regulären Rand.
\end{thm}

\begin{proof}
  Sei $K \subset X$ kompakt. Da $K$ kompakt ist, kann es nur endlich
  viele Zusammenhangskomponenten besitzen, ansonsten könnten wir eine
  Folge ohne konvergente Teilfolgen konstruieren. Wählen wir nun aus
  jeder dieser endlich vielen Zusammenhangskomponenten einen Punkt
  aus, so können wir diese durch endlich viele Kurven
  verbinden. Vereinigen wir nun $K$ mit den Bildern der Kurven, die
  auch kompakt sind, so ist dies eine endliche Vereinigung und das
  Resultat ist wieder kompakt und nach Konstruktion
  zusammenhängend. Wir bezeichnen es mit $K_1$. Es folgt $K \subset
  K_1$. Nach Lemma \ref{lemma:komp-enthalten} finden wir ein kompaktes $K_2
  \subset X$ mit $K_1 \subset \mathring{K_2}$. Nach Lemma
  \ref{lemma:zwischen-runge}
  existiert eine offene Runge-Teilmenge $\tilde Y_1 \subset X$ mit
  $K_1 \subset \tilde Y_1 \subset \runge(K_2)$ mit regulärem Rand.
  
  Wählen wir nun die Zusammenhangskomponente $Y_1$, die $K_1$ enthält,
  so ist $Y_1$ nach Satz \ref{thm:runge-zshkomp} ein Runge-Gebiet
  und hat nach Bemerkung \ref{rem:reg-rand} auch regulären Rand.
  Nun finden wir eine Ausschöpfung von $X$ durch Kompakta $K_1 \subset
  K_2 \subset \dots$. 
  Durch das obige Vorgehen erhalten wir ein $Y_1$, das $K_1$
  enthält.
  Seien nun $Y_1, \dots, Y_k$ bereits konstruiert. Wir setzen $\tilde
  K_k = K_k \cup \bar Y_k$. Dann ist $\tilde K_k$ kompakt und wir
  können wieder wie oben ein Runge-Gebiet $Y_k \supset \tilde K_k$
  finden. Damit ist die gewünschte Ausschöpfung konstruiert.
\end{proof}

%%% Local Variables: 
%%% mode: latex
%%% TeX-master: "../Bachelor"
%%% End: 
