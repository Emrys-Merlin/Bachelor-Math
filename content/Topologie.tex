
\section{Abzählbare Topologie}
\label{sec:Topologie}

\begin{lemma}
  \label{lemma:surj-offen-top}
  Seien $X,Y$ topologische Räume und $f: X \ra Y$ stet, offen und
  surjektiv. \\
  Dann besitzt $Y$ eine abzählbare Basis, falls $X$ eine besitzt.
\end{lemma}

\begin{proof}
  Sei $\fu$ eine abzählbare Basis der Topologie von $X$ und
  \[
  \be := \{ f(U) : U \in \fu \}
  \]
  eine abzählbare Familie von offenen Teilmengen von $Y$. \\
  Behauptung: $\be$ ist eine Basis der Topologie von $Y$. \\
  Um dies zu beweisen wir $D \subset Y$ offen und $y \in D$ und
  müssen nun zeigen, dass ein $V \in \be$ existiert mit $y \in V
  \subset D$. \\
  Aus der Surjektivität von $f$ erhalten wir ein $x \in X$ mit $f(x) =
  y$. Weiterhin ist $f^{-1}(D)$ offen und eine Umgebung von $x$, da
  $f$ stetig ist. \\
  Da $\fu$ eine Basis der Topologie von $X$ ist existiert ein $U \in
  \fu$ mit $x \in U \subset f^{-1}(D)$. Also genügt $V:=f(U)$ den
  geforderten Eigenschaften.
\end{proof}

\begin{lemma}[Poincar´e-Volterra]
  \label{lemma:poincare-volterra}
  Sei $X$ eine zusammenhängende Mannigfaltigkeit und $Y$ ein
  Hausdorff-Raum mit abzählbarer Basis der Topologie. Sei weiterhin
  $f: X \ra Y$ stetig und diskret. \\
  Dann besitzt $X$ eine abzählbare Basis der Topologie.
\end{lemma}

\begin{proof}
  Sei $\fu$ eine abzählbare Basis der Topologie von $Y$. \\
  Bezeichne mit $\be$ die Menge aller offenen Teilmengen $V \subset X$
  mit
  \begin{enumerate}
  \item $V$ besitzt eine abzählbare Basis der Topologie
  \item $V$ ist eine Zusammenhangskomponente eines $f^{-1}(U)$ für ein
    $U \in \fu$.
  \end{enumerate}
  \begin{enumerate}
  \item Behauptung: $\be$ ist eine Basis der Topologie von $X$. \\
    Sei $D \subset X$ offen mit $x \in D$. Zu zeigen ist, dass ein $V
    \in \be$ existiert, so dass $x \in V \subset D$. \\
    Da $f$ diskret ist existiert ein $W \subset D$ offen und relativ
    kompakt, $x \in W$ und $\partial W \cap f^{-1}(f(x)) =
    \varnothing$. Dies lässt sich einfach auf einer Karte um $x$
    verifizieren. \\
    Nun ist $f(\partial w)$ kompakt, also abgeschlossen und $f(x)
    \notin f(\partial W)$, d.h. $f(x)$ liegt in der offenen Menge $\C
    \setminus f(\partial W)$, also existiert ein $U \in \fu$, so dass
    $f(x) \in U$ und $U \cap f(\partial W) = \varnothing$. Sei  $V$
    die Zusammenhangskomponente von $f^{-1}(U)$, die $x$ enthält. Dann
    gilt $V \cap \partial W = \varnothing$ und da $V$ zusammenhängend
    ist folgt $V \subset W$. Also besitzt $V$ eine abzählbare Basis
    der Topologie als Teilmenge einer relativ kompakten Menge und
    damit $V \in be$.
  \item Behauptung: Jedes $V_0 \in \be$ schneidet höchstens abzählbar
    viele $V \in \be$. \\
    Sei $U \in \fu$, dann sind per Definition die
    Zusammenhangskomponenten von $f^{-1}(U)$ disjunkt und da $V_0$
    eine abzählbare Basis der Topologie besitzt, kann $V_0$ höchstens
    abzählbarviele Zusammenhangskomponenten von $f^{-1}(U)$
    treffen. \\
    Da es weiterhin nur abzählbar viele $f^{-1}(U)$ gibt, folgt, dass
    $V_0$ höchstens abzählbar viele $V \in \be$ schneidet.
  \item Behauptung: $\be$ ist abzählbar.\\
    Fixiere dazu $V^\ast \in \be$ und definiere für jedes $n \in \N$
    $\be_n \subset \be$ wie folgt: \\
    $\be_n$ besteht asu allen $V \in \be$, so dass $V_0, \dots, V_n
    \in \be$ existieren mit $V_0 = V^\ast$, $V_n = V$ und $V_{k-1}
    \cap V_k \neq \varnothing$ für $k = 1, \dots, n$. \\
    % TODO: eventuell genauer erklären: Durch Wege verbinden und überdecken
    Da $X$ zusammenhängend ist folgt $\bigcup_{n \in \N} \be_n =
    \be$. \\
    Also reduziert sich unser Problem darauf zu zeigen, dass $\be_n$
    für jedes $n \in \N$ abzählbar ist; dies tun wir induktiv. \\
    Klarerweise ist $\be_0 = \{V^\ast\}$ abzählbar. Angenommen $\be_n$
    ist abzählbar. $\be_{n+1}$ besteht dann aus allen $V \in be$, so
    dass ein $\tilde V \in be_n$ existiert mit $V \cap \tilde V \neq
    \varnothing$. Nach 2. existieren zu jedem $\tilde V \in be_n$ nur
    abzählbar viele $V$, die nicht-leeren Schnitt damit besitzen. \\
    Da $\be_n$ nun abzählbar ist und pro $\tilde V$ nur abzählbar
    viele $V$ existieren, ist auch $\be_{n+1}$ abzählbar.
  \end{enumerate}
  Damit besitzt $X$ eine abzählbare Basis der Topologie.
\end{proof}

\begin{thm}[Rad´o]
  \label[thm:rado]
  Jede Riemannsche Fläche $X$ besitzt eine abzählbare Basis der Topologie.
\end{thm}

\begin{proof}
  Sei $U$ eine Koordinatenumgebung von $X$. Wähle $K_0, K_1 \subset
  U$, so dass $z(K_0)$ und $z(K_1)$ disjunkte, kompakte Kreisscheiben
  sind und setze $Y := X \setminus (K_0 \cup K_1)$. \\
  Nun genügt $\partial Y = \partial K_0 \cap \partial K_1$ dem
  % TODO: Referenz
  Regulariätskriterium von (22.18). Also existiert eine stetige
  Funktion $u: \bar Y \ra \R$, die harmonisch auf $Y$ ist und
  $u|_{\partial K_0} \equiv 0$ und $u|_{\partial K_1} \equiv 1$
  genügt. \\
  Da $u$ aufgrund der Randwerte nicht konstant sein kan erahlten wir
  eine nicht-triviale holomorphe 1-Form $\omega := d' U$ auf $Y$. \\
  % TODO: Referenz? Wo genau...
  Sei $f$ eine holomorphe Stammfunktion von $p^\ast\omega$ auf der
  universellen Überlagerung $p: \tilde Y \ra Y$. \\
  Da $f$ nicht konstant ist, genügt $f$ den Voraussetzungen von
  Lemma \label{lemma:poincare-volterra} (denn $\C$ ist hausdorffsch
  und besitzt eine abzählbare Basis der Topologie). Also besitzt auch
  $\tilde y$ eine abzählbare Basis der Topologie und aus
  Lemma \label{lemma:surj-offen-top} folgt, dass $Y$ eine abzählbare
  Basis der Topologie. \\
  Nun ist $X = Y \cup U$ und $U$ ist homöomorph zu einer offenen
  Teilmenge in $\C$, besitzt also auch eine abzählbare Basis der
  Topologie. \\
  Insgesamt ergibt sich also, dass auch $X$ eine abzählbare Basis der
  Topologie besitzt.
\end{proof}

\begin{defin}
  Sei $X$ eine Riemannsche Fläche. Für alle $Y \subset X$ definieren
  wir $\runge(Y)$ durch $Y$ vereinigt mit allen relativ kompakten
  Zusammenhangskomponenten von $X \setminus Y$. \\
  $Y \subset X$ heißt \init{Runge-Teilmenge} falls $Y = \runge(Y)$. \\
  Es gilt:
  \begin{enumerate}
  \item $\runge(\runge(Y)) = \runge(Y) \quad \forall Y \subset X$
    %TODO: Referenz!
  \item $Y_1 \subset Y_2$ impliziert $\runge(Y_1) \subset \runge(Y_2)$
  \end{enumerate}
\end{defin}

\begin{thm}
  Sei $Y \subset X$ und $X$ eine Riemannsche Fläche. Dann gilt:
  \begin{enumerate}
  \item $Y$ abgeschlossen $\Ra$ $\runge(Y)$ abgeschlossen
  \item $Y$ kompakt $\Ra$ $\runge(Y)$ kompakt
  \end{enumerate}
\end{thm}

\begin{proof}
  \begin{enumerate}
  \item Seien $C_j, j \in J$ die Zusammenhangskomponenten von $X
    \setminus Y$ Da $X \setminus Y$ ist offen und $X$ ist eine
    Mannigfaltigkeit, deshalb sind alle $C_j$ offen. \\
    Sei $J_0 \subset J$, die Menge der $j \in J$, so dass $C_j$
    relativ kompakt ist. Dann erhalten wir
    \[
    X \setminus \runge(Y) = \bigcup_{j \in J\setminus J_0} C_j
    \]
    und ist damit offen, als Vereinigung offener Mengen und folglich
    ist $\runge(Y)$ abgeschlossen.
  \item Wir können ohne Einschränkung $Y \neq \varnothing$
    % TODO: evtl. genauer rel komp Umgebung U. Überdeckung durch
    % rel. komp KoordUmg
    annehmen. Sei $U$ eine offene, relativ kompakte Umgebung von $Y$
    und $C_j, j\in J$ wie oben.
    \begin{enumerate}
    \item Behauptung: Jedes $C_j$ trifft $\bar U$. \\
      Ansonsten wäre $C_j \subset X \setminus \bar U$, also $\bar C_j
      \subset X \setminus U \setminus X \setminus Y$. Da $C_j$ eine
      Zusammenhangskomponente von $X \setminus Y$ ist, folgt $C_j =
      \bar C_j$. Also ist $C_j$ offen und abgeschlossen, also $C_j =
      \varnothing$ oder $C_j = X$. Beides führt zum Widerspruch
    \item Behauptung: Nur endlich viele $C_j$ treffen $\partial U$. \\
      Dies folgt daraus, dass $\partial U$ kompakt ist und durch die
      disjunkten $C_j$ überdeckt wird.
    \end{enumerate}
    Seien nun wieder $C_j, j \in J_0$ die relativ kompakten
    Zusammenhangskomponente von $X \setminus Y$ und $C_{j_1}, \dots,
    % TODO: Referenz
    C_{j_n}$ diejenigen, die $\partial U$ schneiden. Nach AKSDF(a)
    liegen alle anderen relativ kompakten Zusammenhangskomponenten in
    $U$. Also ist
    \[
    \runge(Y) \subset U \cup C_{j_1} \cup \dots \cup C_{j_n}
    \]
    und damit relativ kompakt. Da $\runge(Y)$ nach 1. aber auch
    abgeschlossen ist, ist es tatsächlich kompakt
  \end{enumerate}
\end{proof}

\begin{cor}
  \label{cor:ausschöpfung-kompakt}
  Sei $X$ eine nicht kompakte Riemannsche Fläche. \\
  Dann existieren kompakte Runge-Teilmenge $K_j \subset X, j \in \N$ mit
  \begin{enumerate}
  \item $K_{j-1} \subset \mathring{K_j} \quad \forall j \geq 1$
  \item $\bigcup_{j\in \N} K_j = X$
  \end{enumerate}
\end{cor}

\begin{proof}
  Da $X$ eine abzählbare Basis der Topologie besitzt existiert, eine
  Ausschöpfung
  \[
  K_0' \subset K_1' \subset \dots
  \]
  durch Kompakta von $X$. \\
  Setze nun $K_0 := \runge(K_0')$ \\
  Seien $K_0, \dots, K_n$ bereits konstruiert. Dann existiert ein
  % TODO: Konstruktion von M. Wieder durch Koordumgebungen überdecken.
  kompaktes $M \subset X$ mit $K_n' \cup K_n \subset \subset \mathring
  M$. Setze dann $K_{n+1} := \runge(M)$.
\end{proof}

\begin{lemma}
  \label{lemma:zwischen-runge}
  Seien $K_1, K_2 \subset X$ kompakte Teilmengen und $X$ eine
  Riemannsche Fläche mit $K_1 \subset \mathring{K_2}$ und $\runge(K_2)
  = K_2$. \\
  Dann existiert eine offene Runge-Teilmenge $Y \subset X$ mit $K_1
  \subset Y \subset K_2$. \\
  Weiterhin kann $Y$ so gewählt werden, dass sein Rand regulär ist.
\end{lemma}

\begin{proof}
  Zu jedem $x \in \partial K_2$ existiert eine Koordinatenumgebung $U$
  von $x$, so dass $K_1 \cap U = \varnothing$. Wähle eine kompakte
  Scheibe $D$, die $x$ im Inneren enthält. Da $\partial K_2$ auch
  kompakt ist, wird es durch endlich viele $D_1, \dots, D_k$
  überdeckt. Setze $Y := K_2 \setminus (D_1 \cup \dots \cup D_k)$. \\
  Dann ist $Y$ offen und liegt zwischen $K_1$ und $K_2$. \\
  Seien $c_j, j \in J$ die Zusammenhangskomponenten von $X \setminus
  K_2$. Nach Voraussetzund sind diese nicht relativ kompakt. Nun
  treffen alle $D_i$ mindestens ein $C_j$, da $D_i \cap (Y \setminus
  K_2) \neq \varnothing$ sein muss. DA alle $D_i$ zusammenhängend
  sind, sind die $D_i \cap C_j$ zusammenhängend un nicht relativ
  kompakt. Also existieren keine relativ kompakten
  Zusammenhangskomponenten in $X\setminus Y$. Also $Y = \runge(Y)$. \\
  Weiterhin sind nach (22.18) alle Randpunkte von $Y$ regulär.
\end{proof}

\begin{thm}
  \label{thm:runge-zshkomp}
  Sei $Y \subset X$ eine offene Runge-Teilmenge und $X$ eine
  Riemannsche Fläche. \\
  Dann ist jede Zusammenhangskomponente runge.
\end{thm}

\begin{proof}
  \begin{enumerate}
  \item Seien $Y_i, i \in I$, die Zusammenhangskomponenten von
    % TODO: Warum offen? evtl. infinty viele ZshKomp. Dann gibt es GegBsp
    $Y$. $Y$ ist offen und $X$ ist eine Mannigfaltigkeit, also sind
    alle $Y_i$ offen. \\
    Setze nun $A := X \setminus Y$ und $A_k, k\in \N$ die
    Zusammenhangskomponenten von $A$. Dann sind alle $A_k$
    abgeschlossen, aber nicht kompakt.
  \item Behauptung: Für jedes $i \in I$ gilt: $\bar Y_i \cap A \neq
    \varnothing$.\\
    Ansonsten wäre $\bar Y_i \subset Y$. Da $\bar Y_i \cap \bigcup_{j
      \neq i} Y_j = \varnothing$. Also ist $Y_i = \bar Y_i$. Dies
    ist ein Widerspruch dazu, dass $X$ zusammenhängend ist.
  \item Behauptung: $C \cap A \neq \varnothing$ für jede
    Zusammenhangskomponente von $X \setminus Y_i$. \\
    Ansonsten gäbe es ein $j \neq i$, so dass $C \cap Y_j \neq
    \varnothing$. \\
    % TODO: Warum?? Irgendwas weil beides zsh ist.
    Nun ist aber $C$ abgeschlossen und $Y_j$ zusammenhängend, also
    $\bar Y_j \subset C$ und damit ist $C \cap Y_j \neq \varnothing$.
  \item Sei nun $C$ eine Zusammenhangskomponente von $X \setminus
    Y_i$. Dann trifft $C$ nach 3. mindestens ein $A_k$. Also ist $A_k
    \subset C$ und da $A_k$ nicht kompakt ist, ist $C$ auch nicht kompakt.
  \end{enumerate}
  Damit ist $Y_i$ runge.
\end{proof}

\begin{thm}
  \label{thm:Ausschöpfung-Runge}
  Sei $X$ eine nicht kompakte Riemannsche Fläche. \\
  Dann existiert eine Ausschöpfung $Y_0 \Subset Y_1 \Subset \dots$ von
  $X$ durch relativ kompakte Runge-Gebiete. Weiterhin hat jedes $Y_i$
  regulären Rand.
\end{thm}

\begin{proof}
  % TODO: zusammenhängende kompakte Mengen: Endliche viele ZshKomp +
  % Verbinden durch Kurve
  Sei $K \subset X$ kompakt. Dann existiert ein zusammenhängendes und
  kompaktes $K_1 \supset K$ und ein kompaktes $K_2 \subset X$ mit $K_1
  \subset \mathring{K_2}$. Nach Lemma \ref{lemma:zwischen-runge}
  existiert eine offene Runge-Teilmenge $\tilde Y_1 \subset X$ mit
  $K_1 \subset \tilde Y_1 \subset \runge(K_2)$. mit regulärem Rand. \\
  Wählen wir nun die Zusammenhangskomponente $Y_1$, die $K_1$ enthält,
  so ist $Y_1$ nach Satz \label{thm:runge-zshkomp} ein Runge-Gebiet
  % TODO: Referenz
  und hat nach Bem nach 22.18 auch regulären Rand. \\
  Nun finden wir eine Ausschöpfung von $X$ durch Kompakta $K_1 \subset
  K_2 \subset \dots$. \\
  Durch das obige vorgehen erhalten wir ein $Y_1$, das $K_1$
  enthält.\\
  Seien nun $Y_1, \dots, Y_k$ bereits konstruiert. Setzen wir $\tilde
  K_k = K_k \cup \bar Y_k$. Dann ist $\tilde K_k$ kompakt und wir
  können wieder wie oben ein Runge-Gebiet $Y_k \supset \tilde K_k$
  finden.\\
  Damit ist die gewünschte Ausschöpfung konstruiert.
\end{proof}

%%% Local Variables: 
%%% mode: latex
%%% TeX-master: "../Bachelor"
%%% End: 
