\section{Uniformisierung kompakter Riemannscher Flächen}
\label{sec:uniformisierung}

Wir nähern uns dem eigentlichen Ziel dieser Arbeit. Am Ende dieses
Kapitels wird der Uniformisierungssatz für kompakte Riemannsche
Flächen stehen. Bevor es jedoch so weit ist, müssen wir noch einige
Eigenschaften von Decktransformationen zusammentragen. Zum Einen
werden wir sehen, dass genau die Fuchsschen Gruppen die möglichen
Decktransformationsgruppen der oberen Halbebene darstellen, zum Anderen
wird klar werden, dass $\P^1$ nur sich selbst überlagert. Zu guter
Letzt müssen wir auch noch die Fundamentalgruppe der Riemannschen
Flächen mit Geschlecht 1 bestimmen, dazu müssen wir einen Satz zur
Charakterisierung kompakter, orientierbarer, topologischer Flächen
importieren.

Nach diesen letzten technischen Spielereien sind wir dann endlich in
der Lage aus dem Riemannschen Abbildungssatz unseren
Uniformisierungssatz herzuleiten.

\begin{lemma}
  \label{lemma:decktrafo-diskret}
  Sei $X$ eine Riemannsche Fläche, $p: \tilde X \ra X$ die
  Universelle Überlagerung und $G = \Deck(X /\tilde X) \leq
  \Aut(\tilde X)$. Dann gelten:
  \begin{enumerate}
  \item Für beliebige $\sigma \in G \setminus \{\id\}$ gilt, dass $\sigma
    (x) \neq x $ für alle $x \in \tilde X$.
  \item Für alle $x \in \tilde X$ ist der Orbit $Gx := \{ \sigma(x)
    \mid x \in G\}$ diskret.
  \end{enumerate}
\end{lemma}

\begin{proof}
  \begin{enumerate}
  \item Jede Decktransformation ist eindeutig durch einen Punkt
    bestimmt. Dies folgt aus der Eindeutigkeit für das Liften von
    Abbildunge (vgl. \cite[Satz 4.8]{For}). Das bedeutet aus
    $\sigma(x) = x$ für ein $x \in \tilde X$, folgt
    bereits $\sigma = \id$.
  \item Nach \cite[Satz 5.6]{For} ist $\Deck(\tilde X/X)$ galoissch
    (bzw. regulär) und
    es folgt \break$Gx = p^{-1}(p(x))$. Die Aussage folgt nun, da die Faser
    von $p(x)$ diskret ist.
  \end{enumerate}
\end{proof}

\begin{lemma}
  \label{lemma:deck-pc}
  \begin{enumerate}
  \item Jeder Automorphismus von $\P^1$ hat einen Fixpunkt.
  \item Sei $G \leq \Aut(\C)$ eine diskret operierende Gruppe ohne
    Fixpunkte. Dann ist $G$ eine der folgenden Gruppen
    \begin{enumerate}
    \item $G = \{\id\}$,
    \item $G = \{ z \mapsto z + n \gamma \mid n \in \Z\}$, wobei
      $\gamma$ in $\C^\times$ liegt oder
    \item $G = \{z \mapsto z + n \gamma_1 + m \gamma_2 \mid n,m \in
      \Z\}$, wobei die $\gamma_1, \gamma_2 \in \C^\times$ linear
      unabhängig über $\R$ sind.
    \end{enumerate}
  \end{enumerate}
\end{lemma}

\begin{proof}
  \begin{enumerate}
  \item Sei $f \in \Aut(\P^1)$. Dann ist nach Satz \ref{thm:aut} $f(z) = M\langle z \rangle$ mit $M
    \in \PSL(2, \C)$. Sei
    \[
    M =
    \begin{pmatrix}
      a & b \\
      c & d
    \end{pmatrix}.
    \]
    Falls $c \neq 0$ gilt, so erhalten wir als Fixpunkte
    \[
    z_{1,2} = \frac{a}{2c} \pm \sqrt{\frac{a^2}{4 c^2} - \frac{d-b}{c}}.
    \]
    Falls $c = 0$ gilt, so muss $a \neq d$ gelten, ansonsten wäre
    $\det M = 0$ und wir erhalten als Fixpunkt
    \[
    z = \frac{b}{d-a}.
    \]
  \item Sei $f \in G \leq \Aut(\C)$. Nach Satz \ref{thm:aut}
    existieren $a \in \C^\times$ und $b \in \C$ mit \break$f(z) = az
    +b$. Sei nun $f \neq \id$. Angenommmen es gälte $a \neq 1$, dann
    wäre $w = \frac{b}{1-a}$ ein Fixpunkt von $f$, allerdings operiert
    $G$ nach Voraussetzung fixpunktfrei. Also muss $a = 1$
    gelten. Also ist $f(z) = z + b$. Definieren wir $\Gamma := \{f(0) \mid
    f \in G\}$, so ist $\Gamma$ eine additive, diskrete Untergruppe
    von $\C$. Nun können alle reellen (Unter-)vektorräume $V \subset
    \C$ die Dimension 0, 1 oder 2 besitzen und $\Gamma$ ist in einem
    dieser drei Typen so enthalten, dass es in keinem echten
    Untervektorraum liegt. Aus Lemma \ref{lemma:gitter} erhalten wir
    also, dass $\Gamma = \{0\}$, $\Gamma = \{ n \gamma \mid n \in \Z\}$,
    wobei $\gamma \in \C^\times$ oder\break $\Gamma = \{n \gamma_1 + m
    \gamma_2 \mid n,m \in \Z \}$, wobei die $\gamma_1, \gamma_2 \in
    \C^\times$ linear unabhängig über $\R$ sind, gilt. Nun ist aber
    $G = \{z \mapsto z + b \mid b \in \Gamma \}$.
    Dies liefert die Behauptung.
  \end{enumerate}
\end{proof}

Bevor wir uns unserem Hauptresultat widmen können, müssen wir noch
einen Satz aus der Topolgie importieren.

\begin{thm}
  \label{thm:top-geschlecht}
  Seien $X$ und $Y$ kompakte Riemannsche Flächen mit Geschlechtern $g$
  und $h$. Dann sind $X$ und $Y$ genau dann homöomorph, wenn $g = h$ ist.
\end{thm}

\begin{proof}
  Siehe zum Beispiel \cite[Korollar 2.4.A.2]{Jos}
\end{proof}

\begin{rem}
  Dieser Satz gilt noch deutlich allgemeiner als hier aufgeführt,
  allerdings muss man sich dann über die Definition des Geschlechts
  Gedanken machen, da wir für unsere Definition auf die
  komplexe Struktur zurückgegriffen haben. Allgemein kann das
  Geschlecht aber bereits für beliebige kompakte, orientierbare, topologische
  Flächen definiert werden.
\end{rem}

\begin{cor}
  \label{cor:torus-homöo}
  Sei $\Gamma \subset \C$ ein Gitter und $X$ eine kompakte Riemannsche
  Fläche mit Geschlecht 1. Dann sind $X$ und $\quot{\C}{\Gamma}$ homöomorph.
\end{cor}

\begin{proof}
  Nach Korollar \ref{cor:torus-geschlecht} hat der Torus Geschlecht
  1. Aus Satz \ref{thm:top-geschlecht} folgt dann die Behauptung.
\end{proof}

\begin{thm}
  \label{thm:geschlecht-1-z}
  Sei $X$ eine kompakte Riemannsche Fläche mit Geschlecht $g =
  1$. Dann ist die Fundamentalgruppe von $X$ isomorph zu $\Z \oplus \Z$.
\end{thm}

\begin{proof}
  Nach Korollar \ref{cor:torus-homöo} ist $X$ homöomorph zu
  $\quot{\C}{\Gamma}$ für ein beliebiges Gitter $\Gamma \subset
  \C$. Nun ist aber die Fundamentalgruppe eine topologische
  Invariante, d.h. es gilt \break$\pi_1(X) \cong
  \pi_1\left(\quot{\C}{\Gamma}\right)$. Weiterhin ist die Universelle
  Überlagerung zu $\quot{\C}{\Gamma}$ durch $\C$ gegeben und es gilt $
  \pi_1\left (\quot{\C}{\Gamma}\right) \cong \Deck \left (\C \setminus
    \quot{\C}{\Gamma} \right )$. Die Decktransformationsgruppe ist
  aber gerade durch Translationen um Elemente aus $\Gamma$ gegeben und
  $\Gamma$ ist klarerweise isomorph zu $\Z \oplus \Z$. Insgesamt
  erhlaten wir also, dass $\pi_1(X)$ isomorph zu $\Z \oplus \Z$ ist.
\end{proof}

Nun haben wir das Handwerkszeug zusammen, um unser zentrales Resultat
zu beweisen.

\begin{thm}
  \label{thm:uniformisierung}
  Sei $X$ eine kompakte Riemannsche Fläche und bezeichne $g$ ihr
  Geschlecht. Dann gelten:
  \begin{enumerate}
  \item Ist $g = 0$, so ist $X$ konform äquivalent zu $\P^1$.
  \item Ist $g = 1$, so existiert ein Gitter $\Gamma \subset \C$, so
    dass $X$ konform äquivalent zu $\quot{\C}{\Gamma}$ ist.
  \item Ist $g \geq 2$, so existiert eine Fuchssche Gruppe $G \leq
    \Aut(\h)$, so dass $X$ konform äquivalent zu $\quot{\h}{G}$ ist.
  \end{enumerate}
\end{thm}

\begin{proof}
  Der erste Fall ergibt sich direkt aus dem Satz von Riemann-Roch
  \cite[Kor. 16.13]{For}. Sei
  nun also $g \geq 1$ und bezeichne $\tilde X$ die Universelle
  Überlagerung von $X$. Dann ist $\tilde X$ einfach zusammenhängend
  und aus dem Riemannschen Abbildungssatz \ref{thm:rmt} folgt, dass
  $\tilde X$ konform äquivalent zu $\P^1$, $\C$ oder $B$ ist. Da wir
  wissen, dass $B$ konform äquivalent zu $\h$ ist, können wir in der
  Betrachtung genau so gut $\h$ verwenden. Nun wissen wir nach Lemma
  \ref{lemma:deck-pc}, dass jeder Automorphismus von $\P^1$ einen
  Fixpunkt besitzt, d.h. es gibt keine fixpunktfrei-operierende
  Untergruppe von $\Aut(\P^1)$. Also überlagert $\P^1$ nur sich
  selbst und es müsste $X \cong \tilde X \cong \P^1$ gelten. Dies ist
  ein Widerspruch zu $g \neq 0$. Dementsprechend ist die Universelle Überlagerung
  von $X$ entweder $\C$ oder $\h$.

  Sei nun $g \geq 2$. Angenommen $\tilde X \cong \C$. In Lemma
  \ref{lemma:deck-pc} wurden alle diskreten, fixpunkfreioperierenden
  Untergruppen von $\Aut(\C)$ bestimmt. Die zugehörigen Riemannschen
  Flächen sind dann $X \cong \C$, $X \cong \C^\times$ oder $X \cong
  \quot{\C}{\Gamma}$ für ein Gitter $\Gamma \subset \C$. Nun sind aber
  die ersten beiden Möglichkeiten nicht kompakt und die Dritte hat
  Geschlecht $g =1$. Ein Widerspruch. Also muss die Universelle
  Überlagerung von $X$ konform äquivalent zu $\h$ sein und damit ist
  $X$ konform äquivalent zu $\quot{\h}{G}$ für eine Fuchssche Gruppe $G
  \leq \Aut(\h)$.


  Nun fehlt noch der Fall $g = 1$. Angenommen $\tilde X \cong \h$. Wir
  wissen aus Satz \ref{thm:geschlecht-1-z}, dass
  \begin{align}
    \label{eq:g-z}
  \Z \oplus \Z \cong G := \Deck(\tilde X \setminus X) \leq \Aut(\h)
  \end{align}
  gelten müsste. Insbesondere müssten wir eine abelsche Fuchssche
  Gruppe finden, allerdings wissen wir nach Satz
  \ref{thm:abelsch-zyklisch}, dass diese alle zyklisch sind. Damit ist
  $G$ also isomorph zu $\Z$ oder zu $\quot{\Z}{n\Z}$ für ein $n \in
  \N$. Dies ist ein Widerspruch dazu, dass Gleichung \eqref{eq:g-z}
  gelten soll. Also ist die Universelle Überlagrung von $X$ konform
  äquivalent zu $\C$. Die Gruppe der Decktransformation ist also eine
  diskrete abelsche Untergruppe von $\C$ und muss isomorph zu $\Z
  \oplus \Z$ sein. In Lemma \ref{lemma:deck-pc} sind die möglichen diskreten
  Untergruppen von $\Aut(\C)$ charakterisiert und die einzige Möglichkeit, die
  isomorph zu $\Z \oplus \Z$ ist, ist die des Gitters. Also ist die
  $\Deck(\tilde X \setminus X)$ isomorph zu einem Gitter $\Gamma
  \subset \C$. Insgesamt erhalten wir, dass $X$ konform äquvialent zu
  $\quot{\C}{\Gamma}$ ist.
\end{proof}

Damit haben wir das Ziel der Arbeit erreicht. Wir haben alle kompakten
Riemannschen Flächen vollständig charakterisiert. Die Reichweite
dieses Resultats wird natürlich erst in der Anwendung spürbar. Diese
würde jedoch den Rahmen der Arbeit endgültig sprengen und wir
wollen an dieser Stelle nur noch auf mögliche Literatur verweisen.

Für einen differentialgeometrischen Zugang zu kompakten Riemannschen
Flächen wäre \cite{Jos} zu empfehlen. Dort wird auch auf den Zusammenhang
zwischen dem Uniformisierungssatz und der Existenz verschiedener
Geometrien eingegangen.
Es findet sich ebenso ein alternativer Beweis des
Uniformisierungssatzes, der deutlich analytischer geprägt
ist. Weiterhin findet eine Einführung in die Theorie der
Teichmüller-Räume statt, die in dieser Arbeit vollkommen ignoriert
wurde.

Für eine ausgiebige Diskussion algebraischer Kurven und deren
Zusammenhang mit Riemannschen Flächen lässt sich auf
\cite{Mir} verweisen. Große Teile dieses Textes beschäftigen sich mit
der Konstruktion von 
Riemannschen Flächen aus algebraischen Kurven. Außerdem wird
als Anwendung der Dualität von Flächen und Kurven gezeigt, wie
mögliche Gruppenstrukturen von Riemannschen Flächen auf
algebraische Kurven übertragen werden können.

%%% Local Variables: 
%%% mode: latex
%%% TeX-master: "../Bachelor"
%%% End: 
