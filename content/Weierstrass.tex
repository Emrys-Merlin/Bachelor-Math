
\section{Der Weierstrasssche Produktsatz}
\label{sec:Weierstrass}

\begin{defin}{Divisor}
  Sei $X$ eine Riemannsche Fläche. Ein \init{Divisor} ist eine
  Abbildung $D: X \ra \Z$, so dass für jede kompakte Teilmenge $K
  \subset X$ gilt: $D(x) = 0$ für fast alle $x \in K$. \\
  Wir bezeichnen mit $Div(X)$ die Menge aller Divisoren. \\
  Für eine Funktion $f \in \mer(X)^\ast$ definieren wir den zu $f$
  gehörigen Divisor $(f): X \ra \Z, \quad x \mapsto \ord_x(f)$. Da die
  Null- und Polstellen eine diskrete Teilmenge von $X$ bilden, ist
  dies tatsächlich ein Divisor.
\end{defin}

\begin{defin}
  Sei $X$ eine Riemannsche Fläche und $D \in Div(X)$. Dann heißt $f
  \in \mer(X)$ eine \emph{Lösung} von $D$, falls $(f) = D$. \\
  Setzen wir $X_D := \{x \in X: D(x) \geq 0 \}$, so heißt $f \in
  \diff(X_D)$ \emph{schwache Lösung} von $D$, falls für alle $a \in X$
  eine Koordinatenumgebung $(U,z)$ mit $z(a) = 0$ und ein $\psi \in
  \diff(U)$ mit $\psi(a) \neq 0$ existieren, so dass $f = \psi z^k$
  auf $ U \cap X_D$ gilt, wobei $k = D(a)$.
\end{defin}

\begin{lemma}
  \label{lemma:schwache-Lösung-Divisor}
  Jeder Divisor $D$ auf einer nicht kompakten Riemannschen Fläche $X$ hat eine
  schwache Lösung.
\end{lemma}

\begin{proof}
  Wir wählen $K_1, K_2, \dots$ kompakte Runge-Teilmengen von $X$ mit
  \begin{enumerate}
  \item $K_j \subset \mathring{K}_{j+1}$ $\forall j \geq 1$ und
  \item $\bigcup_{j \in \N} K_j = X$
  \end{enumerate}
  Dies ist nach Korollar \ref{cor:ausschöpfung-kompakt} möglich.
  \begin{itemize}
  \item Behauptung: Sei $a_0 \in X \setminus K_j$ und $A_0 \in Div(X)$
    mit $A_0(a_0) = 1$ und $A_0(x) = 0$, falls $x \neq a_0$. Dann
    existiert eine schwache Lösung $\phi$ von $A_0$ mit $\phi |_{K_j}
    = 1$. \\
    \\
    $K_j$ ist eine Runge-Teilmenge, d.h. $K_j = \runge{K_j}$. Also
    liegt $a_0$ in einer nicht relativ kompakten
    Zusammenhangskomponente $U \subset X \setminus K_j$ \\
    Nun existiert ein $a_1 \in U \setminus K_{j+1}$ (ansonsten wäre
    $K_{j+1}$ nicht kompakt) und eine Kurve $c_0: I \ra U$ mit
    $c_0(0)=a_1$ und $c_0(1) = a_0$. \\
    Nach \cite[Lemma 20.5]{For} existiert eine schwache Lösung $\phi_0$ des Divisors
    $\partial c_0$ mit $\phi_0|_{K_j} = 1$. \\
    Induktiv sind wir nun in der Lage eine Folge von Punkten $a_\nu \in
    X \setminus K_{j+ \nu}$, $\nu \in \N$, und Kurven in $X \setminus
    K_{j+\nu}$ von $a_{\nu+1}$ nach $a_\nu$ konstruieren. Analog zum
    obigen erfahren erhalten wir dann wieder schwache Lösungen
    $\phi_\nu$ des Divisors $\partial c_\nu$ mit $\phi_\nu |_{K_{j+\nu}}
    = 1$. \\
    Nach Konstruktion gilt $\partial c_\nu = A_\nu - A_{\nu+1}$, wobei
    $A_\nu(a_\nu) = 1$ und sonst verschwindet $A_\nu$. \\
    Weiterhin folgt, dass $\phi_0 \cdot \phi_1 \dots \phi_n$ eine
    schwache Lösung von $A_0 - A_{n+1}$ ist.
    \todo{Warum ist das eine schwache Lösun?}
    Lassen wir $n$ gegen unendlich gehen, so erhalten wir $\phi :=
    \prod_{\nu = 0}^\infty \phi_\nu$. Dieses Produkt konvergiert, da
    auf jeder kompakten Teilmenge nur endlich viele Faktoren von 1
    verschieden sind. Dies folgt aus Lemma
    \ref{lemma:kompakt-in-ausschöpfung}, denn zu jeder kompakten Menge
    $K$ existiert ein $n \in \N$, so dass $K \subset K_n$. Dann gilt
    aber $\phi_j |_K \equiv 1$ für alle $j > n$.
    Damit ist $\phi$ eine schwache Lösung von $A_0$.
  \item Sei nun $D$ ein beliebiger Divisor auf $X$. Für $\nu \in \N$
    setzen wir
    \[
    D_\nu(x) := \begin{cases} D(x) & x \in K_{\nu+1}\setminus K_\nu\\ 0
      & \text{sonst}\end{cases}
    \]
    wobei $K_0$ als leere Menge definiert wird. Wir erhalten dann $D =
    \sum_{\nu = 0}^\infty D_\nu$. \\
    Nun ist $D_\nu$ nur an endlich vielen stellen von 0 verschieden und
    kann deshalb als endliche Summe von Divisoren der Form $A_\nu$
    dargestellt werden. Für diese wurde oben eine schwache Lösung
    konstruiert und das (endliche) Produkt dieser Lösungen liefert uns
    eine Lösung $\psi_\nu$ von $D_\nu$ und wir erhalten sogar $\psi_\nu
    |_{K_\nu} = 1$. \\
    Setzen wir nun $\psi := \prod_{\nu=0}^\infty \psi_\nu$, so
    konvergiert das Produkt nach den gleichen Argumenten, wie oben und
    ist eine schwache Lösung von $D$.
  \end{itemize}
\end{proof}


\begin{thm}
  \label{thm:Lösung-Divisor}
  Sei $X$ eine nicht kompakte Riemannsche Fläche und $D \in
  Div(X)$. Dann existiert ein $f \in \mer(X)^\ast$ mit $(f) = D$.
\end{thm}

\begin{proof}
  Wir können $X$ mit relativ kompakten, einfach zusammenhängenden
  Koordinatenumgebungen $(U_i)_{i \in I}$ überdecken. Auf diesen
  finden wir $f_i \in \mer(U_i)^\ast$, so dass $(f_i) =
  D|_{U_i}$. Dies ist möglich, da $D$ auf dem relativ kompakten $U_i$
  fast überall verschwindet und $f_i$ deshalb in lokalen Koordinaten
  einfach als rationale Funktion darstellen.\\
  Mit dieser Konstruktion haben $f_i$ und $f_j$ dies selben Pol- und
  Nullstellen auf $U_i \cap U_j$, also ist
  \[
  \frac{f_i}{f_j} \in hol(U_i \cap U_j)^\ast \qquad \forall i,j \in I
  \]
  Nach Lemma \ref{lemma:schwache-Lösung-Divisor} existiert eine
  schwache Lösung $\psi$ von $D$ und auf den $U_i$ finden wir $\psi_i
  \in \diff(U_i)$, so dass $\psi = \psi_i \cdot f_i$ und $\psi_i(x)
  \neq 0$ für alle $x \in U_i$. (Wir setzend abei $\psi$ auf ganz $X$
  fort, in dem wir es an den nicht definierten Stellen $\infty$
  setzen). \\
  Da $U_i$ einfach zusammenhängend ist und $\psi_i$ nicht
  verschwindet, existiert ein $\phi_i \in \diff(U_i)$ mit $\psi_i =
  e^{\phi_i}$. \\
  Somit erhalten wir
  \[
  \psi = e^{\phi_i} f_i \qquad \text{auf } U_i
  \]
  und
  \[
  e^{\phi_j - \phi_i} = \frac{f_i}{f_j} \in \hol(U_i \cap U_j)^\ast
  \qquad \text{auf } U_i \cap U_j
  \]
  Damit ist $\phi_{ij} := \phi_i - \phi_j \in \hol(U_i \cap U_j)$, da
  die $\exp$-Funktion lokal biholomorph ist. Aus der Definition folgt
  auch direkt $\phi_{ij} + \phi_{jk} = \phi_{ik}$ auf $U_i \cap U_j
  \cap U_k$ und damit ist $(\phi_{ij}) \in Z^1(\fu, \hol)$. \\
  Nun ist $H^1(X, \hol) = 0$, also existieren $g_i \in \hol(U_i)$ mit
  $\phi_{ij} = \phi_j - \phi_i = g_j - g_i$ auf $U_i \cap U_j$. \\
  Eingesetzt folgt
  \[
  e^{g_j} f_j = e^{g_i} f_i \qquad \text{auf } U_i \cap U_j
  \]
  also finden wir ein $f \in \mer(X)^\ast$ mit $f = e^{f_j} f_j$ auf
  $U_j$. \\
  Damit ist $f$ die gesuchte Lösung von $D$.
\end{proof}

\begin{cor}
  \label{cor:nicht-verschwindende-1-form}
  Sei $X$ eine nicht kompakte Riemannsche Fläche. \\
  Dann existiert ein nicht-verschwindendes $\omega \in \Omega(X)$.
\end{cor}

\begin{proof}
  Sei $g$ eine nicht-konstante meromorphe Funktion auf $X$ und $f \in
  \mer(X)^\ast$ mit $(f) = -(\d[g])$. Diese existiert nach Satz
  \ref{thm:Lösung-Divisor}.\\
  Dann ist $\omega := f \d[g]$ die gesuchte 1-Form.
\end{proof}


%%% Local Variables: 
%%% mode: latex
%%% TeX-master: "../Bachelor"
%%% End: 
