
\section{Der Weierstraßsche Produktsatz}
\label{sec:Weierstrass}

Wie im letzten Kapitel handelt es sich auch hier um den Versuch ein
wohlbekanntes, funktionentheoretisches Resultat auf Riemannsche
Flächen auszuweiten. Diesmal handelt es sich um den
Weierstraßschen Produktsatz, der aussagt, dass es zu einer beliebigen
diskreten Teilmenge $D \subset \C$ eine holomorphe Funktion gibt, die
genau diese Menge als Nullstellenmenge aufweist. Es kann sogar die
Ordnung der Nullstelle an jedem Punkt vorgegeben werden.

Wir werden nun zeigen, dass sich dieses Resultat direkt auf (nicht
kompakte) Riemannsche Flächen überträgt und es sogar noch verschärfen,
denn es kann sogar eine Polstellenmenge mit zugehörigen Ordnungen
angegeben werden und man findet immer eine meromorphe Funktion, die
diese Vorgaben erfüllt. Die Sprache in der dieses Resultat auf
Riemannschen Flächen formuliert wird, ist die der Divisoren. Auch wenn
wir in Kapitel \ref{sec:serre} bereits mit Divisoren gearbeitet haben,
wollen wir hier noch einmal die Definition wiederholen.

\begin{defin}{Divisor}
  Sei $X$ eine Riemannsche Fläche. Ein \init{Divisor} ist eine
  Abbildung $D: X \ra \Z$, so dass für jede kompakte Teilmenge $K
  \subset X$ gilt: $D(x) = 0$ für fast alle $x \in K$. 
  Wir bezeichnen mit $\Div(X)$ die Menge aller Divisoren.
  
  Für eine Funktion $f \in \mer(X)^\times$ definieren wir den zu $f$
  gehörigen Divisor
  \[
  (f): X \ra \Z, \quad x \mapsto \ord_x(f).
  \]
  Da die Null- und Polstellen eine diskrete Teilmenge von $X$ bilden,
  ist dies tatsächlich ein Divisor.
\end{defin}

\begin{defin}
  \label{def:divisor-lsg}
  Sei $X$ eine Riemannsche Fläche und $D \in \Div(X)$. Dann heißt $f
  \in \mer(X)$ eine \emph{Lösung} von $D$, falls $(f) = D$ gilt.
  
  Setzen wir $X_D := \{x \in X: D(x) \geq 0 \}$, so heißt $f \in
  \diff(X_D)$ \emph{schwache Lösung} von $D$, falls für alle $a \in X$
  eine Koordinatenumgebung $(U,z)$ mit $z(a) = 0$ und ein $\psi \in
  \diff(U)$ mit $\psi(a) \neq 0$ existieren, so dass $f = \psi z^k$
  auf $ U \cap X_D$ gilt, wobei $k := D(a)$.
\end{defin}

\begin{lemma}
  Sei $X$ eine Riemannsche Fläche und $D \in \Div(X)$. Seien weiterhin
  $f,g \in \diff(X_D)$ schwache Lösungen von $D$. Dann existiert ein
  $h \in \diff(X)$ mit $h(x) \neq 0$ für alle $x \in X$ und $ f
  = h\cdot g$.
\end{lemma}

\begin{proof}
  Sie $a \in X$ und $(U_a, z_a)$ eine Karte um $a$ mit $z(a) = 0$, so
  dass $f|_{U_a} \equiv \psi_a z^{k_a}$ und $g|_{U_a} \equiv \phi_a
  z^{k_a}$ auf $X_D \cap U_a$ gilt, wobei $k_a := D(a)$. Weiterhin sei
  $U_a$ so klein gewählt, dass $\psi_a(x) \neq 0 \neq \phi_a(x)$ für
  alle $x \in U_a$. Dann ist klarerweise $(U_a)_{a \in X}$ eine
  Überdeckung von $X$. Nun definieren wir auf $U_a$ die Funktion
  \[
  h_a := \frac{\psi_a}{\phi_a} \in \diff(U)
  \]
  und es gilt $f|_{U_a} = h_a g|_{U_a}$. Setzen wir $S := \{ a \in X
  \mid D(a) \neq 0\}$, so ist $S$ diskret und setzen wir weiterhin
  $\tilde X = X \setminus S$, so ist $\tilde X$ offen und es gilt auf $U_a \cap U_b \cap
  \tilde X$
  \[
  h_a|_{U_a \cap U_b \cap \tilde X} = \frac{f|_{U_a \cap U_b \cap
      \tilde X}}{g|_{U_a \cap U_b \cap \tilde X}} = h_b|_{U_a \cap U_b
    \cap \tilde X}.
  \]
  Also finden wir ein $h \in \diff(\tilde X)$ mit $h|_{U_a \cap \tilde
    X} = h_a|_{\tilde X}$. Nun ist aber $h_a$ glatt fortsetzbar auf
  ganz $U_a$ und wir erhalten, dass $h$ glatt auf ganz $X$ fortgesetzt
  werden kann. Weiterhin folgt aus $h_a(x) \neq 0$ für alle $x \in
  U_a$ und $a \in X$, dass auch $h(x) \neq 0$ für alle $x \in X$. Und
  als letztes erahlten wir auch $f = h \cdot g$.
\end{proof}

\begin{lemma}
  \label{lemma:lsg-produkt}
  Sei $X$ eine Riemannsche Fläche. Sei weiterhin $f_i$ 
  eine schwache Lösung des Divisors $D_i$, wobei $i \in \{1,2\}$ gilt. Dann ist
  $f_1 \cdot f_2$ eine schwache Lösung von $D_1 + D_2$.
\end{lemma}

\begin{proof}
  Auf $X_{D_1} \cap X_{D_2}$ ist klar, dass $f_1 \cdot f_2$ die
  gewünschten Eigenschaften erfüllt. Allerdings ist falls ein $a \in
  X$ mit $D(a) := D_1(a) +
  D_2(a) \geq 0$, aber $D_1(a) < 0$ oder $D_2(a) < 0$ existiert, entweder $f_1(a)$ oder
  $f_2(a)$ nicht definiert. Nehmen wir ohne Einschränkung an, dass
  $D_1(a) < 0$ gilt, so können wir eine Karte $(z,U)$ um $a$ finden,
  so dass $\psi, \phi \in \diff(U)$ existieren, die der Definition
  \ref{def:divisor-lsg} genügen. Dann gilt zunächst auf $U \cap
  X_{D_1} \cap X_{D_2}$ die Gleichung $f_1 \cdot f_2 = \psi \cdot \phi
  z^{D_1(a) + D_2(a)}$. Da nun aber $D_1(a) + D_2(a) \geq 0$ gilt,
  lässt sich die rechte Seite der Gleichung glatt auf ganz $U$
  fortsetzen. Also besitzt $f_1 \cdot f_2$ an dieser stelle eine
  hebbare Singularität und wir können das Produkt einfach fortsetzen.
\end{proof}

\begin{defin}
  Sei $c: [0,1] \ra X$ eine glatte, nicht geschlossene Kurve und $X$
  eine Riemannsche Fläche. Dann
  bezeichnet $\partial c$ den Divisor, der durch
  \[
  \partial c(x) :=
  \begin{cases}
    1 & \text{für } x = c(1) \\
    -1 & \text{für } x = c(0) \\
    0 & \text{sonst}
  \end{cases}
  \]
  gegeben ist.
\end{defin}

\begin{lemma}
  \label{lemma:schw-lsg-kurve}
  Sei $X$ eine Riemannsche Fläche und $c: [0,1] \ra X$ eine glatte,
  nicht geschlossene
  Kurve. Sei weiterhin $U \subset X$ eine relativ kompakte, offene
  Umgebung von $c([0,1])$, so existiert eine schwache Lösung $f$ von
  $\partial c$ mit $f|_{X \setminus U} \equiv 1$.
\end{lemma}

\begin{proof}
  Wir gehen zunächst davon aus, dass $U$ eine Kartenumgebung ist und
  bezeichnen mit $z$ die zugehöriger Kartenabbildung. Nun können wir
  ohne Einschränkung annehmen, dass $z(U) = B_1(0)$ gilt. Weiterhin
  identifizieren wir $U$ mit $B_1(0)$. Sei $a := c(0)$ und $b :=
  c(1)$. Dann existiert ein $0 < r < 1$, so dass $a, b \in B_r(0)$
  gilt. Wir definieren
  \[
  \phi: \P^1 \ra \P^1, \quad z \mapsto \frac{z-b}{z-a}.
  \]
  Dann ist $\phi(a) = \infty$ und $\phi(b) = 0$. Wenn wir also $\phi$
  auf $\P^1 \setminus \bar B_r(0)$ einschränken, so können wir $\phi$
  als Abbildung nach $\C^\times$ auffassen. Weiterhin ist $\P^1
  \setminus \bar B_r(0)$ einfach zusammenhängend. Dann können wir aber
  nach \cite[Beispiel 4.18]{For} die Abbildung $\phi$ entlang der
  Überlagerung $\exp: \C \ra \C^\times$ liften. Mit anderen Worten ist
  die Abbildung
  \begin{align}
    \label{eq:log}
  \P^1\setminus \bar B_r(0) \ra \C, z \mapsto \log \left (
    \frac{z-b}{z-a} \right )
  \end{align}
  wohldefiniert und holomorph. Nun können wir die Abbildung aus
  \eqref{eq:log} noch weiter auf $B_1(0) \setminus \bar B_r(0)$
  einschränken.

  Als nächsten Schritt benötigen wir nun noch ein $\psi \in
  \diff(B_1(0))$, das $\psi|_{\bar B_r(0)} \equiv 1$ und $\psi|_{B_1(0)
    \setminus \bar B_{\tilde r}(0)} \equiv 0$ erfüllt, wobei $r <
  \tilde r <1$
  gewählt wurde. Eine stetige Funktion mit diesen Eigenschaften
  existiert auf jeden Fall. Glätten wir diese stetige Funktion, so
  erhalten wir unser $\psi$. Nun können wir ein $f_0 \in \diff(U
  \setminus \{a\})$ durch
  \[
  f_0(z) :=
  \begin{cases}
    \exp \left (\psi \log \middle ( \frac{z-b}{z-a} \middle ) \right
    ) & r < |z| <1 \\
    \frac{z-b}{z-a} & |z| \leq r
  \end{cases}
  \]
  definieren. Dank $\psi$ wird $f_0$ zu einer glatten Abbildung, die
  weiterhin $f_0|_{\partial U} \equiv 1$ erfüllt. Also können wir
  $f_0$ auf ganz $X \setminus \{a\}$ durch 1 fortsetzen. Nach
  Konstruktion ist $f_0$ aber auch eine schwache Lösung von
  $\partial c$, so dass wir die Behauptung in diesem Fall gezeigt
  haben.

  Betrachten wir nun den Fall, dass $U$ keine Koordinatenumgebung
  ist. Dann finden wir eine Unterteilung des Intervalls $0= t_0 <
  t_1 \dots < t_n = 1$ und Karten $(z_j, U_j)$, wobei $j = 1, \dots,
  n$, so dass $c([t_{j-1}, t_j]) \subset U_j \subset U$ und
  $z_j(U_j) = B_1(0)$ gelten. Diese Überdeckung existiert aufgrund
  der Kompaktheit von $c([0,1])$. Es folgt nun aber, dass wir den
  ersten Teil des Beweises jeweils auf $\partial c|_{[t_{j-1},t_j]}$
  anwenden können und erhalten somit schwache Lösungen $f_j$ mit $f_j|_{X
    \setminus U} \equiv 1$.
  Setzen wir $f = f_1 \dots f_n$, so ist $f$ nach Lemma
  \ref{lemma:lsg-produkt} eine schwache Lösung zu
  $\sum_{j=1}^n \partial c|_{[t_{j-1},t_j]} = \partial c$ und sie
  erfüllt $f|_{X \setminus U} \equiv 1$.
\end{proof}

\begin{lemma}
  \label{lemma:schwache-Lösung-Divisor}
  Jeder Divisor $D$ auf einer nicht kompakten Riemannschen Fläche $X$ hat eine
  schwache Lösung.
\end{lemma}

\begin{proof}
  Wir wählen $K_1, K_2, \dots$ kompakte Runge-Teilmengen von $X$ mit
  \begin{enumerate}
  \item $K_j \subset \mathring{K}_{j+1}$ $\forall j \geq 1$ und
  \item $\bigcup_{j \in \N} K_j = X$.
  \end{enumerate}
  Dies ist nach Korollar \ref{cor:ausschöpfung-kompakt} möglich.
  Sei $a_0 \in X \setminus K_j$ und $A_0 \in \Div(X)$
  mit $A_0(a_0) = 1$ und $A_0(x) = 0$, falls $x \neq a_0$. Wir
  behaupten, dass dann eine schwache Lösung $\phi$ von $A_0$ mit $\phi |_{K_j}
  = 1$ existiert. 
  Denn $K_j$ ist eine Runge-Teilmenge, d.h. $K_j = \runge(K_j)$. Also
  liegt $a_0$ in einer nicht relativ kompakten
  Zusammenhangskomponente $U \subset X \setminus K_j$.
  Nun existiert ein $a_1 \in U \setminus K_{j+1}$\footnote{Ansonsten wäre
  $K_{j+1}$ nicht kompakt.} und eine Kurve $c_0: I \ra U$ mit
  $c_0(0)=a_1$ und $c_0(1) = a_0$. 
  Nach Lemma \ref{lemma:schw-lsg-kurve} existiert eine schwache Lösung
  $\phi_0$ des Divisors $\partial c_0$ mit $\phi_0|_{K_j} \equiv 1$. 
  Induktiv sind wir nun in der Lage eine Folge von Punkten $a_\nu \in
  X \setminus K_{j+ \nu}$, $\nu \in \N$, und Kurven in $X \setminus
  K_{j+\nu}$ von $a_{\nu+1}$ nach $a_\nu$ konstruieren. Analog zum
  obigen Verfahren erhalten wir dann wieder schwache Lösungen
  $\phi_\nu$ des Divisors $\partial c_\nu$ mit $\phi_\nu |_{K_{j+\nu}}
  \equiv 1$. 
  Nach Konstruktion gilt $\partial c_\nu = A_\nu - A_{\nu+1}$, wobei
  $A_\nu$ nur für $a_\nu$ nicht verschwindet und an dieser Stelle 1
  ergibt.
  Weiterhin folgt aus Lemma \ref{lemma:lsg-produkt}, dass $\phi_0 \cdot \phi_1 \dots \phi_n$ eine
  schwache Lösung von $A_0 - A_{n+1}$ ist.
  Lassen wir $n$ gegen unendlich gehen, so erhalten wir $\phi :=
  \prod_{\nu = 0}^\infty \phi_\nu$. Dieses Produkt konvergiert, da
  auf jeder kompakten Teilmenge nur endlich viele Faktoren von 1
  verschieden sind. Dies folgt aus Lemma
  \ref{lemma:kompakt-in-ausschöpfung}, denn zu jeder kompakten Menge
  $K$ existiert ein $n \in \N$, so dass $K \subset K_n$. Dann gilt
  aber $\phi_j |_K \equiv 1$ für alle $j > n$.
  Damit ist $\phi$ eine schwache Lösung von $A_0$.
  
  Sei nun $D$ ein beliebiger Divisor auf $X$. Für $\nu \in \N$
  setzen wir
  \[
  D_\nu(x) := \begin{cases} D(x) & x \in K_{\nu+1}\setminus K_\nu\\ 0
    & \text{sonst}\end{cases}, 
  \]
  wobei $K_0$ als leere Menge definiert wird. Wir erhalten dann $D =
  \sum_{\nu = 0}^\infty D_\nu$. 
  Nun ist $D_\nu$ nur an endlich vielen stellen von 0 verschieden und
  kann deshalb als endliche Summe von Divisoren der Form $A_\nu$
  dargestellt werden. Für diese wurde oben eine schwache Lösung
  konstruiert und das (endliche) Produkt dieser Lösungen liefert uns
  eine Lösung $\psi_\nu$ von $D_\nu$ und wir erhalten zusätzlich $\psi_\nu
  |_{K_\nu} \equiv 1$. 
  Setzen wir $\psi := \prod_{\nu=0}^\infty \psi_\nu$, so
  konvergiert das Produkt nach den gleichen Argumenten, wie oben und
  ist eine schwache Lösung von $D$.
\end{proof}

\begin{thm}
  \label{thm:Lösung-Divisor}
  Sei $X$ eine nicht kompakte Riemannsche Fläche und $D \in
  \Div(X)$. Dann existiert ein $f \in \mer(X)^\times$ mit $(f) = D$.
\end{thm}

\begin{proof}
  Wir können $X$ mit relativ kompakten, einfach zusammenhängenden
  Koordinatenumgebungen $(U_i)_{i \in I}$ überdecken. Auf diesen
  finden wir $f_i \in \mer(U_i)^\times$, so dass $(f_i) =
  D|_{U_i}$. Dies ist möglich, da $D$ auf dem relativ kompakten $U_i$
  fast überall verschwindet und $f_i$ deshalb in lokalen Koordinaten
  einfach als rationale Funktion darstellbar ist. 
  Mit dieser Konstruktion haben $f_i$ und $f_j$ dies selben Pol- und
  Nullstellen auf $U_i \cap U_j$, also ist
  \[
  \frac{f_i}{f_j} \in \hol(U_i \cap U_j)^\times \qquad \forall i,j \in I
  \]
  Nach Lemma \ref{lemma:schwache-Lösung-Divisor} existiert eine
  schwache Lösung $\psi$ von $D$ und auf den $U_i$ finden wir $\psi_i
  \in \diff(U_i)$, so dass $\psi = \psi_i \cdot f_i$ und $\psi_i(x)
  \neq 0$ für alle $x \in U_i$. 
  Da $U_i$ einfach zusammenhängend ist und $\psi_i$ nicht
  verschwindet, existiert ein $\phi_i \in \diff(U_i)$ mit $\psi_i =
  e^{\phi_i}$. \\
  Somit erhalten wir
  \[
  \psi = e^{\phi_i} f_i \qquad \text{auf } U_i
  \]
  und
  \[
  e^{\phi_j - \phi_i} = \frac{f_i}{f_j} \in \hol(U_i \cap U_j)^\times
  \]
  Damit ist $\phi_{ij} := \phi_i - \phi_j \in \hol(U_i \cap U_j)$, da
  die $\exp$-Funktion lokal biholomorph ist. Aus der Definition folgt
  auch direkt $\phi_{ij} + \phi_{jk} = \phi_{ik}$ auf $U_i \cap U_j
  \cap U_k$ und damit ist $(\phi_{ij}) \in Z^1(\fu, \hol)$. 
  Nun gilt nach Korollar \ref{cor:h-hol} $H^1(X, \hol) = 0$, also existieren $g_i \in \hol(U_i)$ mit
  $\phi_{ij} = \phi_j - \phi_i = g_j - g_i$ auf $U_i \cap U_j$.
  Eingesetzt folgt
  \[
  e^{g_j} f_j = e^{g_i} f_i \qquad \text{auf } U_i \cap U_j
  \]
  also finden wir ein $f \in \mer(X)^\times$ mit $f = e^{f_j} f_j$ auf
  $U_j$. Damit ist $f$ die gesuchte Lösung von $D$.
\end{proof}

\begin{cor}
  \label{cor:nicht-verschwindende-1-form}
  Sei $X$ eine nicht kompakte Riemannsche Fläche. 
  Dann existiert ein nicht-verschwindendes $\omega \in \Omega(X)$.
\end{cor}

\begin{proof}
  Sei $g$ eine nicht-konstante meromorphe Funktion auf $X$ und $f \in
  \mer(X)^\ast$ mit $(f) = -(\d[g])$. Diese existiert nach Satz
  \ref{thm:Lösung-Divisor}. Dann ist $\omega := f \d[g]$ die gesuchte 1-Form.
\end{proof}


%%% Local Variables: 
%%% mode: latex
%%% TeX-master: "../Bachelor"
%%% End: 
