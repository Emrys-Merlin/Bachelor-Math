
\section{Weyls Lemma}
\label{sec:Weyl}

Wir sind in diesem Kapitel an der Aussage interessiert, dass jede
harmonische Distribution bereits durch eine harmonische Funktion
erzeugt wird. Das ist genau der Inhalt von Weyls Lemma
\ref{thm:weyl}. Bis wir die Aussage jedoch überhaupt formulieren
können, ist einiges an (technischer) Vorarbeit nötig. Zunächst muss
der Begriff einer Distribution eingeführt und Glättungen behandelt
werden, bevor wir zu unserem eigentlichen Hauptresultat gelangen
können.

Die technischen Methoden, die auf dem Weg zu Weyls Lemma entwickelt
werden, spielen im weiteren Verlauf dieser Arbeit keine Rolle
mehr. Wer befürchtet aufgrund der vielen sehr technischen Beweise den
Überblick zu verlieren, kann deshalb getrost die Aussagen von Satz
\ref{thm:weyl} und Korollar \ref{cor:hol-dist} als gegeben
voraussetzen und zum nächsten Kapitel übergehen und später zu den
notwendigen Beweisen zurückspringen.

\begin{defin}
  Sei $X \subset \C$ offen und $f \in \diff(X)$.
  Der Raum der \init{Testfunktionen} auf $X$ ist definiert als
  \[
  \dist(X) := \{ f \in \diff(X) \mid \Supp(f) \subset X \text{ kompakt} \}.
  \]
  Eine Folge $(f_\nu)_{\nu \in \N} \subset \dist(X)$ heißt konvergent
  gegen $f \in \dist(X)$, schreibe $f_\nu \xrightarrow{\dist} f$,
  falls ein kompaktes $K \subset X$ mit $\Supp(f), \Supp(f_\nu)
  \subset K$ für jedes $\nu \in \N$ existiert und für alle $\alpha = (\alpha_1,
  \alpha_2)^T \in \N^2$ die zugehörige Folge $(D^\alpha f_\nu)$ auf
  $K$ gleichmäßig gegen $D^\alpha f$ konvergiert, wobei $D^\alpha = \frac{\partial^{|\alpha|}}{\partial
    x^{\alpha_1} \partial y^{\alpha_2}}$ bezeichnet.
\end{defin}

\begin{defin}[Distribution]
  Der topologische Dualraum, $\dist(X)'$, von $\dist(X)$, wird als
  \init{Raum der Distributionen} bezeichnet. Ein stetiges, lineares
  Funktional $T \in \dist(X)'$ heißt \init{Distribution}.
\end{defin}

\begin{bsp}
  \begin{enumerate}
  \item Jedes $h \in C(X)$ definiert eine Distribution $T_h \in
    \dist(X)'$ durch
    \[
    T_h[f] := \iint_X h(z) f(z) \d[x]\d[y] \qquad z = x+iy \qquad f
    \in \dist(X).
    \]
    $T_h$ ist klarerweise linear und für die Stetigkeit wählen wir ein
    beliebiges $\epsilon > 0$ und eine Folge $(f_n)_{n \in\N} \subset
    \dist(X)$ mit $f_n \xrightarrow{\dist} f \in \dist(X)$. \\
    Also finden wir ein kompaktes $K \subset X$ mit $\Supp(f_n),
    \Supp(f) \subset K$ für jedes $n \in \N$ und ein $N \in \N$, so
    dass
    \[
    \|f_n -f\|_K < \frac{\epsilon}{\|h\|_K |K|} \qquad \forall n \geq N
    \]
    $|K|$ bezeichnet dabei das Volumen von $K$. Dann erhalten wir für
    alle $n \geq N$
    \begin{align*}
      |T_h[f_n] - T_h[f]| & = | T_h[f_n -f]| \\
      & = | \iint_X h(z) (f_n(z) - f(z)) \d[x]\d[y]| \\
      & \leq \iint_K |h(z)| \cdot |f_n(z) - f(z)| \d[x]\d[y]\\
      & \leq \|h\|_K \cdot\|f_n - f\|_K \cdot \iint_K \d[x] \d[y] \\
      & = \|h\|_K \cdot \|f_n - f\|_K \cdot |K| \\
      & < \|h\|_K \frac{\epsilon}{\|h\|_K |K|} |K| \\
      & = \epsilon
    \end{align*}
    und damit die Stetigkeit von $T_h$.
    Weiterhin ist die Abbildung $C(X) \ra \dist(X)', \quad h \mapsto
    T_h$ sogar injektiv. Dies ist eine direkte Folge des 
    Fundamentallemmas der Variationsrechnung (vgl. \cite[§10 HS3]{ForAna}).
  \item Nicht alle Distributionen sind von der obigen Form. Ein
    Gegenbeispiel stellt die Diracsche Delta Distribution dar. Für ein
    $a \in X$ ist sie durch
    \[
    \delta_a[f] := f(a) \qquad \forall f \in \dist(X)
    \]
    definiert. Wir werden im weiteren Verlauf versuchen
    zu zeigen, dass alle für uns interessanten Distributionen
    ausschließlich von der
    ersten Form sind.
  \end{enumerate}
\end{bsp}

\begin{defin}[Ableiten von Distributionen] \index{Ableitung, Distribution}
  Ausgehend von unserem ersten Beispiel und der partiellen Integration
  definieren wir für $\alpha \in \N^2$
  \[
  (D^\alpha T)[f] := (-1)^{|\alpha|} T[D^\alpha f] \qquad \forall f
  \in \dist(X).
  \]
  Nun bedeutet unsere spezielle Wahl der Konvergenz, dass die
  Konvergenz $f_\nu \xrightarrow{\dist} f$ auch die Konvergenz
  $D^\alpha f_\nu \xrightarrow{\dist} D^\alpha f$ impliziert, was zur
  Folge hat, dass $D^\alpha T$ wieder stetig ist, also in $\dist(X)'$ liegt.
\end{defin}

\begin{lemma}
  \label{lemma:dist-glatt}
  Sei $X \subset \C$ offen, $K \subset X$ kompakt und $I \subset \R$ ein
  offenes Intervall. Sei weiterhin $g: X \times I \ra \C$ glatt und
  $\Supp(g) \subset K \times I$. Wählen wir nun ein $T \in \dist(X)'$, so ist die Abbildung
  \[
  t \mapsto T_z[g(z,t)]
  \]
  glatt auf $I$ und genügt
\begin{align}
  \frac{\d}{\d[t]} T_z[g(z,t)] = T_z \left [ \frac{\partial
      g(z,t)}{\partial t} \right ]. \label{eq:dist-ableitung}
  \end{align}
  $T_z$ soll dabei nur verdeutlichen dass $T$ nur auf $g(z,t)$ als
  Variable von $z$ operiert; $t$ bleibt als Parameter unberührt.
\end{lemma}

\begin{proof}
  Es genügt die Formel \eqref{eq:dist-ableitung} zu zeigen, da dann
  die Glattheit von $T_z[g(z,t)]$ aus der Glattheit von $g$ folgt. 
  Es gilt
  \begin{align*}
    \frac{\d}{\d[t]} T_z[g(z,t)] & = \lim_{h \ra 0} \frac1h (
    T_z[g(z,t+h)] - T_z[g(z,t)] ) \\
    & = \lim_{h \ra 0} T_z \left [ \frac1h (g(z,t+h) - g(z,t)) \right ].
  \end{align*}
  Für festes $t \in I$ und $h \in \R^\times$ klein genug, ist $f_h(z)
  := \frac1h ( g(z, t+h) - g(z,t))$ wohldefiniert und in
  $\dist(X)$. Weiterhin gilt $f_n \xrightarrow{\dist}
  \frac{\partial g(\cdot, t)}{\partial t}$, da $g$ glatt ist. 
  Aus der Stetigkeit von $T$ folgt damit
  \[
  \lim_{h\ra 0} T_z[f_h] = T_z[\lim_{h \ra 0} f_h] = T_z\left[
    \frac{\partial g(z,t)}{\partial t} \right].
  \]
\end{proof}

\begin{lemma}
  \label{lemma:dist-int}
  Seien $X,Y \subset \C$ offen, $K \subset X$ bzw. $L \subset Y$
  kompakt. Sei weiterhin $g: X \times Y \ra \C$ glatt mit $\Supp(g)
  \subset K \times L$. 
  Dann gilt für beliebige $T \in \dist(X)'$
  \[
  T_w\left [ \iint_Y g(w,z) \d[x]\d[y] \right ] = \iint_Y T_w [
  g(w,z)] \d[x]\d[y],
  \]
  wobei $z = x + iy$ gesetzt wurde.
\end{lemma}

\begin{proof}
  Nach Lemma \ref{lemma:dist-glatt} ist $T_w[g(w,z)]$ glatt und es
  gilt $\Supp(T_w[g(w,z)]) \subset L$, denn für $z \notin L$ ist
  $g(\cdot, z) \equiv 0$ und damit $T_w[(g(w,z)] = 0$. 
  Also existiert das Integral $\iint_Y T_w[g(w,z)] \d[x]\d[y]$,
  insbesondere lässt es sich durch Riemann-Summen approximieren. 
  Dazu wählen wir ein Rechteck $R \subset \C$, so dass $L \subset R$
  und die Seiten von $R$ parallel zu den Koordinatenachsen
  verlaufen. 
  Wir können $g$ durch 0 auf $K \times R$ fortsetzen, d.h. wir können
  es als Funktion auf $K \times R$ auffassen. 
  Als nächstes unterteilen wir für jedes $n \in \N$ $R$ in $n^2$
  Teilrechtecke $R_{n\nu}$, $\nu = 1, \dots, n^2$. Diese Teilrechtecke
  werden dabei alle gleich groß gewählt. Sei weiterhin $z_{n\nu}
  \in R_{n \nu}$ und $F = |R|$. 
  Dann konvergiert die Funktionenfolge
  \[
  G_n(w) := \frac{F}{n^2} \sum_{\nu=1}^{n^2} g(w, z_{n\nu})
  \]
  gegen $\iint_Y g(w,z) \d[x]\d[y]$. Nun ist $g$ gleichmäßig stetig,
  d.h. zu jedem $\epsilon > 0$ finden wir ein $\delta >0$, so dass
  \[
  | g(w,z) - g(\tilde w, \tilde z) | < \frac{\epsilon}{F}
  \]
  für beliebige $\| (w,z) - (\tilde w, \tilde z) \| < \delta$. Wir
  finden also ein $N \in \N$, so dass $|z_{n_\nu} - z| < \delta$ für
  alle $z \in R_{n_\nu}$ und für alle $n \geq N$. Insgesamt erhalten
  wir
  \begin{align*}
    |G_n(w) - \iint_R g(w,z) \d[x]\d[y]| & = \left | \sum_{\nu = 1}^{n^2}
    \middle ( \frac{F}{n^2} g(w, z_{n_\nu}) - \iint_{R_{n_\nu}} g(w,z)
      \d[x]\d[y] \middle ) \right | \\
    & \leq \sum_{\nu=1}^{n^2} \left | \iint_{R_{n_\nu}}
      (g(w,z_{n_\nu}) - g(w,z)) \d[x]\d[y] \right | \\
    & \leq \sum_{\nu=1}^{n^2} \iint_{R_{n_\nu}} |g(w, z_{n_\nu}) -
    g(w,z)| \d[x]\d[y] \\
    & \leq \sum_{\nu=1}^{n^2} \iint_{R_{n_\nu}} \frac{\epsilon}{F}
    \d[x] \d[y] \\
    & = \epsilon.
  \end{align*}
  Also konvergiert $G_n$ gleichmäßig gegen $\iint_R g(\cdot, z)
  \d[x]\d[y]$. Die gleiche Argumentation kann für beliebiege
  Ableitungen $D^\alpha G_n$ durchgeführt werden und wir erhalten,
  dass
  \[
  G_n \xrightarrow{\dist} \iint_Y g(\cdot, z) \d[x]\d[y]
  \]
  gilt. Also folgt aus der Stetigkeit von $T$
  \[
  T_w \left [ \iint_Y g(w, z) \d[x]\d[y] \right ] = \lim_{n \ra
    \infty} T[G_n] = \iint_Y T_w[g(w,z)] \d[x]\d[y].
  \]
\end{proof}

\subsection{Glättung von Funktionen}
\label{sec:Glättung}

\begin{defin}[Glättungskern]
  $\rho \in \dist(\C)$ heißt \init{Glättungskern}, falls die folgenden
  Eigenschaften erfüllt sind:
  \begin{enumerate}
  \item Es gilt $\Supp(\rho) \subset \overline{B_1(0)}$.
  \item $\rho$ ist rotationssymmetrisch zum Ursprung.
  \item Es gilt $\rho \geq 0$\footnote{Insbesondere ist $\rho$ also reellwertig.}.
  \item Es gilt $\iint_\C \rho(z) \d[x]\d[y] = 1$, wobei $z = x + iy$
    gesetzt wurde.
  \end{enumerate}
  Für $\epsilon > 0$ und $z \in \C$ definieren wir
  \[
  \rho_\epsilon(z) = \frac{1}{\epsilon^2} \rho\left(
    \frac{z}{\epsilon} \right )
  \]
  Dann ist $\Supp(\rho_\epsilon) \subset \overline{B_\epsilon(0)}$ und
  $\iint_\C \rho_\epsilon(z) \d[x]\d[y] = 1$.
\end{defin}


\begin{bsp}
  Das Standardbeispiel für einen Glättungskern stellt die Funktion
  \[
  \phi(z) :=
  \begin{cases}
    \exp \left ( - \frac{1}{1 - |z|^2} \right ) & |z| < 1 \\
    0 & \text{sonst}
  \end{cases}
  \]
  dar. 
  Man kann zeigen, dass diese glatt ist. Weiterhin ist ihr Integral
  echt größer 0, weshalb sie auch so normiert werden kann, dass
  Bedingung 4 erfüllt ist.
\end{bsp}

\begin{defin}
  Sei $U \subset \C$ offen und $\epsilon > 0$. Dann definieren wir
  $U^{(\epsilon)} := \{z \in U | \bar B_\epsilon(z) \subset U \}$ und
  für $f \in C(U)$ setzen wir
  \[
  \sm_\epsilon f : U^{(\epsilon)} \ra \C, \quad \sm_\epsilon f(w) :=
  \iint_U \rho_\epsilon(w- z) f(z) \d[x]\d[y] \qquad z = x + iy
  \]
  $\sm_\epsilon f$ wird als \init{Glättung} von $f$ bezeichnet. 
  Aus der Leibniz-Regel folgt, dass Integration und Differentiation
  kommutiert, also ist $\sm_\epsilon f \in \diff(U^{(\epsilon)})$, was
  auch den Namen "`Glättung"' erklärt.
\end{defin}

\begin{lemma}
  \label{lemma:glättung-eigenschaften}
  Sei $U \subset \C$ offen, $f \in \diff(U)$ und $\epsilon > 0 $. Dann
  gelten:
  \begin{enumerate}
  \item Für jedes $\alpha \in \N^2$ ist
    \[
    D^\alpha( \sm_\epsilon f) =
    \sm_\epsilon (D^\alpha f).
    \]
  \item Falls $z \in U^{(\epsilon)}$ und $f$ harmonisch auf
    $B_\epsilon(z)$ ist, so gilt
    \[
    \sm_\epsilon f(z) = f(z).
    \]
  \end{enumerate}
\end{lemma}


\begin{proof}
  \begin{enumerate}
  \item Wir berechnen
    \begin{align*}
      D^\alpha (\sm_\epsilon f)(w) & = D^\alpha \iint_U \rho_\epsilon(w
      - z) f(z) \d[x] \d[y] \\
      & \stackrel{\tag{1}}{=} \iint_U D_w^\alpha \rho(w - z) f(z)
      \d[x] \d[y] \label{eq:glättung-leibniz}\\
      & \stackrel{\tag{2}}{=} \iint_U (-1)^{|\alpha|} D_z^\alpha
      \rho(w-z) f(z) \d[x] \d[y] \label{eq:glättung-index}\\
      & \stackrel{\tag{3}}{=} \iint_U (-1)^{2|\alpha|}
      \rho_\epsilon(w-z) D^\alpha_z f(z)
      \d[x]\d[y] \label{eq:glättung-partiell}\\
      & = (\sm_\epsilon D^\alpha f)(z).
    \end{align*}
    \eqref{eq:glättung-leibniz} folgt dabei aus der Leibnizregel,
    \eqref{eq:glättung-index} erhält man daraus, dass $D^\alpha$
    zunächst auf $w$ wirkt (deshalb auch der Index), wir wollen aber,
    dass es auf $z$ wirkt um die Partielle Integration in
    \eqref{eq:glättung-partiell} anwenden zu können. Dies ist möglich
    in dem man das Vorzeichen, das durch die Kettenregel entsteht
    ausgleicht.
  \item Wenn $f$ harmonisch auf $B_\epsilon(z)$ ist, können wir die
    Mittelwerteigenschaft \ref{cor:harm-darstellung} anwenden und erhalten
    \[
    f(z) = \frac{1}{2\pi} \int_0^{2\pi}f(z + re^{i\phi}) \d[\phi]
    \qquad \forall r \in ]0, \epsilon[.
    \]
    Die folgende Rechnung liefert dann 
    \begin{align}
      (\sm_\epsilon f)(w) & = \iint_u \rho_\epsilon(w- z) f(z)
      \d[x]\d[y] \nonumber\\
      & = \iint_{B_\epsilon(z)} \rho_\epsilon(z -w ) f(z) \d[x]\d[y]
      \nonumber \\
      & = \iint_{B_\epsilon(0)} \rho_\epsilon(z) f(z + w) \d[x]\d[y]
      \nonumber \\
      & = \int_0^{2\pi} \int_0^\epsilon \rho_\epsilon(re^{i\phi}) f(w +
       re^{i\phi}) r \d[r] \d[\phi] \nonumber \\
      & = \int_0^\epsilon \tilde \rho_\epsilon(r) r\d[r] \cdot 2 \pi
      f(w) \nonumber \\
      & = f(w). \label{eq:glättung-norm} 
    \end{align}
    \eqref{eq:glättung-norm} gilt dabei, aufgrund der Gleichung
    \begin{align*}
      1 & = \iint_{B_\epsilon(0)}\rho_\epsilon(z) \d[x]\d[y] \\
      & = \int_0^{2\pi}\int_0^{\epsilon} \tilde \rho_\epsilon(r) r
      \d[r] \d[\phi] \\
      & = 2 \pi \int_0^\epsilon \tilde \rho_\epsilon(r) r \d[r].
    \end{align*}
  \end{enumerate}
\end{proof}

\begin{thm}[Weyls Lemma]
  \label{thm:weyl}
  Sei $U \subset \C$ offen und $T \in \dist(U)'$ mit $\Delta T \equiv
  0$. 
  Dann existiert ein $h \in \diff(U)$ mit $\Delta h = 0$, so dass
  \[T[f] = \iint_{U^{(\epsilon)}} h(z) f(z) \d[x]\d[y] \qquad \forall f \in \dist(U)
  \]
  gilt.
\end{thm}

\begin{proof}
  Sei $\epsilon > 0$. Für $z \in U^{(\epsilon)}$ hat $w \mapsto
  \rho_\epsilon(w - z)$ kompakten Träger in $U$. Damit ist
  \[
  h_\epsilon : U^{(\epsilon)} \ra \C, \quad h_\epsilon(z) :=
  T_w[\rho_\epsilon(w - z)]
  \]
  wohldefiniert. Nach Lemma \ref{lemma:dist-glatt} ist $h_\epsilon \in
  \diff(U^{(\epsilon)})$. 
  Sei nun $f \in \dist(U)$ mit $\Supp(f) \subset U^{(\epsilon)}$. Wir
  wollen zunächst zeigen, dass $\sm_\epsilon f$ kompakten Träger in
  $U$ hat. Denn für $w \in U$ mit
  $B_\epsilon(w) \cap U \neq \varnothing$ ist $f|_{B_\epsilon(w) \cap U}
  \equiv 0$ und es folgt $\sm_\epsilon f(w) = 0$. Also gilt
  \[
  \Supp \sm_\epsilon f \subset \{ z \in U | B_\epsilon(w) \cap U \neq
  \varnothing\} =: M.
  \]
  Nun ist $\Supp(f)$ kompakt und damit
  beschränkt. Sei $S$ das zugehörige Supremum. Für ein beliebiges $w
  \in M$, finden wir nun ein $z \in B_\epsilon(w) \cap U$ und wir
  erhalten
  \[
  |w| \leq |w-z| + |z| < S + \epsilon.
  \]
  Also ist $M$ und damit $\Supp \sm_\epsilon f$ beschränkt. Da $\Supp
  \sm_\epsilon f$ auch abgeschlossen ist, ist es sogar kompakt und wir können folgende
  Rechnung durchführen
  \begin{align}
    T[\sm_\epsilon f] & = T_w \left [ \iint_U \rho_\epsilon(w - z)
      f(z) \d[x]\d[y] \right ] \nonumber \\
    & = \iint_U T_w[\rho_\epsilon(w-z)] f(z)
    \d[x]\d[y] \label{eq:weyl} \\
    & = \iint_U h_\epsilon(z) f(z) \d[x]\d[y]. \nonumber
  \end{align}
  Als nächstes wollen wir zeigen, dass $T[f] = T[\sm_\epsilon f]$
  gilt. Wir können $f$ auch als Funktion auf $\C$ auffassen, in dem
  wir es durch 0 fortsetzen. Dann liefert uns \cite[Kor. 13.3]{For} die
  Existenz von einem $\psi \in \diff(\C)$ mit $\Delta \psi = f$. 
  Mit anderen Worten ist $\psi$ harmonisch auf $V:= \C \setminus \Supp(f)$. Also
  liefert uns Lemma \ref{lemma:glättung-eigenschaften} $\psi|_{V^{(\epsilon)}}
  \equiv \sm_\epsilon \psi|_{V^{(\epsilon)}}$. Wir setzen $\phi := \psi -
  \sm_\epsilon\psi$. Es folgt $\Supp(\phi) \subset V^c = \Supp(f)
  \subset U$, d.h. $\phi$ hat kompakten Träger in $U$ und es gilt
  \[
  \Delta \phi = \Delta \psi - \sm_\epsilon \Delta f = f - \sm_\epsilon
  f.
  \]
  Da nun aber $\Delta T = 0$, folgt $T[\Delta \phi] = 0$ und
  schlußendlich
  \begin{align*}
    T[f] & = T[\sm_\epsilon f + \Delta \phi] \\
    & = T[\sm_\epsilon f] \\
    & = \iint_{U^{(\epsilon)}} h_\epsilon(z) f(z) \d[x]\d[y].
  \end{align*}
  Für beliebige $0 < \epsilon', \epsilon'' \leq \epsilon$ gilt
  $U^{(\epsilon)} \subset U^{(\epsilon')}, U^{(\epsilon'')}$ und es
  folgt
  \[
  T[f] = \iint_{U^{(\epsilon)}}h_{\epsilon'}(z) f(z) \d[x]\d[y] =
  \iint_{U^{(\epsilon'')}}(z) f(z) \d[x] \d[y].
  \]
  Also folgt aus dem Fundamentallemma der Variationsrechnung
  $h_{\epsilon'} \equiv h_{\epsilon''}$ auf $U^{(\epsilon)}$,
  d.h. $h_{\epsilon'}$ wird für $\epsilon' < \epsilon$ konstant auf
  $U^{(\epsilon)}$ und da $\epsilon > 0$ beliebig gewählt war, finden
  wir ein $h \in \diff(U)$ mit den gewünschten Eigenschaften.
\end{proof}

\begin{cor}
  \label{cor:hol-dist}
  Sei $U \subset \C$ offen und $T \in \dist(U)'$ mit $\frac{\partial
    T}{\partial \bar z} = 0$. 
  Dann existiert ein $h \in \hol(U)$, so dass
  \[
  T[f] = \iint_U h(z)f(z) \d[x]\d[y] \qquad \forall f \in \dist(U)'
  \]
  gilt.
\end{cor}

\begin{proof}
  Aus $\frac{\partial T}{\partial \bar z} = 0$ folgt
  \[
  \Delta T  = 4 \frac{\partial}{\partial z} \left (
    \frac{\partial}{\partial \bar z} T \right ) = 0
  \]
  Also existiert aufgrund von Satz \ref{thm:weyl} ein $h \in
  \diff(U)$, so dass $T = T_h$ und erneut aus dem Fundamentallemma der
  Variationsrechnung folgt
  \[
  \frac{\partial h}{\partial \bar z} = 0
  \]
  und damit $h \in \hol(U)$.
\end{proof}



%%% Local Variables: 
%%% mode: latex
%%% TeX-master: "../Bachelor"
%%% End: 
