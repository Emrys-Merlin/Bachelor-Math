\message{ !name(Serre.tex)}
\message{ !name(Serre.tex) !offset(-2) }

\section{Der Serresche Dualitätssatz}
\label{sec:serre}

\begin{defin}
  \label{def:res}
  % TODO: Referenz
  Sei $X$ eine kompakte Riemannsche Fläche. Nach (15.14) ist
  \[
  0 \ra \Omega \ra \diff^{1,0} \xrightarrow{\d} \diff^{(2)} \ra 0
  \]
  exakt und es gilt $H^1(X, \Omega) \equiv
  \quot{\diff^{(2)}(X)}{\d[\diff^{1,0}(X)]}$. Sei $\zeta \in H^1(X,
  \Omega)$ und $\omega \in \diff^{(2)}(X)$ ein Repräsentant von
  $\zeta$. Setzen wir
  \[
  \res(\zeta) := \frac{1}{2\pi i} \iint_X \omega
  \]
  so ist diese Definition aufgrund von (10.20) repräsentantenunabhängig.
\end{defin}

\begin{defin}[Mittag-Leffler-Verteilung von Differentialformen]
  \label{def:mlv}
  Sei $X$ eine Riemannsche Fläche und $\mer^{(1)}$ die Garbe der
  meromorphen 1-Formen auf $X$. Wählen wir eine offene Überdeckung
  $\fu = (U_i)_{i \in I}$ von $X$, so nennen wir
  \[
  \mu = (\omega_i) \in C^0( \fu, \mer^{(1)})
  \]
  eine \init{Mittag-Leffler-Verteilung}, falls für beliebige $i,j \in
  I$ die 1-Form $\omega_j - \omega_i$ auf $U_i \cap U_j$ holomorph
  ist, d.h. $ \delta \mu \in Z^1(\fu, \Omega)$.

  Wir bezeichnen mit $[\delta \mu] \in H^1(X, \Omega)$ die
  Kohomologieklasse von $\delta \mu$.

  Weiterhin definieren wir zu $a \in X$
  \[
  \res_a(\mu) := \res_a(\omega_i),
  \]
  wobei $a \in U_i$. Falls $a \in U_i \cap U_j$, so gilt
  $\res_a(\omega_i) = \res_a(\omega_j)$, denn $\omega_j - \omega_i$
  ist holomorph. Ist $X$ kompakt, so ist $\res_a(\mu) = 0$ für fast
  alle $a \in X$ und wir können
  \[
  \res(\mu) := \sum_{a \in X} \res_a(\mu)
  \]
  definieren.
\end{defin}

\begin{thm}
  Mit der Notation aus Definition \ref{def:res} und \ref{def:mlv} gilt
  \[
  \res(\mu) = \res([\delta \mu])
  \]
\end{thm}

\begin{proof}
  Um $\res([\delta \mu])$ zu berechnen, konstruieren wir $H^1(X,
  \Omega) \equiv \quot{\diff^{(2)}(X)}{\d[\diff^{1,0}(X)]}$
  explizit. Da $\delta \mu = (\omega_j - \omega_i) \in Z^1(\fu,
  \Omega) \subseteq Z^1(\fu, \diff^{1,0})$ und $H^1(X, \diff^{1,0}) =
  0$ (Kapitel 12) gilt, finden wir ein $(\sgima_i) \in C^0(\fu,
  % TODO: Referenz Kapitel 12
  \diff^{1,0})$ mit
  \[
  \omega_j - \omega_i \cong \sigma_j - \sigma_i \qquad \text{auf } U_i
  \cap U_j.
  \]
  Nun ist jede holomorphe 1-Form geschlossen, d.h. $\d[(\omega_j -
  \omega_i)] = 0$ und wir erhalten $\d[\sigma_i] = \d[\sigma_j]$ auf
  $U_i \cap U_j$. Also finden wir ein $ \tau \in \diff^{(2)}(X)$ mit
  $\tau|_{U_i} \cong \d[\sigma_i]$. Dieses $\tau$ ist der Repräsentant
  von $[\delta \mu]$, also gilt
  \[
  \res([\delta \mu]) = \frac{1}{2\pi i} \iint_X \tau
  \]
  Seien nun $a_1, \dots, a_n \in X$ die endlich vielen Pole von $\mu$
  und $X' = X \setminus \{a_1, \dots, a_n\}$. Auf $X' \cap U_i \cap
  U_j$ gilt $\sigma_i - \omega_i \cong \sigma_j - \omega_j$. Erneut
  verwenden wir die Garbeneigenschaften und finden ein $\sigma \in
  \diff^{1,0}(X')$ mit $\sigma|_{X' \cap U_i} \cong \sigma_i -
  \omega_i$. Wir erhalten
  \[
  \d[\sigma] \equiv \d[\sigma_i] - \underbrace{\d[\omega_i]}_{= 0} \equiv \tau
  \qquad \text{ auf } X' \cap U_i.
  \]
  Und damit gilt $\d[\sigma] \equiv \tau$ auf $X'$. Als nächstes
  wählen wir zu jedem $a_k$ ein $i(k) \in I$, so dass $a_k \in
  U_{i(k)}$ gilt. Weiterhin wählen wird Koordinatenumgebungen $(V_k,
  z_k)$ mit folgenden Eigenschaften
  \begin{enumerate}
  \item Es gelten $V_k \subset U_{i(k)}$ und $z_k(a_k) = 0$,
  \item es ist $V_k \cap V_j = \varnothing$ für alle $k \neq j$ und
  \item $z_k(V_k) \subset \C$ ist eine Kreisscheibe.
  \end{enumerate}
  Wählen wir $f_k \in \diff(X)$ mit $\Supp(f_k) \subset V_k$ und so
  dass eine offene Umgebung $V_k' \subset V_k$ von $a_k$ mit
  $f_k|_{V_k'} \equiv 1$, so können wir $g:= 1 - (f_1 + \dots + f_k)$
  definieren. Dies erlaubt uns $g \cdot \sigma$ auf ganz $X$
  fortzusetzen, denn $g|_{V_k'} \equiv 0$. Also liegt $g \sigma \in
  \diff^{1,0}(X)$. Nach (10.20) gilt
  \begin{align}
  \iint_X \d[(g \sigma)] = 0. \label{eq:g-sigma}
  \end{align}
  Auf $V_k' \setminus \{a_k\}$ erhalten wir
  \[
  \d[(f_k \sigma)] = \d[\sigma] = \d[\sigma_{i(k)} - \omega_{i(k)}] =
  \d[\sigma_{i(k)}]
  \]
  Nun ist aber $\sigma_i \in \diff^{1,0}(U_i)$, also kann $\d[(f_k
  \sigma)]$ glatt auf $a_k$ fortgesetzt werden. Da $f_k\sigma$ auf $X'
  \setminus \Supp(f_k)$ verschwindet, können wir $\d[f_k \sigma) \in
  \diff^{(2)}(X)$ auffassen. Wir erhalten die folgende Gleichung
  \[
  \tau = \d[1 \cdot \sigma] = \d[(g\sigma)] + \sum_{k=1}^n \d[(f_k
  \sigma)]
  \]
  Unter Ausnutzung von \eqref{eq:g-sigma} erhalten wir
  \[
  \iint_X \tau = \sum_{k=1}^n \iint_X \d[f_k \sigma] = \sum_{k=1}^n
  \iint_{V_k} \d[(f_k\sigma_{i(k)} - f_k \omega_{i(k)} )]
  \]
  Erneut wegen (10.20) gilt $\iint_{V_k} \d[f_k \sigma_{i(k)}] = $ und
  analog zum Beweis von (10.21) folgt
  \[
  \iint_{V_k} \d[(f_k \omega_{i(k)})] = - 2\pi i
  \res_{a_k}(\omega_{i(k)})
  \]
  Bauen wir alles zusammen, so erhalten wir
  \[
  \res([\delta \mu]) = \frac{1}{2\pi i} \iint_X \tau = \sum_{k=1}^n
  \res_{a_k}(\omega_{i(k)}) = \res(\mu)
  \]
  

\end{proof}

%%% Local Variables: 
%%% mode: latex
%%% TeX-master: "../Bachelor"
%%% End: 


\message{ !name(Serre.tex) !offset(-147) }
