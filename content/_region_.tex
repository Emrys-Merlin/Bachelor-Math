\message{ !name(Weierstrass.tex)}
\message{ !name(Weierstrass.tex) !offset(-2) }

\section{Der Weierstrasssche Produktsatz}
\label{sec:Weierstrass}

\begin{defin}{Divisor}
  Sei $X$ eine Riemannsche Fläche. Ein \init{Divisor} ist eine
  Abbildung $D: X \ra \Z$, so dass für jede kompakte Teilmenge $K
  \subset X$ gilt: $D(x) = 0$ für fast alle $x \in K$. \\
  Wir bezeichnen mit $Div(X)$ die Menge aller Divisoren. \\
  Für eine Funktion $f \in \mer(X)^\ast$ definieren wir den zu $f$
  gehörigen Divisor $(f): X \ra \Z, \quad x \mapsto \ord_x(f)$. Da die
  Null- und Polstellen eine diskrete Teilmenge von $X$ bilden, ist
  dies tatsächlich ein Divisor.
\end{defin}

\begin{defin}
  Sei $X$ eine Riemannsche Fläche und $D \in Div(X)$. Dann heißt $f
  \in \mer(X)$ eine \emph{Lösung} von $D$, falls $(f) = D$.
\end{defin}



%%% Local Variables: 
%%% mode: latex
%%% TeX-master: "../Bachelor"
%%% End: 

\message{ !name(Weierstrass.tex) !offset(-30) }
